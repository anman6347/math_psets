\pgsc{代数学}{chocolate}{algebra}

\begin{nmprob}
ペアノシステムの数学的帰納法.
\pbenumex[peano]{
次の条件 (N1) から (N3) を満たす, 集合 $\bm{N}$, その一つの元 $0$, および写像 $\sigma :\bm{N} \to \bm{N}$ の対 $\langle \bm{N}, 0, \sigma \rangle$
をペアノシステムという.
\begin{description}\setlength{\leftskip}{16pt}
    \item[\rm (N1)] $\sigma :\bm{N} \to \bm{N}$ は単射である.
    \item[\rm (N2)] $0 \notin \sigma(\bm{N})$
    \item[\rm (N3)] $S \subset \bm{N}$ のとき, $S$ が次の二条件を満たせば $S = \bm{N}$ となる.
    \begin{align}
        0 \in S\\
        \sigma(S) \subset S
    \end{align}
    \setcounter{equation}{0}
\end{description}
特に, (N3) を数学的帰納法の公理とよぶ. また, $\sigma$ は後継者写像といい, $n$ に対して $\sigma(n)$ はその後継者とよばれる.
}{
\item 任意の $n \in \bm{N}$ に対して, $n \in S \implies \sigma(n) \in S$ が成り立つとすると, $\sigma(S) \subset S$ が成り立つことを示せ. ただし, $S \subset \bm{N}$ とする.
\item 任意の $n \in \bm{N}$ に対して, $\sigma(n) \neq n$ を示せ.
\item 任意の $n \in \bm{N}\backslash \{0\}$ に対して, $\exists m\in \bm{N},\ \sigma (m) = n$ を示せ. 
}{
\item $y \in \sigma (S)$ とすると, $\sigma (S)$ の定義から, $\langle x, y\rangle \in \sigma$ となる $x \in S$ が存在する.
今, $\forall n \in \bm{N},\ n \in S \implies \sigma (n) \in S$ が成り立つので, $\sigma (x) \in S$ となる. また, $\sigma$ は写像より $y = \sigma (x)$ となるから, $y \in S$ となる.
よって, $\sigma (S) \subset S$ となる.
\item $S = \{ n \in \bm{N}\ |\  \sigma (n) \neq n \}$ とする. まず, (N2) より $0 \in S$ となる.
次に, 任意の $n \in \bm{N}$ を一つとると, $n \in S$ ならば $\sigma (n) \neq n$ より, (N1) から $\sigma (\sigma (n)) \neq \sigma(n)$ となり, $\sigma (n) \in S$ となる.
よって, (1) より, $\sigma (S) \subset S$ となるから, (N3) より $S = \bm{N}$ となる. これは, $\forall n \in \bm{N},\ \sigma (n) \neq n$ を意味する.
\footnote[1]{$P(0)$が真, かつ$P(n)$が真ならば$P(\sigma (n))$ が真であれば $S = \{ n\in \bm{N}\ |\ P(n)\}$ とすることで N3 から $S = \bm{N}$ を示す方法はよく使われる.\vspace{30pt}}
\item $S = \{0\} \cup \sigma (\bm{N})$ とすると, 明らかに $0 \in S$ となる. また, 任意の $n \in \bm{N}$ に対して, $n \in S$ とすると, $\sigma (n) \in \sigma (\bm{N}) \subset S$ となる.
よって, (N3) より, $S = \bm{N}$ となる. ゆえに $\bm{N}$ の $0$ 以外の元は $\sigma (\bm{N})$ に含まれる.
以上より, 任意の $n \in \bm{N}\backslash \{0\}$ に対して, $\exists m\in \bm{N},\ \sigma (m) = n$ となる. ちなみに, (N2) とから, $\bm{N}\backslash \{0\} = \sigma (\bm{N})$ が成り立つ(外延性から示せる)
} 
\end{nmprob}



\begin{nmprob}
ペアノシステムの一意性.
\renewcommand{\labelenumii}{(\Roman{enumii})}
\pbenum{
\item $X$ を一つの集合とし, $X$ の一つの元 $x_0$ と写像 $\phi :X \to X$ とが与えられたとする.
このとき, 次の二条件を満たすような写像 $f :\bm{N} \to X$ がただ一つ存在することを示せ.

\renewcommand{\theequation}{\roman{equation}}
\begin{align}
& f(0) = x_0\\
& \forall n \in \bm{N},\ f(\sigma (n)) = \phi (f(n))
\end{align}
\setcounter{equation}{0}
\item $\langle \bm{N}, 0, \sigma \rangle$ と $\langle \bm{N^{'}}, 0^{'}, \sigma^{'} \rangle$ が共にペアノシステムであるとき,
$\bm{N}$ から $\bm{N^{'}}$ への全単射 $f$ で, $f(0) = 0^{'}$ かつ, 任意の $n \in \bm{N}$ に対して $f(\sigma (n)) = \sigma^{'}(f(n))$ となるもの
が一意的に存在することを示せ.
}{
\item
\begin{enumerate}
\item
まず, $f$ の一意性から示す.\\
相違な写像 $f: \bm{N} \to X,\ f^{'}: \bm{N} \to X$ が共に条件 (\rnum{1}), (\rnum{2}) を満たすとする.
ここで, $S = \{ n \in \bm{N}\ |\ f(n) = f^{'}(n)\}$ とすると, (\rnum{1}) より $0 \in S$ である.
また, 任意の $n \in \bm{N}$ に対して, $n \in S$ とすると, (\rnum{2}) より $f(\sigma (n)) = \phi (f(n)) = \phi (f^{'}(n)) = f^{'}(\sigma (n))$ となるから, $\sigma (n) \in S$ となるから
問\ref{sec:algebra}-\ref{pname:peano}\ (1) より, $S \subset \sigma (S)$ となる. よって, 数学的帰納法の公理より, $S = \bm{N}$ となる. よって, $f = f^{'}$ となり, 矛盾. よって, $f$ が存在するとすれば一意である.
\item
次に, $f$ の存在を示す.\\
まず, 次の二条件を満たす $\bm{N} \times X$ の部分集合 $R$ を考える.
\begin{align}
    \langle & 0, x_0\rangle \in R\\
    \forall & (n\in \bm{N})[\langle n, x\rangle \in R \implies \langle \sigma (n), \phi (x) \rangle \in R]
\end{align}
\setcounter{equation}{0}
条件(1), (2) を満たす集合全体を $\mathcal{F}$ とすると, $\bm{N} \times X \in \mathcal{F}$ より $\mathcal{F} \neq \emptyset$ である.
よって, $\bigcap_{R \in \mathcal{F}} R$ を考えることができる.\\
ここで, 
\begin{align*}
f = \bigcap_{R \in \mathcal{F}} R    
\end{align*}
としたとき, この $f$ が $\bm{N}$ から $X$ への写像であり, 二条件 (\rnum{1}), (\rnum{2}) を満たすことを示す.\\
まず, $f$ が $\bm{N}$ から $X$ への写像であることを示すために, $T = \{ n \in \bm{N}\ |\ \langle n, x\rangle \in f \text{なる} x \text{が唯一つ存在する}\}$ を考える.

\begin{enumerate}
\item $0 \in T$ を示す.
\item $\forall n \in \bm{N},\ n \in T \implies \sigma (n) \in T$ を示す.                                                                                                                                                                                                                                                                                                                                                                                                                                                                                                                       
\end{enumerate}
\end{enumerate}

}
\newpage
\hspace{-3zw}{\color{orangered}●●\ 前問題の補足\ ●●}
\end{nmprob}



\begin{nmprob}
整列性から一般的な数学的帰納法.
\pbenumex{
    以降の問題では, 正の整数全体の集合を $\mathbb{Z}^+$ で表す. このとき, 次の問に答えよ.
}{
\item 次の条件を満たす $\mathbb{Z}^+$ の部分集合 $S$ を考える.
\begin{align}
    &1 \in S\\
    &\forall n \in \mathbb{Z}^+ [n \in S \implies n + 1 \in s]
\end{align}
整列性(任意の空でない自然数の集合は最小限を持つ)を認めた上で, $S = \mathbb{Z}^+$ を満たすことを示せ.
\item $\mathbb{Z}^+$ の元の各々に対し, 命題 $P(n)$ が与えられたとし, それについて次の二つのことが示されたとする.
\setcounter{equation}{0}
\begin{align}
    &P(1) \text{は真}\\
    &\forall n \in \mathbb{Z}^+ [P(n) \text{が真} \implies P(n+1) \text{も真}]
\end{align}
\setcounter{equation}{0}
このとき, 全ての $n \in \mathbb{Z}^+$ に対して, $P(n)$ が真となることを示せ.
}{
\item $\mathbb{Z}^+ - S = S^{'}$ とし, $S^{'} = \emptyset$ となることを背理法により示す.\\
$S^{'} \neq \emptyset$ と仮定すると, 整列性より $n_0 = \min S^{'}$ となる $n_0 \in \mathbb{Z}^+$ が存在する.
今, $n_0 > 1$ より, $n_0 - 1 \geq 1$ となり $n_0 - 1 \in \mathbb{Z}^+$ である.
ここで $n_0 - 1 < n_0 = \min \mathbb{Z}^+$ より $n_0 - 1 \notin S^{'}$ , すなわち $n_0 - 1 \in S$ となる.\\
これより, $S$ が満たす条件(2)より, $n_0 \in S$ となり, $n_0 \in S^{'}$ に矛盾.\\
以上より, $S^{'} = \emptyset$ であり, $S = \mathbb{Z}^+$ が成り立つ.
\item $S =  \{\ n \in \mathbb{Z}^+\ |\ P(n) \text{が真} \}$ と集合 $S$ を定義すると, (1)より $S = \mathbb{Z}^+$ が成り立つ.
よって, 全ての $n \in \mathbb{Z}^+$ に対して, $P(n)$ が真となる
}
\end{nmprob}

