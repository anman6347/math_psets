\pgsc{代数学}{chocolate}{algebra}

\begin{nmprob}
ペアノシステムの数学的帰納法.
\pbenumex[peano]{
次の条件 (N1) から (N3) を満たす, 集合 $\bm{N}$, その一つの元 $0$, および写像 $\sigma :\bm{N} \to \bm{N}$ の対 $\langle \bm{N}, 0, \sigma \rangle$
をペアノシステムという.
\begin{description}\setlength{\leftskip}{16pt}
    \item[\rm (N1)] $\sigma :\bm{N} \to \bm{N}$ は単射である.
    \item[\rm (N2)] $0 \notin \sigma(\bm{N})$
    \item[\rm (N3)] $S \subset \bm{N}$ のとき, $S$ が次の二条件を満たせば $S = \bm{N}$ となる.
    \begin{align}
        0 \in S\\
        \sigma(S) \subset S
    \end{align}
    \setcounter{equation}{0}
\end{description}
特に, (N3) を数学的帰納法の公理とよぶ. また, $\sigma$ は後継者写像といい, $n$ に対して $\sigma(n)$ はその後継者とよばれる.
}{
\item 任意の $n \in \bm{N}$ に対して, $n \in S \implies \sigma(n) \in S$ が成り立つとすると, $\sigma(S) \subset S$ が成り立つことを示せ. ただし, $S \subset \bm{N}$ とする.
\item 任意の $n \in \bm{N}$ に対して, $\sigma(n) \neq n$ を示せ.
\item 任意の $n \in \bm{N}\backslash \{0\}$ に対して, $\exists m\in \bm{N},\ \sigma (m) = n$ を示せ. 
}{
\item $y \in \sigma (S)$ とすると, $\sigma (S)$ の定義から, $\langle x, y\rangle \in \sigma$ となる $x \in S$ が存在する.
今, $\forall n \in \bm{N},\ n \in S \implies \sigma (n) \in S$ が成り立つので, $\sigma (x) \in S$ となる. また, $\sigma$ は写像より $y = \sigma (x)$ となるから, $y \in S$ となる.
よって, $\sigma (S) \subset S$ となる.
\item $S = \{ n \in \bm{N}\ |\  \sigma (n) \neq n \}$ とする. まず, (N2) より $0 \in S$ となる.
次に, 任意の $n \in \bm{N}$ を一つとると, $n \in S$ ならば $\sigma (n) \neq n$ より, (N1) から $\sigma (\sigma (n)) \neq \sigma(n)$ となり, $\sigma (n) \in S$ となる.
よって, (1) より, $\sigma (S) \subset S$ となるから, (N3) より $S = \bm{N}$ となる. これは, $\forall n \in \bm{N},\ \sigma (n) \neq n$ を意味する.
\footnote[1]{$P(0)$が真, かつ$P(n)$が真ならば$P(\sigma (n))$ が真であれば $S = \{ n\in \bm{N}\ |\ P(n)\}$ とすることで N3 から $S = \bm{N}$ を示す方法はよく使われる.\vspace{30pt}}
\item $S = \{0\} \cup \sigma (\bm{N})$ とすると, 明らかに $0 \in S$ となる. また, 任意の $n \in \bm{N}$ に対して, $n \in S$ とすると, $\sigma (n) \in \sigma (\bm{N}) \subset S$ となる.
よって, (N3) より, $S = \bm{N}$ となる. ゆえに $\bm{N}$ の $0$ 以外の元は $\sigma (\bm{N})$ に含まれる.
以上より, 任意の $n \in \bm{N}\backslash \{0\}$ に対して, $\exists m\in \bm{N},\ \sigma (m) = n$ となる. ちなみに, (N2) とから, $\bm{N}\backslash \{0\} = \sigma (\bm{N})$ が成り立つ(外延性から示せる)
} 
\end{nmprob}



\begin{nmprob}
ペアノシステムの一意性.
\renewcommand{\labelenumii}{(\Roman{enumii})}
\pbenum{
\item $X$ を一つの集合とし, $X$ の一つの元 $x_0$ と写像 $\phi :X \to X$ とが与えられたとする.
このとき, 次の二条件を満たすような写像 $f :\bm{N} \to X$ がただ一つ存在することを示せ.

\renewcommand{\theequation}{\roman{equation}}
\begin{align}
& f(0) = x_0\\
& \forall n \in \bm{N},\ f(\sigma (n)) = \phi (f(n))
\end{align}
\setcounter{equation}{0}
\item $\langle \bm{N}, 0, \sigma \rangle$ と $\langle \bm{N^{'}}, 0^{'}, \sigma^{'} \rangle$ が共にペアノシステムであるとき,
$\bm{N}$ から $\bm{N^{'}}$ への全単射 $f$ で, $f(0) = 0^{'}$ かつ, 任意の $n \in \bm{N}$ に対して $f(\sigma (n)) = \sigma^{'}(f(n))$ となるもの
が一意的に存在することを示せ.
}{
\item
\begin{enumerate}
\item
まず, $f$ の一意性から示す.\\
相違な写像 $f: \bm{N} \to X,\ f^{'}: \bm{N} \to X$ が共に条件 (\rnum{1}), (\rnum{2}) を満たすとする.
ここで, $S = \{ n \in \bm{N}\ |\ f(n) = f^{'}(n)\}$ とすると, (\rnum{1}) より $0 \in S$ である.
また, 任意の $n \in \bm{N}$ に対して, $n \in S$ とすると, (\rnum{2}) より $f(\sigma (n)) = \phi (f(n)) = \phi (f^{'}(n)) = f^{'}(\sigma (n))$ となるから, $\sigma (n) \in S$ となるから
問\ref{sec:algebra}-\ref{pname:peano}\ (1) より, $S \subset \sigma (S)$ となる. よって, 数学的帰納法の公理より, $S = \bm{N}$ となる. よって, $f = f^{'}$ となり, 矛盾. よって, $f$ が存在するとすれば一意である.
\item
次に, $f$ の存在を示す.\\
まず, 次の二条件を満たす $\bm{N} \times X$ の部分集合 $R$ を考える.
\begin{align}
    \langle & 0, x_0\rangle \in R\\
    \forall & (n\in \bm{N})[\langle n, x\rangle \in R \implies \langle \sigma (n), \phi (x) \rangle \in R]
\end{align}
\setcounter{equation}{0}
条件(1), (2) を満たす集合全体を $\mathcal{F}$ とすると, $\bm{N} \times X \in \mathcal{F}$ より $\mathcal{F} \neq \emptyset$ である(条件 (2) は $\sigma (n) \in \bm{N},\ \phi (x) \in X$ より満たされる)
よって, $\bigcap_{R \in \mathcal{F}} R$ を考えることができる.\\
ここで, 
\begin{align*}
f = \bigcap_{R \in \mathcal{F}} R    
\end{align*}
としたとき, この $f$ が $\bm{N}$ から $X$ への写像であり, 二条件 (\rnum{1}), (\rnum{2}) を満たすことを示す.\\
まず, $f$ が $\bm{N}$ から $X$ への写像であることを示すために, $T = \{ n \in \bm{N}\ |\ \langle n, x\rangle \in f \text{なる} x \text{が唯一つ存在する}\}$ を考える.
\begin{enumerate}
\item $0 \in T$ を示す.\\
$f$ についてすぐにわかる通り, $f \in \mathcal{F}$ であるから, $0 \in T$ となる. 
\item $\forall n \in \bm{N},\ n \in T \implies \sigma (n) \in T$ を示す.\\
$n \in T$ とすると, $\langle \sigma (n), x_n \rangle \in f$ となる $x_n \in X$ がただ一つ存在する. まず, $f \in \mathcal{F}$ より, 条件 (2) から $\langle \sigma (n), \phi (x_n) \rangle \in f$ より,
$\exists x \in X,\ \langle n, x\rangle \in f$ となる. よって, この $x$ の一意性(つまり $x = \phi (x_n)$ のみである)を示せば, $\sigma (n) \in T$ となる.
一意性を得るために $\langle \sigma (n), y \rangle \in f$ となる $y \in X\backslash \{\phi (x_n) \}$ が存在すると仮定する.
ここで, $g = f\backslash \{ \langle \sigma (n), y \rangle\}$ として, $g \in \mathcal{F}$ を示す.
\begin{itemize}
\item 条件 (1)\\
ペアノシステムの公理 N2 より, $0 \notin \bm{N}$ であるから, $\sigma (n) \neq 0$ より, $\langle 0, x_0 \rangle \in g$ である.
\item 条件 (2)\\
$\langle n_g, x_g\rangle \in g$ を任意にとったとき, $\langle n_g, x_g\rangle \in f$ であり, $f \in \mathcal{F}$ とから, $f$ が条件 (2) を満たすことから, $\langle \sigma (n_g), \phi (x_g)\rangle \in f$ となる.
$n_g \neq n$ の場合はペアノシステムの公理 N1 より, $\langle \sigma (n_g), x_g\rangle \neq \langle \sigma (n), y\rangle$ である(第一成分が異なることより)から, $\langle \sigma (n_g), x_g\rangle \in g$ となる.
また, $n_g = n$ の場合は, 今 $n \in T$ で $\langle n, x_n \rangle \in f$ となる $x_n \in X$ はただ一つ存在することから, $x_g = x_n$ でなくてはならない. よって, $\langle \sigma (n_g), \phi (x_g) = \phi (x_n) \rangle \in f$
であり, $y \neq \phi (x_n)$ より, $\langle \sigma (n_g), x_g\rangle \neq \langle \sigma (n), y\rangle$ となるから $\langle \sigma (n_g), x_g\rangle \in g$ となる.
以上より, $\langle n_g, x_g\rangle \in g \implies \langle \sigma (n_g), x_g\rangle \in g$ となる.
\end{itemize}
以上より, $g \in \mathcal{F}$ である. さて, 今 $g \subsetneq f$ である. ここで, $f = \bigcap_{R \in \mathcal{F}} R$ より, $\forall R \in \mathcal{F},\ f \subset R$ より,
$g \in \mathcal{F}$ とから $g \subset f$ となるが, これは $g \subsetneq f$ に矛盾する. よって, $\langle \sigma (n), y \rangle \in f$ となる $y \in X\backslash \{\phi (x_n) \}$ は存在しない,
すなわち $\langle \sigma (n), x_n \rangle \in f$ となる $x_n \in X$ がただ一つ存在する. 以上より, $\sigma (n) \in T$ となり, $\forall n \in \bm{N},\ n \in T \implies \sigma (n) \in T$ が成り立つ.
\end{enumerate}
上の (a), (b) より, 問\ref{sec:algebra}-\ref{pname:peano}\ (1)より $T = \bm{N}$ となる. よって, $f$ は $\bm{N}$ から $X$ への写像となる. また, $0 \in T$ を示す際に述べた通り, $f \in \mathcal{F}$ より, $f$ は二条件 (1), (2) を満たす.
これより, すぐにわかる通り $f$ が満たすべき二条件 (\rnum{1}), (\rnum{2}) も満たされる. 以上より, $f$ の存在性が示された.
\end{enumerate}
\item
(1) において, $X = \bm{N}^{'},\ \phi = \sigma,\ x_0 = 0^{'}$ とすればよい.
}

\hspace{-3zw}{\color{orangered}●●\ 前問題の補足\ ●●}

前問題における各写像の様子を次図に示す.
\begin{figure}[htbp]
    \centering
    \begin{tikzpicture}
        \coordinate (O);       
        \node (A0) at ($(O) + (0, 1)$) {$0$};
        \node (B0) at ($(O) + (0, -1)$) {$x_0$};
        \node (A1) at ($(A0) + (2, 0)$) {$\sigma (0)$};
        \node (B1) at ($(B0) + (2, 0)$) {$\phi (x_0)$};
        \node (A2) at ($(A1) + (2, 0)$) {$\ldots$};
        \node (B2) at ($(B1) + (2, 0)$) {$\ldots$};
        \node (A3) at ($(A2) + (2, 0)$) {$n$};
        \node (B3) at ($(B2) + (2, 0)$) {$f(n)$};
        \node (A4) at ($(A3) + (2, 0)$) {$\sigma (n)$};
        \node (B4) at ($(B3) + (2, 0)$) {$\phi (f(n))$};

        \draw [|->] (A0) to node [above] {$\sigma$} (A1);
        \draw [|->] (A1) to node [above] {$\sigma$} (A2);
        \draw [|->] (A2) to node [above] {$\sigma$} (A3);
        \draw [|->] (A3) to node [above] {$\sigma$} (A4);
        \draw [|->] (B0) to node [below] {$\phi$} (B1);
        \draw [|->] (B1) to node [below] {$\phi$} (B2);
        \draw [|->] (B2) to node [below] {$\phi$} (B3);
        \draw [|->] (B3) to node [below] {$\phi$} (B4);
        \draw [bend right,distance=10,->] (A0) to node [left] {$f$} (B0);
        \draw [bend right,distance=10,->] (A1) to node [left] {$f$} (B1);
        \draw [bend right,distance=10,->] (A3) to node [left] {$f$} (B3);
        \draw [bend right,distance=10,->] (A4) to node [left] {$f$} (B4);
    \end{tikzpicture}
    \caption{前問題のイメージ}
\end{figure}

また, 写像の存在だけでなく一意性も示したのは以下のような要素の順番の入れ替えを防ぐためである. 以下ではわかりやすさを優先し, 全単射の例としている. 
\begin{figure}[htbp]
    \begin{minipage}[b]{0.45\linewidth}
    \centering
    \begin{tikzpicture}
        \coordinate (O);
        \coordinate (A1) at ($(O) + (-1, 1)$);
        \coordinate (A2) at ($(O) + (-1, 0)$);
        \coordinate (A3) at ($(O) + (-1, -1)$);
        \coordinate (B1) at ($(O) + (1, 1)$);
        \coordinate (B2) at ($(O) + (1, 0)$);
        \coordinate (B3) at ($(O) + (1, -1)$);
        \fill (A1) circle [radius=0.06];
        \fill (A2) circle [radius=0.06];
        \fill (A3) circle [radius=0.06];
        \fill (B1) circle [radius=0.06];
        \fill (B2) circle [radius=0.06];
        \fill (B3) circle [radius=0.06];
        \draw (A1) node [above] {$0$};
        \draw (A2) node [above] {$1$};
        \draw (A3) node [above] {$2$};
        \draw (B1) node [above] {$0^{'}$};
        \draw (B2) node [above] {$1^{'}$};
        \draw (B3) node [above] {$2^{'}$};
        \draw (A2) circle [x radius=0.8, y radius=1.6];
        \draw (B2) circle [x radius=0.8, y radius=1.6];
        \draw [->] (A1) to (B1);
        \draw [->] (A2) to (B2);
        \draw [->] (A3) to (B3);
    \end{tikzpicture}
    \caption{全単射の例 1}
    \end{minipage}
    \begin{minipage}[b]{0.45\linewidth}
    \centering
    \begin{tikzpicture}
        \coordinate (O);
        \coordinate (A1) at ($(O) + (-1, 1)$);
        \coordinate (A2) at ($(O) + (-1, 0)$);
        \coordinate (A3) at ($(O) + (-1, -1)$);
        \coordinate (B1) at ($(O) + (1, 1)$);
        \coordinate (B2) at ($(O) + (1, 0)$);
        \coordinate (B3) at ($(O) + (1, -1)$);
        \fill (A1) circle [radius=0.06];
        \fill (A2) circle [radius=0.06];
        \fill (A3) circle [radius=0.06];
        \fill (B1) circle [radius=0.06];
        \fill (B2) circle [radius=0.06];
        \fill (B3) circle [radius=0.06];
        \draw (A1) node [above] {$0$};
        \draw (A2) node [above] {$1$};
        \draw (A3) node [above] {$2$};
        \draw (B1) node [above] {$0^{'}$};
        \draw (B2) node [above] {$1^{'}$};
        \draw (B3) node [above] {$2^{'}$};
        \draw (A2) circle [x radius=0.8, y radius=1.6];
        \draw (B2) circle [x radius=0.8, y radius=1.6];
        \draw [->] (A1) to (B2);
        \draw [->] (A2) to (B3);
        \draw [->] (A3) to (B1);
    \end{tikzpicture}
    \caption{全単射の例 2}
    \end{minipage}
\end{figure}
\setcounter{figure}{0}
\end{nmprob}



\begin{nmprob}
整列性から一般的な数学的帰納法.
\pbenumex{
    以降の問題では, 正の整数全体の集合を $\mathbb{Z}^+$ で表す. このとき, 次の問に答えよ.
}{
\item 次の条件を満たす $\mathbb{Z}^+$ の部分集合 $S$ を考える.
\begin{align}
    &1 \in S\\
    &\forall n \in \mathbb{Z}^+ [n \in S \implies n + 1 \in s]
\end{align}
整列性(任意の空でない自然数の集合は最小限を持つ)を認めた上で, $S = \mathbb{Z}^+$ を満たすことを示せ.
\item $\mathbb{Z}^+$ の元の各々に対し, 命題 $P(n)$ が与えられたとし, それについて次の二つのことが示されたとする.
\setcounter{equation}{0}
\begin{align}
    &P(1) \text{は真}\\
    &\forall n \in \mathbb{Z}^+ [P(n) \text{が真} \implies P(n+1) \text{も真}]
\end{align}
\setcounter{equation}{0}
このとき, 全ての $n \in \mathbb{Z}^+$ に対して, $P(n)$ が真となることを示せ.
}{
\item $\mathbb{Z}^+ - S = S^{'}$ とし, $S^{'} = \emptyset$ となることを背理法により示す.\\
$S^{'} \neq \emptyset$ と仮定すると, 整列性より $n_0 = \min S^{'}$ となる $n_0 \in \mathbb{Z}^+$ が存在する.
今, $n_0 > 1$ より, $n_0 - 1 \geq 1$ となり $n_0 - 1 \in \mathbb{Z}^+$ である.
ここで $n_0 - 1 < n_0 = \min \mathbb{Z}^+$ より $n_0 - 1 \notin S^{'}$ , すなわち $n_0 - 1 \in S$ となる.\\
これより, $S$ が満たす条件(2)より, $n_0 \in S$ となり, $n_0 \in S^{'}$ に矛盾.\\
以上より, $S^{'} = \emptyset$ であり, $S = \mathbb{Z}^+$ が成り立つ.
\item $S =  \{\ n \in \mathbb{Z}^+\ |\ P(n) \text{が真} \}$ と集合 $S$ を定義すると, (1)より $S = \mathbb{Z}^+$ が成り立つ.
よって, 全ての $n \in \mathbb{Z}^+$ に対して, $P(n)$ が真となる
}
\end{nmprob}

