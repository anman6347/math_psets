\pgsc{代数学}{chocolate}{algebra}

\begin{nmprob}
ペアノシステムの数学的帰納法.
\pbenumex[peano]{
次の条件 (N1) から (N3) を満たす, 集合 $\bm{N}$, その一つの元 $0$, および写像 $\sigma :\bm{N} \to \bm{N}$ の対 $\langle \bm{N}, 0, \sigma \rangle$
をペアノシステムという.
\begin{description}\setlength{\leftskip}{16pt}
    \item[\rm (N1)] $\sigma :\bm{N} \to \bm{N}$ は単射である.
    \item[\rm (N2)] $0 \notin \sigma(\bm{N})$
    \item[\rm (N3)] $S \subset \bm{N}$ のとき, $S$ が次の二条件を満たせば $S = \bm{N}$ となる.
    \begin{align}
        0 \in S\\
        \sigma(S) \subset S
    \end{align}
    \setcounter{equation}{0}
\end{description}
特に, (N3) を数学的帰納法の公理とよぶ. また, $\sigma$ は後継者写像といい, $n$ に対して $\sigma(n)$ はその後継者とよばれる.
}{
\item 任意の $n \in \bm{N}$ に対して, $n \in S \implies \sigma(n) \in S$ が成り立つとすると, $\sigma(S) \subset S$ が成り立つことを示せ. ただし, $S \subset \bm{N}$ とする.
\item 任意の $n \in \bm{N}$ に対して, $\sigma(n) \neq n$ を示せ.
\item 任意の $n \in \bm{N}\backslash \{0\}$ に対して, $\exists m\in \bm{N},\ \sigma (m) = n$ を示せ. 
}{
\item $y \in \sigma (S)$ とすると, $\sigma (S)$ の定義から, $\langle x, y\rangle \in \sigma$ となる $x \in S$ が存在する.
今, $\forall n \in \bm{N},\ n \in S \implies \sigma (n) \in S$ が成り立つので, $\sigma (x) \in S$ となる. また, $\sigma$ は写像より $y = \sigma (x)$ となるから, $y \in S$ となる.
よって, $\sigma (S) \subset S$ となる.
\item $S = \{ n \in \bm{N}\ |\  \sigma (n) \neq n \}$ とする. まず, (N2) より $0 \in S$ となる.
次に, 任意の $n \in \bm{N}$ を一つとると, $n \in S$ ならば $\sigma (n) \neq n$ より, (N1) から $\sigma (\sigma (n)) \neq \sigma(n)$ となり, $\sigma (n) \in S$ となる.
よって, (1) より, $\sigma (S) \subset S$ となるから, (N3) より $S = \bm{N}$ となる. これは, $\forall n \in \bm{N},\ \sigma (n) \neq n$ を意味する.
\footnote[1]{$P(0)$が真, かつ$P(n)$が真ならば$P(\sigma (n))$ が真であれば $S = \{ n\in \bm{N}\ |\ P(n)\}$ とすることで N3 から $S = \bm{N}$ を示す方法はよく使われる.\vspace{30pt}}
\item $S = \{0\} \cup \sigma (\bm{N})$ とすると, 明らかに $0 \in S$ となる. また, 任意の $n \in \bm{N}$ に対して, $n \in S$ とすると, $\sigma (n) \in \sigma (\bm{N}) \subset S$ となる.
よって, (N3) より, $S = \bm{N}$ となる. ゆえに $\bm{N}$ の $0$ 以外の元は $\sigma (\bm{N})$ に含まれる.
以上より, 任意の $n \in \bm{N}\backslash \{0\}$ に対して, $\exists m\in \bm{N},\ \sigma (m) = n$ となる. ちなみに, (N2) とから, $\bm{N}\backslash \{0\} = \sigma (\bm{N})$ が成り立つ(外延性から示せる)
} 
\end{nmprob}



\begin{nmprob}
ペアノシステムの一意性.
\renewcommand{\labelenumii}{(\Roman{enumii})}
\pbenum[peanouq]{
\item $X$ を一つの集合とし, $X$ の一つの元 $x_0$ と写像 $\phi :X \to X$ とが与えられたとする.
このとき, 次の二条件を満たすような写像 $f :\bm{N} \to X$ がただ一つ存在することを示せ.

\renewcommand{\theequation}{\roman{equation}}
\begin{align}
& f(0) = x_0\\
& \forall n \in \bm{N},\ f(\sigma (n)) = \phi (f(n))
\end{align}
\setcounter{equation}{0}
\item $\langle \bm{N}, 0, \sigma \rangle$ と $\langle \bm{N^{'}}, 0^{'}, \sigma^{'} \rangle$ が共にペアノシステムであるとき,
$\bm{N}$ から $\bm{N^{'}}$ への全単射 $f$ で, $f(0) = 0^{'}$ かつ, 任意の $n \in \bm{N}$ に対して $f(\sigma (n)) = \sigma^{'}(f(n))$ となるもの
が一意的に存在することを示せ.
}{
\item
\begin{enumerate}
\item
まず, $f$ の一意性から示す.\\
相違な写像 $f: \bm{N} \to X,\ f^{'}: \bm{N} \to X$ が共に条件 (\rnum{1}), (\rnum{2}) を満たすとする.
ここで, $S = \{ n \in \bm{N}\ |\ f(n) = f^{'}(n)\}$ とすると, (\rnum{1}) より $0 \in S$ である.
また, 任意の $n \in \bm{N}$ に対して, $n \in S$ とすると, (\rnum{2}) より $f(\sigma (n)) = \phi (f(n)) = \phi (f^{'}(n)) = f^{'}(\sigma (n))$ となるから, $\sigma (n) \in S$ となるから
問\ref{sec:algebra}-\ref{pname:peano}\ (1) より, $S \subset \sigma (S)$ となる. よって, 数学的帰納法の公理より, $S = \bm{N}$ となる. よって, $f = f^{'}$ となり, 矛盾. よって, $f$ が存在するとすれば一意である.
\item
次に, $f$ の存在を示す.\\
まず, 次の二条件を満たす $\bm{N} \times X$ の部分集合 $R$ を考える.
\begin{align}
    \langle & 0, x_0\rangle \in R\\
    \forall & (n\in \bm{N})[\langle n, x\rangle \in R \implies \langle \sigma (n), \phi (x) \rangle \in R]
\end{align}
\setcounter{equation}{0}
条件(1), (2) を満たす集合全体を $\mathcal{F}$ とすると, $\bm{N} \times X \in \mathcal{F}$ より $\mathcal{F} \neq \emptyset$ である(条件 (2) は $\sigma (n) \in \bm{N},\ \phi (x) \in X$ より満たされる)
よって, $\bigcap_{R \in \mathcal{F}} R$ を考えることができる.\\
ここで, 
\begin{align*}
f = \bigcap_{R \in \mathcal{F}} R    
\end{align*}
としたとき, この $f$ が $\bm{N}$ から $X$ への写像であり, 二条件 (\rnum{1}), (\rnum{2}) を満たすことを示す.\\
まず, $f$ が $\bm{N}$ から $X$ への写像であることを示すために, $T = \{ n \in \bm{N}\ |\ \langle n, x\rangle \in f \text{なる} x \text{が唯一つ存在する}\}$ を考える.
\begin{enumerate}
\item $0 \in T$ を示す.\\
$f$ についてすぐにわかる通り, $f \in \mathcal{F}$ であるから, $0 \in T$ となる. 
\item $\forall n \in \bm{N},\ n \in T \implies \sigma (n) \in T$ を示す.\\
$n \in T$ とすると, $\langle \sigma (n), x_n \rangle \in f$ となる $x_n \in X$ がただ一つ存在する. まず, $f \in \mathcal{F}$ より, 条件 (2) から $\langle \sigma (n), \phi (x_n) \rangle \in f$ より,
$\exists x \in X,\ \langle n, x\rangle \in f$ となる. よって, この $x$ の一意性(つまり $x = \phi (x_n)$ のみである)を示せば, $\sigma (n) \in T$ となる.
一意性を得るために $\langle \sigma (n), y \rangle \in f$ となる $y \in X\backslash \{\phi (x_n) \}$ が存在すると仮定する.
ここで, $g = f\backslash \{ \langle \sigma (n), y \rangle\}$ として, $g \in \mathcal{F}$ を示す.
\begin{itemize}
\item 条件 (1)\\
ペアノシステムの公理 N2 より, $0 \notin \bm{N}$ であるから, $\sigma (n) \neq 0$ より, $\langle 0, x_0 \rangle \in g$ である.
\item 条件 (2)\\
$\langle n_g, x_g\rangle \in g$ を任意にとったとき, $\langle n_g, x_g\rangle \in f$ であり, $f \in \mathcal{F}$ とから, $f$ が条件 (2) を満たすことから, $\langle \sigma (n_g), \phi (x_g)\rangle \in f$ となる.
$n_g \neq n$ の場合はペアノシステムの公理 N1 より, $\langle \sigma (n_g), x_g\rangle \neq \langle \sigma (n), y\rangle$ である(第一成分が異なることより)から, $\langle \sigma (n_g), x_g\rangle \in g$ となる.
また, $n_g = n$ の場合は, 今 $n \in T$ で $\langle n, x_n \rangle \in f$ となる $x_n \in X$ はただ一つ存在することから, $x_g = x_n$ でなくてはならない. よって, $\langle \sigma (n_g), \phi (x_g) = \phi (x_n) \rangle \in f$
であり, $y \neq \phi (x_n)$ より, $\langle \sigma (n_g), x_g\rangle \neq \langle \sigma (n), y\rangle$ となるから $\langle \sigma (n_g), x_g\rangle \in g$ となる.
以上より, $\langle n_g, x_g\rangle \in g \implies \langle \sigma (n_g), x_g\rangle \in g$ となる.
\end{itemize}
以上より, $g \in \mathcal{F}$ である. さて, 今 $g \subsetneq f$ である. ここで, $f = \bigcap_{R \in \mathcal{F}} R$ より, $\forall R \in \mathcal{F},\ f \subset R$ より,
$g \in \mathcal{F}$ とから $g \subset f$ となるが, これは $g \subsetneq f$ に矛盾する. よって, $\langle \sigma (n), y \rangle \in f$ となる $y \in X\backslash \{\phi (x_n) \}$ は存在しない,
すなわち $\langle \sigma (n), x_n \rangle \in f$ となる $x_n \in X$ がただ一つ存在する. 以上より, $\sigma (n) \in T$ となり, $\forall n \in \bm{N},\ n \in T \implies \sigma (n) \in T$ が成り立つ.
\end{enumerate}
上の (a), (b) より, 問\ref{sec:algebra}-\ref{pname:peano}\ (1)より $T = \bm{N}$ となる. よって, $f$ は $\bm{N}$ から $X$ への写像となる. また, $0 \in T$ を示す際に述べた通り, $f \in \mathcal{F}$ より, $f$ は二条件 (1), (2) を満たす.
これより, すぐにわかる通り $f$ が満たすべき二条件 (\rnum{1}), (\rnum{2}) も満たされる. 以上より, $f$ の存在性が示された.
\end{enumerate}
\item
(1) において, $X = \bm{N}^{'},\ \phi = \sigma,\ x_0 = 0^{'}$ とすると, $f(0) = 0^{'}$ かつ $\forall n\in \bm{N},\ f(\sigma (n)) = \sigma^{'} (f(n))$ となる $f: \bm{N} \to \bm{N^{'}}$ が存在する.
この $f$ が全単射であることを示す. 先と逆に考えて $f^{'}(0^{'}) = 0$ かつ $\forall n^{'} \in \bm{N^{'}},\ f^{'}(\sigma^{'}(n^{'})) = \sigma (f^{'}(n^{'}))$ となる $f^{'}: \bm{N^{'}} \to \bm{N}$ も存在する. ここで, $g = f^{'} \circ f$ とすると, $g$ は $\bm{N}$ から $\bm{N}$ への写像となり,
$g(0) = 0$ かつ $\forall n \in \bm{N},\ g(\sigma (n)) = f^{'}(f(\sigma (n))) = f^{'}(\sigma^{'}(f(n))) = \sigma (f^{'}(f(n))) = \sigma (g(n))$ より, $g$ は条件 (\rnum{1}), (\rnum{2}) を満たす.
ここで, $\bm{N}$ から $\bm{N}$ への写像である $I_{\bm{N}}$ も, 条件 (\rnum{1}), (\rnum{2}) を満たす. よって, (1)より $g$ は一意であるから $g = f^{'} \circ f = I_{\bm{N}}$ となる. 同様に, $f \circ f^{'} = I_{\bm{N^{'}}}$ となるから,
問\ref{sec:set}-\ref{pname:comp_inv}\ (4) より, $f$ は全単射である.
} 

\hspace{-3zw}{\color{orangered}●●\ 前問題の補足\ ●●}

前問題における各写像の様子を次図に示す.
\begin{figure}[htbp]
    \centering
    \begin{tikzpicture}
        \coordinate (O);       
        \node (A0) at ($(O) + (0, 1)$) {$0$};
        \node (B0) at ($(O) + (0, -1)$) {$x_0$};
        \node (A1) at ($(A0) + (2, 0)$) {$\sigma (0)$};
        \node (B1) at ($(B0) + (2, 0)$) {$\phi (x_0)$};
        \node (A2) at ($(A1) + (2, 0)$) {$\ldots$};
        \node (B2) at ($(B1) + (2, 0)$) {$\ldots$};
        \node (A3) at ($(A2) + (2, 0)$) {$n$};
        \node (B3) at ($(B2) + (2, 0)$) {$f(n)$};
        \node (A4) at ($(A3) + (2, 0)$) {$\sigma (n)$};
        \node (B4) at ($(B3) + (2, 0)$) {$\phi (f(n))$};

        \draw [|->] (A0) to node [above] {$\sigma$} (A1);
        \draw [|->] (A1) to node [above] {$\sigma$} (A2);
        \draw [|->] (A2) to node [above] {$\sigma$} (A3);
        \draw [|->] (A3) to node [above] {$\sigma$} (A4);
        \draw [|->] (B0) to node [below] {$\phi$} (B1);
        \draw [|->] (B1) to node [below] {$\phi$} (B2);
        \draw [|->] (B2) to node [below] {$\phi$} (B3);
        \draw [|->] (B3) to node [below] {$\phi$} (B4);
        \draw [bend right,distance=10,->] (A0) to node [left] {$f$} (B0);
        \draw [bend right,distance=10,->] (A1) to node [left] {$f$} (B1);
        \draw [bend right,distance=10,->] (A3) to node [left] {$f$} (B3);
        \draw [bend right,distance=10,->] (A4) to node [left] {$f$} (B4);
    \end{tikzpicture}
    \caption{前問題のイメージ}
\end{figure}

また, 写像の存在だけでなく一意性も示したのは以下のような要素の順番の入れ替えを防ぐためである. 以下ではわかりやすさを優先し, 全単射の例としている. 
\begin{figure}[htbp]
    \begin{minipage}[b]{0.45\linewidth}
    \centering
    \begin{tikzpicture}
        \coordinate (O);
        \coordinate (A1) at ($(O) + (-1, 1)$);
        \coordinate (A2) at ($(O) + (-1, 0)$);
        \coordinate (A3) at ($(O) + (-1, -1)$);
        \coordinate (B1) at ($(O) + (1, 1)$);
        \coordinate (B2) at ($(O) + (1, 0)$);
        \coordinate (B3) at ($(O) + (1, -1)$);
        \fill (A1) circle [radius=0.06];
        \fill (A2) circle [radius=0.06];
        \fill (A3) circle [radius=0.06];
        \fill (B1) circle [radius=0.06];
        \fill (B2) circle [radius=0.06];
        \fill (B3) circle [radius=0.06];
        \draw (A1) node [above] {$0$};
        \draw (A2) node [above] {$1$};
        \draw (A3) node [above] {$2$};
        \draw (B1) node [above] {$0^{'}$};
        \draw (B2) node [above] {$1^{'}$};
        \draw (B3) node [above] {$2^{'}$};
        \draw (A2) circle [x radius=0.8, y radius=1.6];
        \draw (B2) circle [x radius=0.8, y radius=1.6];
        \draw [->] (A1) to (B1);
        \draw [->] (A2) to (B2);
        \draw [->] (A3) to (B3);
    \end{tikzpicture}
    \caption{全単射の例 1}
    \end{minipage}
    \begin{minipage}[b]{0.45\linewidth}
    \centering
    \begin{tikzpicture}
        \coordinate (O);
        \coordinate (A1) at ($(O) + (-1, 1)$);
        \coordinate (A2) at ($(O) + (-1, 0)$);
        \coordinate (A3) at ($(O) + (-1, -1)$);
        \coordinate (B1) at ($(O) + (1, 1)$);
        \coordinate (B2) at ($(O) + (1, 0)$);
        \coordinate (B3) at ($(O) + (1, -1)$);
        \fill (A1) circle [radius=0.06];
        \fill (A2) circle [radius=0.06];
        \fill (A3) circle [radius=0.06];
        \fill (B1) circle [radius=0.06];
        \fill (B2) circle [radius=0.06];
        \fill (B3) circle [radius=0.06];
        \draw (A1) node [above] {$0$};
        \draw (A2) node [above] {$1$};
        \draw (A3) node [above] {$2$};
        \draw (B1) node [above] {$0^{'}$};
        \draw (B2) node [above] {$1^{'}$};
        \draw (B3) node [above] {$2^{'}$};
        \draw (A2) circle [x radius=0.8, y radius=1.6];
        \draw (B2) circle [x radius=0.8, y radius=1.6];
        \draw [->] (A1) to (B2);
        \draw [->] (A2) to (B3);
        \draw [->] (A3) to (B1);
    \end{tikzpicture}
    \caption{全単射の例 2}
    \end{minipage}
\end{figure}
\setcounter{figure}{0}
\end{nmprob}



\begin{nmprob}
自然数の加法.
\makeatletter\tagsleft@true\makeatother
\pbenumex[Nplus]{
問\ref{sec:algebra}-\ref{pname:peanouq} より, $m \in \bm{N}$ のとき,
\begin{align}
    f_m(0) &= m \tag*{\hspace{3zw}(A1)$_m$} \\
    f_m\circ \sigma &= \sigma\circ f_m\hspace{2zw}\left(\text{すなわち, }\forall n \in \bm{N},\ f_m(\sigma (n))= \sigma(f_m (n))\right) \tag*{\hspace{3zw}(A2)$_m$}
\end{align}
を満たす$f_m:\bm{N} \to \bm{N}$ が一意的に存在する. ここで, $m, n \in \bm{N}$ に対して, $f_m(n)$ を $m, n$ の和とよび, $m + n$ と表す.
}{
\item $m \in \bm{N}$ のとき, $m + 0 = m$ となることを示せ.
\item $m, n \in \bm{N}$ のとき, $m + \sigma (n) = \sigma (m + n)$ となることを示せ.
\item $f_0 = I_{\bm{N}}$ となることを示せ.
\item $n \in \bm{N}$ のとき, $0 + n = n$ となることを示せ.
\item $m, n \in \bm{N}$ のとき, $\sigma (m) + n = \sigma (m + n)$ となることを示せ.
}{
\item (A1)$_m$ より, $m + 0 = m$ が成り立つ.
\item (A2)$_m$ より, $m + \sigma (n) = f_m(\sigma (n)) = \sigma (f_m(n)) = \sigma (m + n)$ となる.
\item 恒等写像の定義から, $I_{\bm{N}}(0) = 0$ である. また, $I_{\bm{N}}(\sigma (n)) = \sigma (n) = \sigma (I_{\bm{N}}(n))$ より, $I_{\bm{N}}$ は, 条件 (A1)$_0$, (A2)$_0$ を満たす.
よって, $f_0$ の一意性から $f_0 = I_{\bm{N}}$ となる.
\item (3) より, $0 + n = f_0(n) = I_{\bm{N}}(n) = n$ となる.
\item $h = \sigma \circ f_m$ とすると, $h(0) = \sigma (m)$ であり,
\begin{align*}
    h\circ \sigma &= (\sigma \circ f_m)\circ \sigma\\
    &= \sigma \circ (f_m \circ \sigma)&&\text{($\because$ 問題\ref{sec:set}-\ref{pname:comp_inv}\ (3)の写像の結合法則より)}\\
    &= \sigma \circ (\sigma \circ f_m)&&\text{($\because$ (A2)$_m$ より)}\\
    &= \sigma \circ h
\end{align*}
より, $h$ は条件 (A1)$_{\sigma (m)}$ (A2)$_{\sigma (m)}$ を満たす. よって, $f_{\sigma (m)}$ の一意性より $f_{\sigma (m)} = \sigma \circ f_m$ となるから
$\sigma (m) + n = f_{\sigma (m)}(n) = \sigma (f_m(n)) = \sigma (m + n)$ となる.
}
\hspace{-3zw}{\color{orangered}●●\ 加法のイメージ\ ●●}
\begin{figure}[htbp]
    \centering
    \begin{tikzpicture}
        \coordinate (O);       
        \node (A0) at ($(O) + (0, 1)$) {$0$};
        \node (B0) at ($(O) + (0, -1)$) {$m$};
        \node (A1) at ($(A0) + (2, 0)$) {$\sigma (0)$};
        \node (B1) at ($(B0) + (2, 0)$) {$\sigma (m)$};
        \node (A2) at ($(A1) + (2, 0)$) {$\ldots$};
        \node (B2) at ($(B1) + (2, 0)$) {$\ldots$};
        \node (A3) at ($(A2) + (2, 0)$) {$n$};
        \node (B3) at ($(B2) + (2, 0)$) {$m + n$};
        \node (A4) at ($(A3) + (2, 0)$) {$\sigma (n)$};
        \node (B4) at ($(B3) + (2, 0)$) {$\sigma (m + n)$};

        \draw [|->] (A0) to node [above] {$\sigma$} (A1);
        \draw [|->] (A1) to node [above] {$\sigma$} (A2);
        \draw [|->] (A2) to node [above] {$\sigma$} (A3);
        \draw [|->] (A3) to node [above] {$\sigma$} (A4);
        \draw [|->] (B0) to node [below] {$\sigma$} (B1);
        \draw [|->] (B1) to node [below] {$\sigma$} (B2);
        \draw [|->] (B2) to node [below] {$\sigma$} (B3);
        \draw [|->] (B3) to node [below] {$\sigma$} (B4);
        \draw [bend right,distance=10,->] (A0) to node [left] {$f_m$} (B0);
        \draw [bend right,distance=10,->] (A1) to node [left] {$f_m$} (B1);
        \draw [bend right,distance=10,->] (A3) to node [left] {$f_m$} (B3);
        \draw [bend right,distance=10,->] (A4) to node [left] {$f_m$} (B4);
    \end{tikzpicture}
    \caption{加法のイメージ}
\end{figure}
\setcounter{figure}{0}
\makeatletter\tagsleft@false\makeatother
\end{nmprob}



\begin{nmprob}
\makeatletter\tagsleft@true\makeatother
自然数の加法の交換律と結合律, 数学的帰納法. 半群, モノイド, 可換半群, 可換モノイドの定義.
\pbenumex[Nplus_monoid]{
空でない集合 $G$ に $G \times G$ から $G$ への 1 つの写像($G$ 上の(二項)演算という) $*$ \footnote[1]{記号 $*$ が二項演算を表すとき, $*$ はしばしば乗法(演算)とよばれる.}が与えられたとする.
$\langle a , b\rangle \in G \times G$ の $*$ による像を $a*b$ と表す\footnote[2]{厳密には, $(a*b)$ と表し, 一番外側の括弧を省略してもよいこととする. 後の問題でわかるように, $*$ が結合的であれば, 全ての括弧を省略することが可能である.}
ことにして, 次の条件 (G1) を満たすとき,
$*$ は結合律を満たす, あるいは結合的であるという.
\begin{align}
& \forall a, b, c \in G,\ (a * b) * c =  a * (b * c) & & (\text{結合律})\tag*{\hspace{3zw}(G1)}\\
\intertext{また, 空でない集合 $G$ 上のある二項演算 $*$ が結合律を満たすとき $\langle G, * \rangle$ は半群であるという.\newline
$\langle G, * \rangle$ が半群であり, さらに条件 (G2) を満たすとき, $\langle G, * \rangle$ はモノイドであるという.}
& \exists e \in G,\ \forall a \in G,\ e * a = a * e = a \tag*{\hspace{3zw}(G2)}
\intertext{条件 (G2) によって存在する $e \in G$ を $G$ の $*$ に関する単位元という.\newline
また, 二項演算 $*$ が次の条件を満たすとき, $*$ は交換律を満たす, あるいは可換であるといい, $*$ が可換かつ $\langle G, *\rangle$ が半群のとき, $\langle G, *\rangle$ を可換半群とよぶ.
同様にモノイドの場合は可換モノイドという.}
& \forall a, b \in G,\ a * b = b * a & &  (\text{交換律}) \tag*{}
\end{align}
}{
\item $\forall m, n\in \bm{N}$ に対して, 問\ref{sec:algebra}-\ref{pname:Nplus}\ における $f_m(n)$ を像とする写像は $\bm{N}$ 上の二項演算となり, これを $+$ で表すこととする.
\footnote[3]{記号 $+$ が二項演算とき, $+$ はしばしば加法(演算)とよばれる.}
$\langle \bm{N}, +\rangle$ が可換モノイドであることを示せ.
\item $1 \coloneqq \sigma(0)$ と定義する. $\bm{N}$ の部分集合 $S$ が $0 \in S$ かつ $\forall n \in \bm{N},\ n \in S \implies n + 1 \in S$ を満たすとき, $S = \bm{N}$ となることを示せ.
}{
\item 問\ref{sec:algebra}-\ref{pname:Nplus}\ (1), (4) より, $\bm{N}$ の加法に単位元 $0$ が存在する.
よって, 加法が結合律と交換律を満たすことを示せばよい.
\begin{enumerate}
    \item 交換律\\
    任意の $n \in \bm{N}$ を一つとり, $S = \{ m \in \bm{N}\ |\ m + n = n + m\}$ とする.\\
    $\bm{N}$ の加法に関して単位元が存在することから, $0 \in S$ となる.
    また $m \in S$ と仮定すると, $m + n = n + m$ となり, 問\ref{sec:algebra}-\ref{pname:Nplus}\ (2), (4) より,
    $\sigma (m) + n = \sigma (m + n) = \sigma (n + m) = n + \sigma (m)$ となるから, $\forall m \in \bm{N},\ m \in S \implies \sigma (m) \in S$ が成り立つ.
    よって, 問\ref{sec:algebra}-\ref{pname:peano}\ (1) とから $S = \bm{N}$ が成り立つ. 今, $n$ は任意にとっているので, $\forall m, n\in \bm{N},\ m + n = n + m$ が成り立つ.
    \item 結合律\\
    交換律の成立を示したものと同等な帰納法で示す.\\
    任意の $n, k \in \bm{N}$ をとる. まず, 問\ref{sec:algebra}-\ref{pname:Nplus}\ (3) より, $(0 + n) + k = n + k = 0 + (n + k)$ が成り立つ.\\
    次に, 任意の $m \in \bm{N}$ を一つとり, この $m$ に対して, $(m + n) + k = m + (n + k)$ が成り立つと仮定する.
    ここで, 問\ref{sec:algebra}-\ref{pname:Nplus}\ (5) より, $(\sigma (m) + n) + k = \sigma (m + n) + k = \sigma ((m + n) + k)$ であり, 先の仮定から, $\sigma ((m + n) + k) = \sigma (m + (n + k)) = \sigma (m) + (n + k)$
    となるから, $(\sigma (m) + n) + k = \sigma (m) + (n + k)$ が成り立つ.\footnote[4]{これにより, $P(0) \land \forall n \in \bm{N},\ P(n) \implies P(n + 1)$ から $\forall n \in \bm{N},\ P(n)$ が成り立つといえる.}
\end{enumerate}
\item 問\ref{sec:algebra}-\ref{pname:Nplus}\ より, $n + 1 = n + \sigma (0) = \sigma (n + 0) = \sigma (n)$ となる.
よって, 問\ref{sec:algebra}-\ref{pname:peano}\ (1) より, $\sigma (S) \subset S$ となり, 数学的帰納法の公理より, $S = \bm{N}$ となる.
}
\makeatletter\tagsleft@false\makeatother
\end{nmprob}



\begin{nmprob}
\makeatletter\tagsleft@true\makeatother
自然数の乗法.
\pbenumex[Nprod]{
問\ref{sec:algebra}-\ref{pname:peanouq} より, $m \in \bm{N}$ のとき, $f_m$ を問\ref{sec:algebra}-\ref{pname:Nplus}\ の $f_m$ として,
\begin{align}
    g_m(0) &= 0 \tag*{\hspace{3zw}(M1)$_m$} \\
    g_m\circ \sigma &= f_m \circ g_m\hspace{2zw}\left(\text{すなわち, }\forall n \in \bm{N},\ f_m(\sigma (n))= \sigma(f_m (n))\right) \tag*{\hspace{3zw}(M2)$_m$}
\end{align}
を満たす$g_m:\bm{N} \to \bm{N}$ が一意的に存在する. ここで, $m, n \in \bm{N}$ に対して, $g_m(n)$ を 像とする写像は $\bm{N}$ 上の二項演算となり, これを $*$ で表すこととする.
$m, n \in \bm{N}$ に対して, $m * n (= g_m(n))$ を $m , n$ の積とよぶ($m, n$ の積は $mn$ や $m \cdot n$ と表す場合もある\footnote[1]{厳密には, $(a*b)$, $(ab)$ または $(a \cdot b)$ と表し, 一番外側の括弧を省略してもよいこととする. 後の問題でわかるように, $*$ が結合的であれば, 全ての括弧を省略することが可能である.})
また, 一般に, 空でない集合 $G$ 上に二つの異なる二項演算が定義されているときに,
一方を加法, もう一方を乗法と見なした場合, 加法の引数となる $G$ の要素が, $(ab)$ のように $G$ の積で表されているとき, その要素の一番外側の括弧を省略し, $ab$ と表すことを許す.
}{
\item $m\in \bm{N}$ のとき, $m0 = 0$ となることを示せ.
\item $m, n \in \bm{N}$ のとき, $m \cdot \sigma (n) = m(n + 1) = mn + m$ となることを示せ.
\item $n \in \bm{N}$ のとき, $0n = 0$ となることを示せ.
\item $m, n \in \bm{N}$ のとき, $\sigma (m) \cdot n = (m + 1)n = mn + n$ となることを示せ.
}{
\item (M1)$_m$ より, $m0 = 0$ が成り立つ.
\item まず, 問\ref{sec:algebra}-\ref{pname:Nplus_monoid}\ (2) より, $\sigma (n) = n + 1$  であるから, $m \cdot \sigma (n) = m(n + 1)$ が成り立つ.
次に, (M2)$_m$ より, $m \cdot \sigma(n) = m + mn$ となるので, 問\ref{sec:algebra}-\ref{pname:Nplus_monoid}\ (1) より, 加法が交換律を満たすことから, $m\cdot \sigma(n) = mn + m$ となる.
よって, $m \cdot \sigma (n) = m(n + 1) = mn + m$ となる.
\item 任意の $n \in \bm{N}$ に対して, $\psi_0(n) = 0$ となるような定値写像 $\psi_0$ を考えると, $\psi_0(0) = 0$ より $\psi_0$ は (M1)$_0$ を満たす.
また, $\psi_0(\sigma(n)) = 0$ であり, 問\ref{sec:algebra}-\ref{pname:Nplus}\ (4) の $f_0(0) = 0$ とから, $\psi_0(\sigma(n)) = f_0(\psi_0(n))$ より, $\psi_0$ は (M2)$_0$ を満たす.
よって, 問\ref{sec:algebra}-\ref{pname:peanouq}\ (1) より, $g_0 = \psi_0$ となるから, $0n = 0$ となる.
\item まず, (2) と同様にして $\sigma (m) \cdot n = (m + 1)n$ が成り立つ.
次に, 任意の $n$ に対して, $f_{g_m(n)}(n)$ を像とする写像 $h$ を考える(すなわち $h(n) = mn + n$ となる)と, $h(0) = f_0(0) = 0$ となる.
また, (2) および加法の結合律と交換律より
\begin{align*}
    h(\sigma (n)) &= m \cdot \sigma (n) + \sigma (n)\\ 
    &= (mn + m) + (n + 1)\\
    &= (n + 1) + (m + mn)\\
    &= n + (1 + (m + mn))\\
    &= n + ((m + 1) + mn)\\
    &= n + (\sigma (m) + mn)\\
    &= \sigma(m) + (mn + n)\\
    &= \sigma(m) + h(n)\\
    &= f_{\sigma (m)}(h(n))
\end{align*}
となるから, $h$ は (M1)$_{\sigma (m)}$ と (M2)$_{\sigma (m)}$ を満たすので, 問\ref{sec:algebra}-\ref{pname:peanouq}\ (1) より, $g_{\sigma (m)} = h$ となる.
よって, $\sigma (m) \cdot n = mn + n$ となるから, $\sigma (m) \cdot n = (m + 1)n = mn + n$ となる.


}
\newpage
\hspace{-3zw}{\color{orangered}●●\ 乗法のイメージ\ ●●}
\begin{figure}[htbp]
    \centering
    \begin{tikzpicture}
        \coordinate (O);       
        \node (A0) at ($(O) + (0, 1)$) {$0$};
        \node (B0) at ($(O) + (0, -1)$) {$0$};
        \node (A1) at ($(A0) + (2, 0)$) {$\sigma (0)$};
        \node [minimum width = 26pt, minimum height = 10pt, label = {[align=center, label distance = -12pt]south:$f_m (0)$\\ \raise2pt\hbox{\rotatebox{90}{$=$}} \\ \raise8pt\hbox{$m$}}](B1) at ($(B0) + (2, 0)$) {};
        \node (A2) at ($(A1) + (2, 0)$) {$\ldots$};
        \node (B2) at ($(B1) + (2, 0)$) {$\ldots$};
        \node (A3) at ($(A2) + (2, 0)$) {$n$};
        \node (B3) at ($(B2) + (2, 0)$) {$mn$};
        \node (A4) at ($(A3) + (2, 0)$) {$\sigma (n)$};
        \node (B4) at ($(B3) + (2, 0)$) {$m + mn$};

        \draw [|->] (A0) to node [above] {$\sigma$} (A1);
        \draw [|->] (A1) to node [above] {$\sigma$} (A2);
        \draw [|->] (A2) to node [above] {$\sigma$} (A3);
        \draw [|->] (A3) to node [above] {$\sigma$} (A4);
        \draw [|->] (B0) to node [below] {$f_m$} (B1);
        \draw [|->] (B1) to node [below] {$f_m$} (B2);
        \draw [|->] (B2) to node [below] {$f_m$} (B3);
        \draw [|->] (B3) to node [below] {$f_m$} (B4);
        \draw [bend right,distance=10,->] (A0) to node [left] {$g_m$} (B0);
        \draw [bend right,distance=10,->] (A1) to node [left] {$g_m$} (B1);
        \draw [bend right,distance=10,->] (A3) to node [left] {$g_m$} (B3);
        \draw [bend right,distance=10,->] (A4) to node [left] {$g_m$} (B4);
    \end{tikzpicture}
    \caption{乗法のイメージ}
\end{figure}
\setcounter{figure}{0}
\makeatletter\tagsleft@false\makeatother
\end{nmprob}



\begin{nmprob}
乗法の交換律, 結合律. 加法と乗法の分配律.
\pbenum{
\item $\bm{N}$ における乗法が交換律を満たす, すなわち, 任意の $m, n \in \bm{N}$ に対して, $mn = nm$ となることを示せ.
\item 一般に, 空でない集合 $G$ に二つの二項演算が与えられ, 各々を加法と乗法としたとき, 任意の $a, b, c \in G$ に対して, 次の条件を満たすとき, 乗法は加法に対して左分配律を満たす, または左側から分配的であるという.
\begin{align*}
    a(b + c) = ab + ac & &  (\text{左分配律})
\end{align*}
また, 次の条件を満たす場合には, 乗法は加法に対して右分配律を満たす, または右側から分配的であるという.
\begin{align*}
    (b + c)a = ba + ca & &  (\text{右分配律})
\end{align*}
乗法が加法に対して左分配律も右分配律も満たすとき, 乗法は加法に対して(両側)分配律を満たす, または両側から分配的だという.\\
$\bm{N}$ において, 乗法は加法に対して分配律を満たすことを示せ.
\item $\bm{N}$ における乗法が結合律を満たす, すなわち, 任意の $m, n, k \in \bm{N}$ に対して, $(mn)k = m(nk)$ となることを示せ.
}{
\item 任意の $n \in \bm{N}$ を一つとり, $m$ に関する数学的帰納法で示す.
まず, 問\ref{sec:algebra}-\ref{pname:Nprod}\ (1), (3) より $0n = 0 = n0$ で $0n = n0$ が成り立つ.
次に, 任意の $m \in \bm{N}$ をとり, この $m$ に対して $mn = nm$ が成り立つと仮定すると, この仮定と問\ref{sec:algebra}-\ref{pname:Nprod}\ (2), (4) から,
$\sigma (m) \cdot n = mn + n = nm + n = n \cdot \sigma (m)$ となり, $\forall m \in \bm{N},\ (mn = nm \implies \sigma (m)\cdot n = n \cdot \sigma (m))$ が成り立つ.
以上より, 任意の $m, n \in \bm{N}$ に対して, $mn = nm$ となる.
\item 乗法が加法に対して左分配律を満たす, すなわち $m(n + k) = mn + mk$ であることを示せば, (1) より, 乗法の交換律から, 乗法は右分配律を満たすので, 左分配律のみ示せば十分である.\\ 
任意の $m, n \in \bm{N}$ をとり, $k$ に関する数学的帰納法で示す.\\
まず, 問\ref{sec:algebra}-\ref{pname:Nplus}\ (1) および 問\ref{sec:algebra}-\ref{pname:Nprod}\ (3) より, $m(n + 0) = mn = mn + m0$ が成り立つ.\\
次に, 任意の $k \in \bm{N}$ を一つとり, この $k$ に対して, $m(n + k) = mn + mk$ が成り立つと仮定すると,
\begin{align*}
m(n + \sigma (k)) &= m(n + (k + 1))\\
&= m((n + k) + 1)\\
&= m(n + k) + m\\
&= (mn + mk) + m\\
&= mn + (mk + m)\\
&= mn + m\cdot \sigma (k)
\end{align*}
となるから, $\forall k \in \bm{N},\ m(n + k) = mn + mk \implies m(n + \sigma (k)) = mn + m\sigma (k)$ が成り立つ.
よって, $\forall k \in \bm{N},\ m(n + k) = mn + mk$ となり, 乗法は加法に対して分配律を満たす.
\item 任意の $m, n\in \bm{N}$ をとり, $k$ に関する数学的帰納法で示す.\\
まず, $(mn)0 = 0 = m0 = m(n0)$ より, $(mn)0 = m(n0)$ が成り立つ.\\
次に, 任意の $k \in \bm{N}$ を一つとり, この $k$ に対して, $(mn)k = m(nk)$ が成り立つと仮定すると, 乗法の加法に対する分配律から
$(mn)\sigma (k) = (mn)k + mn = m(nk) + mn = m(nk + n) = m(n \cdot \sigma (k))$ より, $(mn) \sigma (k) = m(n\cdot \sigma (k))$ となる.
以上より, $\bm{N}$ における乗法は結合律を満たす.
}
\end{nmprob}



\begin{nmprob}
自然数の順序への準備
\pbenum[Nord_pre]{
\item $k \in \bm{N}\backslash \{0\}$ のとき, 任意の $n \in \bm{N}$ に対して, $n + k \neq n$ となることを示せ.
\item $k \in \bm{N}\backslash \{0\}$ のとき, 任意の $n \in \bm{N}$ に対して, $n + k \neq 0$ となることを示せ. 
\item 任意の $m, n \in \bm{N}$ に対して, 次の 3 つのいずれか一つだけ必ず成り立つことを示せ.
\begin{enumerate}
    \item ある $k \in \bm{N}\backslash\{0\}$ に対して $m = n + k$
    \item $m = n$
    \item ある $l \in \bm{N}\backslash\{0\}$ に対して $n = m + l$
\end{enumerate}
}{
\item $n$ に関する数学的帰納法で示す.\\
まず, $0 + k = k \neq 0$ より, $0 + k \neq 0$ より, $n = 0$ のとき, $n + k \neq n$ が成り立つ.\\
次に, 任意の $n \in \bm{N}$ を一つとり, この $n$ に対して, $n + k \neq n$ となるとを仮定すると,
$\sigma (n) + k = \sigma (n + k)$ で, $\sigma$ の単射性より, $n + k \neq n$ とから, $\sigma (n + k) \neq \sigma (n)$となり, $\sigma (n) + k \neq \sigma (n)$ となる.
以上より, $k \in \bm{N}\backslash \{0\}$ のとき, 任意の $n \in \bm{N}$ に対して, $n + k \neq n$ となる.
\item 背理法で示す. ある $n_0 \in \bm{N}$ に対して, $n_0 + k = 0$ となると仮定する.
まず, $k \neq 0$ より, 問\ref{sec:algebra}-\ref{pname:peano}\ (3) より, ある $m \in \bm{N}$ に対して, $\sigma (m) = k$ となる.
ここで, $n_0 + k = n_0 + \sigma (m) = \sigma (n_0 + m)$ となるが, ペアノシステムの条件 (N2) より, $\sigma (n_0 + m) \neq 0$ となる.
よって, 背理法の過程に矛盾することから, $k \in \bm{N}\backslash \{0\}$ のとき, 任意の $n \in \bm{N}$ に対して, $n + k \neq 0$ となる.
\item まず, (\rnum{1}) から (\rnum{3}) のどの 2 つも両立し得ないことを示す.\\
(\rnum{1}) かつ (\rnum{2}) が同時に成り立つと仮定すると, $n = n + k$ となり, これは (1) により矛盾する.
次に, (\rnum{2}) かつ (\rnum{3}) が同時に成り立つと仮定すると, 同様に (1) により矛盾する.
最後に, (\rnum{1}) かつ (\rnum{3}) が同時に成り立つと仮定すると, $m = m + (l + k)$ となる. ここで, (2) より $l + k \neq 0$ であるから, (1) により $m \neq m + (l + k)$ となり矛盾する.
以上より, (\rnum{1}) から (\rnum{3}) のどの 2 つも両立し得ない.\\
次に, (\rnum{1}) から (\rnum{3}) のいずれかが必ず成り立つことを示す.\\
任意の $n$ をとり, (\rnum{1}) から (\rnum{3}) のいずれかが成り立つような $m \in \bm{N}$ の集合を $S$ として $m$ に関する数学的帰納法で示す.
まず, $n = 0$ であれば, $m = 0$ のとき (\rnum{2}) が成り立ち, $n \neq 0$ であれば, $m = 0$ のとき, $l = n$ として (\rnum{3}) が成り立つので, $m \in S$ である.\\
次に, $m \in S$ とする. $m$ について, (\rnum{1}) が成り立つと仮定すると, $m = n + k$ より, $\sigma (m) = \sigma (n + k) = n + \sigma (k)$ で $\sigma (k) \neq 0$ より $\sigma (m) \in S$ となる.
また, $m$ について, (\rnum{2}) が成り立つと仮定すると, $m = n$ より, $\sigma (m) = \sigma (n) = n + \sigma (0)$ で $\sigma (0) \neq 0$ より, $\sigma (m) \in S$ となる.
最後に, $m$ について, (\rnum{3}) が成り立つと仮定すると, $n = m + l$ かつ $l \neq 0$ であり, $l = \sigma (l^{'})$ とすると $n = m + \sigma (l^{'}) = \sigma (m) + l^{'}$ であり,
$l^{'} = 0$ であれば $\sigma (m) = n$ が, $l^{'} \neq 0$ であれば $n = \sigma (m) + l^{'}$ かつ $l^{'} \neq 0$ が成り立つから, $\sigma (m) \in S$ となる.\\
よって, $\forall m \in \bm{N},\ m \in S \implies \sigma (m) \in S$ が成り立つ.\\
以上より, 任意の $m, n \in \bm{N}$ に対して, (\rnum{1}) から (\rnum{3}) のいずれか一つが必ず成り立つ.
} 
\end{nmprob}



\begin{nmprob}
自然数の順序
\pbenum[Nord]{
\item $\bm{N}$ 上の二項関係 $<$ を $< \coloneqq \{ \langle m, n\rangle \in \bm{N} \times \bm{N}\ |\ \exists l \in \bm{N}\backslash \{0\},\ n = m + l\}$ と定義する.
この二項関係 $<$ が問 \ref{sec:set}-\ref{pname:ordr}\ の (O4) と (O5) を満たすことを示せ.
\item (1) により, 問 \ref{sec:set}-\ref{pname:ordr}\ (2) により定義される $\bm{N}$ 上の二項関係 $\leq$ は $\bm{N}$ 上の順序関係となる.
ここで順序集合 $\langle \bm{N}, \leq \rangle$ は全順序集合であることを示せ.
\item $0$ は $\bm{N}$ の最小元であることを示せ.
\item $m < n$ ならば $m + 1 \leq n$ となることを示せ.
\item $\sigma : \bm{N} \to \bm{N}$ は順序単射であることを示せ.
\item $\bm{N}$ の中で $\sigma (m)$ は $m$ の直後の元であることを示せ.
}{
\item まず, (O4) が成り立つことを示す. $m < n$ が成り立つ下で, $n < m$ が成り立つと仮定すると,
ある $k, l \in \bm{N}\backslash \{0\}$ に対して, $n = m + k$ かつ $m = n + l$ となることから $n = k + (n + l)$ となるが, これは問\ref{sec:algebra}-\ref{pname:Nord_pre}\ (3) の証明と同様にして矛盾である.
よって, $m < n \implies n \nless m$ となる.\\
次に, (O5) が成り立つことをを示す. $m < n$ かつ $n < k$ が成り立つと仮定すると, ある $l, l^{'} \in \bm{N}\backslash \{0\}$ に対して,
$n = m + l$ かつ $k = n + l^{'}$ となる. よって, $k = m + (l + l^{'})$ となる. ここで, $l + l^{'} = 0$ と仮定すると, $k = m$ となり, $m < n$ かつ $n < m$ となり, 矛盾する.
よって, $l + l^{'} \neq 0$ となるから, $m < k$ が成り立つ.\\
以上より, (O4) と (O5) が成り立つ.
\item 問\ref{sec:algebra}-\ref{pname:Nord_pre}\ (3) より, 任意の自然数 $m, n$ に対して $m < n,\ m = n,\ n < m$ のいずれか一つのみが必ず成り立つ.
よって, $m \leq n = m < n \lor m = n$ より, $\langle \bm{n}, \leq \rangle$ は全順序集合となる.
\item $m, n \in \bm{N}$ に対して, $m \leq n \iff \exists l \in \bm{N},\ n = m + l$ であるから, 任意の $n \in \bm{N}$ に対して,
$n = 0 + n$ であることとから, $0 \leq n$ より, $0$ は $\bm{N}$ の最小元となる.
\item $m < n$ とすると, ある $l \in \bm{N}\backslash \{0\}$ に対して, $n = m + l$ となる. $l \neq 0$ より, ある $l^{'} \in \bm{N}$ に対して, $l = \sigma (l^{'})$ と
表せるから, $n = m + \sigma (l^{'}) = \sigma (m) + l^{'}$ より, $m + 1 \leq n$ となる. 
\item $m, n\in \bm{N}$ に対して, $m \leq n$ とすると, ある $l \in \bm{N}$ に対して, $n = m + l$ となる.
よって, $n + 1 = (m + l) + 1 = (m + 1) + l$ より, $\sigma (m) \leq \sigma (n)$ であるから, $\sigma$ は順序写像. また, $\sigma (m) \leq \sigma (n)$ とすると,
$\sigma (m) = \sigma (n)$ の場合は, $\sigma$ の単射性から $m = n$ であり, $m \leq n$ となる. $\sigma (m) < \sigma (n)$ の場合, $m \geq n$ と仮定すると, $\sigma$ は順序写像により $\sigma (m) \geq \sigma (n)$ となり矛盾するので,
$m \geq n$ とはならず, $m, n$ に対して $m < n,\ m = n,\ n < m$ のいずれか一つのみが必ず成り立つことから, $m < n$ となり, $m \leq n$ となる. よって, $\sigma (m) \leq \sigma (n)$ とすると $m \leq n$ となるので $\sigma$ は順序単射である.
\item (2) より, 問\ref{sec:set}-\ref{pname:w_ordset}\ (2) とから, $\sigma (m)$ が $m$ の直後の元であることを示すには, $m = \max \text{seg}_{\bm{N}}(\sigma (m))$ となることを示せば十分である.\\
まず, $\sigma (m) = m + 1$ で $1 = \sigma (0) \neq 0$ より, $m < \sigma (m)$ であるから, $m \in \text{seg}_{\bm{N}}(\sigma (m))$ である.
次に, 任意の $n \in \text{seg}_{\bm{N}}(\sigma (m))$ を一つとると, $n < \sigma (m)$ より, (4) から $n + 1 \leq \sigma (m)$ , すなわち $\sigma (n) \leq \sigma (m)$ となる.
ここで, (5) より $\sigma$ は順序単射であるから $n \leq m$ となる. よって, $m = \max \text{seg}_{\bm{N}}(\sigma (m))$ となるので, $\sigma (m)$ が $m$ の直後の元となる.
}
\end{nmprob}



\begin{nmprob}
自然数の整列性, 加法・乗法の単調性.
\pbenum{
\item $\bm{N}$ の任意の空でない部分集合 $S$ は最小元をもつことを示せ.
\item $m < n$ ならば, 任意の $k \in \bm{N}$ に対して, $m + k < n + k$ となること(加法の単調性)を示せ.
\item $m + k = n + k$ ならば $m = n$ となること(加法に関する簡約律)を示せ.
\item $m \neq 0 \land n \neq 0$ ならば $mn \neq 0$ であることを示せ.
\item $m < n \land k \neq 0$ ならば $mk < nk$ となること(乗法の単調性)を示せ.
\item $mk = nk \land k \neq 0$ ならば $m = n$ となること(乗法の簡約律)を示せ.
}{
\item $T = \{ n \in \bm{N}\ |\ \forall x \in S,\ n \leq x\}$ とすると, 問\ref{sec:algebra}-\ref{pname:Nord}\ (3) より, $0 \in T$ であり,
また, $S$ の元の一つを $c$ とすると, $\sigma (c) \notin T$ となるから, $T \neq \bm{N}$ となる.
よって, 数学的帰納法の公理より, $m \in T \land \sigma (m) \notin T$ となる $m \in \bm{N}$ が存在する(仮に存在しなければ, 任意の $m \in \bm{N}$ に対して, $m \in T \implies \sigma (m) \in T$ で $T = \bm{N}$ となってしまう)
この $m$ に対して, $m \in T$ より, $\forall x \in S,\ x \leq m$ であるが, もし $m \notin S$ ならば, $m < x$ で, 問\ref{sec:algebra}-\ref{pname:Nord}\ (4) より, $\sigma (m) \leq x$ となり, $\sigma (m) \notin T$ に矛盾する.
よって, $m \in S$ であり, $m \in T$ とから, $m$ は $S$ の最小元である.
\item $m < n$ より, ある $l \neq \bm{N}\backslash\{0\}$ に対して, $n = m + l$ となり, $n + k = (m + l) + k = (m + k) + l$ より, $m + k < n + k$ となる.
\item $m + k = n + k$ の下で $m < n$ と仮定すると, (2) より, $m + k < n + k$ より, $m + k \neq n + k$ で矛盾.
また, $n < m$ と仮定しても同様に矛盾する. よって, $m = n$ となる.
\item $m \neq 0$ と $n \neq 0$ より, ある$m^{'}, n^{'} \in \bm{N}$ に対して $m = \sigma (m^{'}) = m^{'} + 1, n = \sigma (n^{'}) = n^{'} + 1$ となる.
よって, $mn = m^{'}n^{'} + m^{'} + n^{'} + 1$ となり, $1 \leq mn$ であり, $mn = 0$ と仮定すると矛盾するから $mn \neq 0$ となる.
\item $n = m + l,\ l \neq 0$ とすると, $nk = mk + lk$ で $l \neq 0$ かつ $k \neq 0$ より, (4) から $lk \neq 0$ で $mk < nk$ となる.
\item (3) と同様にして示せる. 
}
\end{nmprob}