\documentclass[dvipdfmx,a4j]{jarticle}
\usepackage{psets}
% ページスタイル : pg, pgs, pgsc

% 通常ページのデザインをデフォルトにする
\pagestyle{pg}

% バージョン
\newcommand{\version}{0.3}

% 問題番号
\newcounter{numpb}
\setcounter{numpb}{0}


% 問題テンプレ箇条書きver
\renewcommand{\labelenumi}{(\arabic{enumi})}
\renewcommand{\labelenumii}{(\roman{enumii})}
\renewcommand{\labelenumiii}{(\alph{enumiii})}
\newcommand{\pbenum}[3][undefined]{
    \vspace{2mm}
    \hrule
    \vspace{2mm}
    \refstepcounter{numpb}
    \label{pname:#1}
    \noindent
    \textbf{問\arabic{secn}--\arabic{numpb}:} 次の問に答えよ.
        \begin{enumerate}
            #2
        \end{enumerate}
    \hrule
    \vspace{2mm}
    {\color{red}\textbf{解:}}
        \begin{enumerate}
            #3
        \end{enumerate}
}

% 問題テンプレ箇条書きver 最初のメッセージも書き換える.
\newcommand{\pbenumex}[4][undefined]{
    \vspace{2mm}
    \hrule
    \vspace{2mm}
    \refstepcounter{numpb}
    \label{pname:#1}
    \noindent
    \textbf{問\arabic{secn}--\arabic{numpb}:} #2
        \begin{enumerate}
            #3
        \end{enumerate}
    \hrule
    \vspace{2mm}
    {\color{red}\textbf{解:}}
        \begin{enumerate}
            #4
        \end{enumerate}
}


% 本文
\begin{document}
\pgs{自習用数学問題集}% 章立てページ
この資料は, 自習用の数学問題集です. 教科書の理論を追っていくような証明問題に比重を置いています. 易しくはありません.\\
問題の順番に規則性はほとんどなく, 分野はごっちゃ, 難易度順に並んでいるわけでもありません. ただし, 後の問題で使うような知識が前半に来ないように編集していく予定です.
最終的には, これ一つで数学の基礎内容を包含できればと夢見ています.
数学記号や用語の定義は, 教科書ごとに著しく差があれば問題文中で定義しますが, 多くの教科書で使われているものは一々定義することはありません.

また, この問題集の解答が正しい保証はありません. 参考にする場合は気を付けてください.
\pgsc{数理論理学}{midnightblue}{mlogic}

\begin{nmprob}
項と論理式の定義. 
\pbenumex{
変数記号, 定数記号, 関数記号, 命題記号, 述語記号, 論理記号, 補助記号を次表のように用意する.
\footnote[1]{記号論理学においては定義に素朴的な集合論が導入されたり, メタ的な考察に直感的な論法や数学的帰納法を用いることが多い.
結局のところ, 論理の構成には別の論理(メタ論理)が必要になり, そのまたメタ論理が必要に... となるため, これは仕方ない. ゼロから構成していくというより, 自動でシミュレートするコンパイラのようなものを人間の直観に従って作ることに近い.}
    \begin{table}[hbtp]
        \caption{論理式で使用される記号}
        \begin{center}
            \begin{tabular}{c|c}
                & 使用する記号 \\ \hline
                変数記号 & 英小文字 1 文字およびそれに $^{'}$ をつけたもの\\ \hline
                定数記号 & zero, unity, two, three \\ \hline
                関数記号 & $\bullet$, $\diamond$ suc \\ \hline
                命題記号 & A, B, C, D \\ \hline
                述語記号 & P, Q, R, = \\ \hline
                論理記号 & $\lnot, \land, \lor, \to, \forall, \exists, \bot$ \\ \hline
                補助記号 & (, ) \\ \hline
            \end{tabular}
        \end{center}
    \end{table}\\
ここで, 上の表の記号によって構成される文字列のうち, 項とよばれるものを次のように定義する.
\begin{itemize}
    \item 変数記号は項である.
    \item 定数記号は項である.
    \item $t_1, t_2$ が項ならば $(t_1 \bullet t_2),\ (t_1 \diamond t_2),\ \text{suc} (t_1)$ も項である. ただし, $t_1$ が変数記号または定数記号である場合は $\text{suc}\ t_1$ と省略可能である.
    \item 上に該当する記号列以外は項ではない.
\end{itemize}
さらに, 論理式とよばれるものを次のように定義する.
\begin{itemize}
    \item 命題記号および $\bot$ は論理式である.
    \item $t$ が項で $P$ が述語記号ならば $P(t)$ は全て論理式である.
    \item $t_1$ と $t_2$ が項ならば $(t_1 = t_2)$ は論理式である.
    \item $\phi$ と $\psi$ が論理式で $x$ が変数記号ならば $(\lnot \phi), (\phi \land \psi), (\phi \lor \psi), (\phi \to \psi), (\forall x \phi), (\exists x \phi)$
    はいずれも論理式である.
    \item 上に該当する記号列以外は論理式ではない. ただし, 論理式の一番外側に括弧がある場合は省略できる.
\end{itemize}
特に, 命題記号単体からなる論理式, 一つの述語記号からなる $P(t)$ の形の論理式, および論理式 $\bot$ を原始論理式とよぶ.
}{
\item $\forall x \forall y ((\text{suc}(x) = \text{suc}(y \diamond \text{two})) \to (x = (y \diamond \text{two})))$ は論理式か.
\item $\exists x(x)$ は論理式か.
}{
\item 論理式である.
\item $(x)$ は論理式でないので $\exists x(x)$ は論理式ではない.\footnote[2]{直感的に場合分けと背理法の論理を使って証明しているが, ここは仕方ない. 自動シミュレートを作った世界の中では厳密でなくてはならないが, メタ的には直感的な論理で証明される.\vspace{30pt}}
}
\setcounter{table}{0}
\end{nmprob}



\begin{nmprob}
自由変数, 束縛変数, 代入可能, 閉論理式, 閉項.
\pbenumex{
括弧を省略していない論理式中で, 次の下線を引いた $x$ のような変数記号の出現をを束縛出現という. また, 変数記号の出現で束縛出現でないものを自由出現という.
(ここで, 問題の(2)の $x$ のように束縛出現と自由出現は両立する場合があることに注意)
\begin{align*}
    \cdots (\forall \underline{x} (\cdots \underline{x} \cdots)) \cdots\\
    \cdots (\exists \underline{x} (\cdots \underline{x} \cdots)) \cdots
\end{align*}
また, 出現する変数記号全てが自由出現しないような項, 論理式をそれぞれ閉項, 閉論理式という.\\
ここで, $\phi$ を括弧を省略していない論理式, $x$ を変数記号, $t$ を項としたとき, 次の 2 条件を満たす変数記号 $y$ が存在するとき, $\phi$ の中の $x$ に $t$ は代入不可能であるという.
\begin{itemize}
    \item $\phi$ は $\cdots (\forall y (\cdots x \cdots)) \cdots$ または $\cdots (\exists y (\cdots x \cdots)) \cdots$ という形をしている(ただし, $x$ は $\phi$ 中で自由出現)
    \item 項 $t$ 中に変数記号 $y$ が出現する.
\end{itemize}
上の 2 条件を満たす変数記号 $y$ が存在しないとき, $\phi$ の中の $x$ に $t$ は代入可能であるという. また, $\phi$ の中の $x$ に $t$ が代入可能である場合に, $\phi$ 中の自由出現する $x$ を全て $t$ に置き換えて得られる論理式を $\phi [t/x]$ と表す($\phi [t/x]$ が論理式となることは (2) 参照)
}{
\item 全ての項に対して, その項に含まれる変数記号 $x$ を項 $t$ に置き換えたものは項となることを示せ.
\item 全ての論理式に対して, その論理式に含まれる全ての自由出現する $x$ を項 $t$ に置き換えたものは論理式となることを示せ.
\item 論理式 $\forall a \forall y ((\exists x (z = x)) \land (x \bullet \text{suc} (y))) [\text{zero}/x]$ を $[\cdots]$ による代入の表現がない形で表せ.
}{
\item 任意の項を一つとり, $s$ とする. 項の定義から, 任意の項は, 変数記号, 定数記号または $t_1, t_2$ を項として $(t_1 \bullet t_2),\ (t_1 \diamond t_2),\ \text{suc} (t_1)$ で構成されるもののみである.
まず, $s$ が 変数記号 $x$ である場合は $t$ に置き換えても項である. また, $s$ が $x$ 以外の変数記号である場合または定数記号である場合は, 置き換える変数記号 $x$ が存在しないため, $x$ を $t$ に置き換えることは何もしないことと同じで, そのまま項となる.
次に, 項 $t_1, t_2$ の $x$ を $t$ に置き換えたものも項になると仮定すると, $(t_1 \bullet t_2),\ (t_1 \diamond t_2),\ \text{suc} (t_1)$ いずれも $x$ を $t$ に置き換えたものも項となる.\footnote[1]{ここで直感的な数学的帰納法を用いているが, 仕方ない. ただし, 論理を構成して, 数学に入れば厳密化できる}
よって, 全ての項に対して, その項に含まれる変数記号 $x$ を項 $t$ に置き換えたものは項となる.
\item 論理式の定義から, 任意の論理式は, 原始論理式であるか, $\psi, \phi$ を論理式として$(\lnot \phi), (\phi \land \psi), (\phi \lor \psi), (\phi \to \psi), (\forall x \phi), (\exists x \phi)$ で構成されるもののみである.
まず, 原始論理式で $P(x)$ という形のものは, $P(t)$ としても論理式である. また, 原始論理式で $x = x$, $s = x$, $x = s$ という形($s$ は項)のものも $t$ を代入しても論理式である. さらに, 原始論理式で命題記号 や $\bot$ となっているものは $x$ を含まず, 論理式となる.
次に, 論理式 $\phi, \psi$ の 自由出現する $x$ を $t$ で置き換えた $\phi [t/x]$ および $\psi [t/x]$ がそれぞれ論理式だと仮定すると, 論理式 $(\lnot \phi), (\phi \land \psi), (\phi \lor \psi), (\phi \to \psi), (\forall x \phi), (\exists x \phi)$ の自由出現する $x$ に $t$ を代入した
$(\lnot \phi[t/x]),\ (\phi[t/x] \land \psi[t/x]),\ (\phi[t/x] \lor \psi[t/x]),\ (\phi[t/x] \to \psi[t/x]),\ (\forall x \phi[t/x]),\ (\exists x \phi[t/x])$ はいずれも論理式となる.
よって, 全ての論理式に対して, その論理式に含まれる全ての自由出現する $x$ を項 $t$ に置き換えたものは論理式となる.
\item $\forall a \forall y ((\exists x (z = x)) \land (\text{zero} \bullet \text{suc} (y)))$
}
\end{nmprob}



\begin{nmprob}
自然演繹の準備. 仮定集合.
\pbenumex{
証明とは,与えられた仮定を使い, 許された推論規則を適用して結論を得る道筋であり, これを可視化する証明図を 3 問かけて定義する.\\
まずは, 仮定集合の定義を行う.\\
制作中
}{
\item 制作中  
}{
\item 制作中
}
\end{nmprob}
 % 数理論理学
\pgsc{集合}{darkolivegreen}{set}

\begin{nmprob}
ZCの公理における空集合. 数理論理学との連係はまだ無理(勉強中)で簡単な略した自己流のものに留める.
\pbenum{
\item $\text{emp} (x) \iff \forall z,\ z\notin x$ と定義する. $\text{emp} (x) \land \text{emp} (y) \implies x = y$ を示せ.
\item $\exists y,\ \text{emp} (y)$ を示せ.
}{
\item 以下の証明図(省略済み)より, $\text{emp} (x) \land \text{emp} (y) \implies x = y$ が成り立つ.
ここでは, 外延性の公理「 $\forall x \forall y((\forall z,\ z \in x \leftrightarrow z \in y) \to x = y)$ 」を用いている.
\begin{prooftree}
    \AxiomC{$[\text{emp}(x) \land \text{emp}(y)]_1$}
    \UnaryInfC{$\forall z, z\notin x$}
    \UnaryInfC{$s \notin x$}
    \UnaryInfC{$\lnot (s \in x)$}
    \AxiomC{$[s \in x]_2$}
    \BinaryInfC{$\bot$}
    \UnaryInfC{$s \in y$}
    \RightLabel{{\scriptsize 2}}
    \UnaryInfC{$s \in x \to s \in y$}
    \AxiomC{$[\text{emp}(x) \land \text{emp}(y)]_1$}
    \UnaryInfC{$\forall z, z\notin y$}
    \UnaryInfC{$s \notin y$}
    \UnaryInfC{$\lnot (s \in y)$}
    \AxiomC{$[s \in y]_3$}
    \BinaryInfC{$\bot$}
    \UnaryInfC{$s \in x$}
    \RightLabel{{\scriptsize 3}}
    \UnaryInfC{$s \in y \to s \in x$}
    \BinaryInfC{$s \in x \leftrightarrow s \in y$}
    \UnaryInfC{$\forall z,\ z \in x \leftrightarrow z \in y$}
    \AxiomC{}
    \UnaryInfC{$\forall x \forall y((\forall z,\ z \in x \leftrightarrow z \in y) \to x = y)$}
    \UnaryInfC{$\forall y((\forall z,\ z \in x \leftrightarrow z \in y) \to x = y)$}
    \UnaryInfC{$(\forall z,\ z \in x \leftrightarrow z \in y) \to x = y$}
    \BinaryInfC{$x = y$}
    \RightLabel{{\scriptsize 1}}
    \UnaryInfC{$(\text{emp}(x) \land \text{emp}(y)) \to x = y$}
\end{prooftree}
\item 以下の証明図より $\exists y,\ \text{emp}(y)$ となる.
分出公理「 $\phi$ が $x, w_1, \ldots w_n$ を自由変数として含み, $y$ を自由変数として含まない論理式のとき, $\forall z \forall w_1, \ldots w_n \exists y \forall x (x \in y \leftrightarrow (x \in z \land \phi))$ 」
を用いている.
\begin{prooftree}
    \AxiomC{$[\forall x(x \in y_0 \leftrightarrow (x \in z \land x \neq x))]_1$}
    \UnaryInfC{$x \in y_0 \leftrightarrow (x \in z \land x \neq x)$}
    %\BinaryInfC{$\forall x(x \in y \leftrightarrow (x \in z \land x \neq x))$}
    \UnaryInfC{$x \in y_0 \to (x \in z \land x \neq x)$}
    \AxiomC{$[x \in y_0]_2$}
    \BinaryInfC{$x \in z \land x \neq x$}
    \UnaryInfC{$x \neq x$}
    \AxiomC{}
    \UnaryInfC{$x = x$}
    \BinaryInfC{$\bot$}
    \RightLabel{{\scriptsize 2}}
    \UnaryInfC{$x \notin y_0$}
    \UnaryInfC{$\forall x, x \notin y_0$}
    \UnaryInfC{$\exists y \forall x, x \notin y$}
    \AxiomC{}
    \UnaryInfC{$\forall z\exists y \forall x(x \in y \leftrightarrow (x \in z \land x \neq x))$}
    \UnaryInfC{$\exists y \forall x(x \in y \leftrightarrow (x \in z \land x \neq x))$}
    \RightLabel{{\scriptsize 1}}
    \BinaryInfC{$\exists y \forall x, x \notin y$}
    \UnaryInfC{$\vdots$}
    \UnaryInfC{$\exists y, \forall z, z \notin y$}
    \UnaryInfC{$\exists y,\ \text{emp}(y)$}
\end{prooftree}
}
\hspace{-3zw}{\color{forestgreen}●●\ まとめ\ ●●}\\
(1), (2) より $\text{emp}(x)$ となる $x$ がただ一つ存在する. この唯一の $x$ を $\emptyset$ で表し, 空集合とよぶ.
よって, $\forall z, z \notin \emptyset$ が成り立つ.
\end{nmprob}



\begin{nmprob}
Kuratowski の順序対を理解しよう.
\pbenum{
\item $\langle a, b\rangle$ を $\langle a, b\rangle \coloneqq \{ \{a\}, \{a, b\} \}$ と定めるとき,
\begin{align*}
    \langle a, b\rangle = \langle c, d\rangle \iff (a = c) \land (b = d)
\end{align*}
となることを示せ. (補足: 外延的記法において, 同一の元を重複して書くことは禁じられていない. 同じものをいくつ書いてもその効果はただ1つだけ書いたのと同じものとしている)
\item 順序対を拡張して $\bm{n}$\textbf{-対}を次のように定義する.
\begin{align*}
    \langle a_1 \rangle \coloneqq a_1,\hspace{3zw} \langle a_1, \cdots, a_{n-1}, a_n \rangle \coloneqq \langle \langle a_1, \cdots, a_{n-1} \rangle, a_n \rangle 
\end{align*}
このとき,
\begin{align*}
    \langle a_1, \cdots, a_n \rangle = \langle b_1, \cdots, b_n \rangle \iff a_1 = b_1 \land \cdots a_n = b_n
\end{align*}
が成り立つことを示せ.
}{
\item $\impliedby$ は明らか. $\implies$ は背理法で示す.\\
まず, $a \neq c$ と仮定すると, $\{a\} \neq \{c\}$ より, $\{a\} = \{c, d\}$ とならなくてはならない. このとき, $a = c = d$ となり, $a \neq c$ に矛盾. よって $a = c$ でなくてはならない.\\
次に $b \neq d$ を仮定する. $a = c$ より $\langle a, b\rangle = \langle c, d\rangle \implies \{ \{a\}, \{a, b\} \} = \{ \{a\}, \{a, d\} \}$ が成り立つ. ここで,
\begin{enumerate}
    \item $a = b$ のとき\\
    $\{ \{a\}, \{a, b\} \} = \{\{a\}\}$ より, $\{a, d\} = \{a\}$ となるから, $d = a = b$ で $b \neq d$ に矛盾. 
    \item $a \neq b$ のとき\\
    $\{a, d\} = \{a\}$ または $\{a, d\} = \{a, b\}$ でなくてはならない. ここで $\{a, d\} = \{a, b\}$ とすると, $b = d$ で矛盾. また, $\{a, d\} = \{a\}$ とすると, $d = a$ となり, 今度は
    $\{ \{a\}, \{a, d\} \} = \{\{a\}\}$ となり, $\{a, b\} = \{a\}$ より, $b = a = d$ より矛盾.
\end{enumerate}
以上より, $b = d$ より, $a = c \land b = d$ が成り立つ.
\item
\begin{enumerate}
    \item $\implies$\\
    帰納法により示す.
    $n = 1$ のときは明らか.\\
    次に $\langle a_1, \cdots, a_k \rangle = \langle b_1, \cdots, b_k \rangle \implies a_1 = b_1 \land \cdots a_k = b_k$ が成り立つと仮定すると,
    \begin{align*}
        \langle a_1, \cdots, a_{k+1} \rangle = \langle b_1, \cdots, b_{k+1} \rangle &\implies \langle \langle a_1, \cdots, a_k \rangle, a_{k+1} \rangle = \langle \langle b_1, \cdots, b_k \rangle, b_{k+1} \rangle \\
        &\implies \langle a_1, \cdots, a_k \rangle = \langle b_1, \cdots, b_k \rangle \land a_{k+1} = b_{k+1}\\
        &\implies a_1 = b_1 \land \cdots a_k = b_k \land a_{k+1} = b_{k+1}
    \end{align*}
    が成り立つことから, $\langle a_1, \cdots, a_n \rangle = \langle b_1, \cdots, b_n \rangle \implies a_1 = b_1 \land \cdots a_n = b_n$ が成り立つ.\\
    (集合の集合を認めれば, (1)において元が集合でも問題ない...はず)
    \item $\impliedby$\\
    上の議論を逆にたどればよい.
\end{enumerate}
以上より, $\langle a_1, \cdots, a_n \rangle = \langle b_1, \cdots, b_n \rangle \iff a_1 = b_1 \land \cdots a_n = b_n$ が成り立つ.
}
\end{nmprob}



\begin{nmprob}
二項関係に関する用語を理解する.
\pbenumex[brelation]{
その元が全て順序対であるような集合を二項関係という. 二項関係 $R$ の定義域 $\text{dom} (R)$ および値域 $\text{rng} (R)$ を次のように定義する. (このとき, 明らかに $R \subset \text{dom} (R) \times \text{rng} (R)$ である)
\begin{align*}
    \text{dom} (R) &\coloneqq \{ x\ |\ \exists y,\ \langle x, y\rangle \in R\}\\
    \text{rng} (R) &\coloneqq \{ y\ |\ \exists x,\ \langle x, y\rangle \in R\}
\end{align*}
また, $R$ による集合 $X$ の像 $R[X]$ を
\begin{align*}
    R[X] \coloneqq \{ y\ |\ \exists x\in X,\ \langle x, y\rangle \in R\}
\end{align*}
によって定義する(ここで, $X$ は $\text{dom} (R)$ の部分集合とは限らない)\\
このとき, 次を証明せよ. ただし, $\text{dom} (R) = A,\ \text{rng} (R) \subset B$ として, $X, X_1 ,X_2$ はそれぞれ $A$ の部分集合とする.
}{
\item $X_1 \subset X_2 \implies R[X_1] \subset R[X_2]$
\item $R[X_1 \cup X_2] = R[X_1] \cup R[X_2]$
\item $R[X_1 \cap X_2] \subset R[X_1] \cap R[X_2]$
\item $R[A - X] \supset R[A] - R[X]$
}{
\item $y \in R[X_1]$ とすると, $\langle x, y\rangle \in R$ となる $x \in X_1$ が存在する. 今, $X_1 \subset X_2$
より $x$ は $x \in X_2$ でもあるから, $y \in R[X_2]$ となる. よって, $R[X_1] \subset R[X_2]$ が成り立つ.
\item
\begin{enumerate}
\item $R[X_1 \cup X_2] \subset R[X_1] \cup R[X_2]$ を示す.\\
$y \in R[X_1 \cup X_2]$ とすると, $\langle x, y\rangle \in R$ となる $x \in X_1 \cup X_2$ が存在する.
この $x$ に関して\footnote[1]{$x$ は複数存在する可能性があるが 1 つについて示せば十分であることに注意. 以後の問題も同様.}, $x \in X_1$ のときは $y \in R[X_1]$, $x \in X_2$ のときは $y \in R[X_2]$ となるので, $y \in R[X_1] \cup R[X_2]$ より, $R[X_1 \cup X_2] \subset R[X_1] \cup R[X_2]$ が成り立つ.
\item $R[X_1 \cup X_2] \supset R[X_1] \cup R[X_2]$ を示す.\\
$y \in R[X_1] \cup R[X_2]$ とすると, $\langle x, y\rangle \in R$ となる $x \in X_1$ が存在する, または $\langle x, y\rangle \in R$ となる $x \in X_2$ が存在する. いずれの場合も $x \in X_1 \cup X_2$ となるので
$y \in R[X_1 \cup X_2]$ となるから, $R[X_1 \cup X_2] \supset R[X_1] \cup R[X_2]$
\end{enumerate}
以上より, $R[X_1 \cup X_2] = R[X_1] \cup R[X_2]$ が成り立つ.
\item $y \in R[X_1 \cap X_2]$ と仮定すると, $\langle x, y\rangle \in R$ となる $x \in X_1 \cap X_2$ が存在する. この $x$ に関して $x \in X_1$ かつ $x \in X_2$ であるから, $y \in R[X_1] \cap R[X_2]$ となる.
よって $R[X_1 \cap X_2] \subset R[X_1] \cap R[X_2]$ となる. \\
ちなみに, $X_1 \cap X_2 = \emptyset$ かつ $R[X_1] \cap R[X_2] \neq \emptyset$ となるように $X_1, X_2$ を上手く設定すると, $R[X_1 \cap X_2] \not\supset R[X_1] \cap R[X_2]$ となる例が見つかる.
\item $y \in R[A] - R[X]$ とすると, $y \in R[A]$ かつ $y \notin R[X]$ となる. $y \in R[A]$ より $\langle x, y\rangle \in R$ となる $x \in A$ が存在する. ここで $x \in X$ と仮定すると, $y \in R[X]$ となってしまうので, $x \notin X$ となる.
よって $x \in A - X$ より, $y \in R[A - X]$ となる. 以上より, $R[A - X] \supset R[A] - R[X]$ が成り立つ.\\
ちなみに $A - X \neq \emptyset$ かつ $R[A] - R[X] = \emptyset$ となるように $X$ を設定すると, $R[A - X] \not\subset R[A] - R[X]$ となる例が見つかる.
}
\end{nmprob}



\begin{nmprob}
逆関係, 合成関係, 写像に関する用語を理解する.
\pbenum[finv]{
\item 関係 $R \subset A \times B$ が与えられたとき, その逆関係 $R^{-1}$ を $R^{-1} \coloneqq \{ \langle y, x\rangle\ |\ \langle x, y\rangle \in R\}$ により定める.
$R^{-1}[Y_1 \cap Y_2] \subset R^{-1}[Y_1] \cap R^{-1}[Y_2]$ を示せ. ただし, $Y_1, Y_2 \subset B$ とする.
\item 2つの関係 $R, S$ が与えられたとき, その合成関係 $R \circ S$ を $R \circ S \coloneqq \{ \langle x, z\rangle \ |\ \exists y,\ \langle x, y\rangle \in S \land \langle y, z\rangle \in R\}$ により定める.
$(R \circ S)^{-1} = S^{-1} \circ R^{-1}$ を示せ.
\item 関係 $R$ が $\langle x, y\rangle \in R \land \langle x, z\rangle \in R \implies y = z$ をみたすとき, $R$ を写像とよぶ.
特に $\text{dom} (R) = A,\ \text{rng} (R) \subset B $ のときは, $R$ は $A$ から $B$ への写像\footnote[1]{$A$ を始集合, $B$ を終集合とよぶ.}と呼ぶ. $R$ が写像のとき, (1) において $R^{-1}[Y_1 \cap Y_2] \supset R^{-1}[Y_1] \cap R^{-1}[Y_2]$ を示せ(つまり(1)とから $R$ が写像であれば $R^{-1}[Y_1 \cap Y_2] = R^{-1}[Y_1] \cap R^{-1}[Y_2]$ が成り立つ)
\item $f$ が $A$ から $B$ への写像であるとき, $A (= \text{dom}(f))$ の元 $x$ に関して $\langle x, y\rangle \in f$ となる $y \in B$ がただ一つ存在する. これを $f$ による $x$ の像とよび, $f(x)$ と表す.
ここで, $(\forall x, x^{'} \in A)[x \neq x^{'} \implies f(x) \neq f(x^{'})]$ が成り立つとき $f$ を単射とよぶ.\footnote[2]{$f$ が単射であることを示す際に, 対偶である $(\forall x, x^{'} \in A)[f(x) = f(x^{'}) \implies x = x^{'}]$ を示す場合が非常に多い.\\ また, $f$ が写像であるもとで, $f$ が単射であることと逆関係 $f^{-1}$ が写像であることは本質的に同じ.}
また, $\text{rng} (f) = B$ のとき, $f$ を全射といい, 全射かつ単射のとき全単射という. $f$ が $A$ から $B$ への全単射のとき, 逆関係 $f^{-1}$ が $B$ から $A$ への全単射の写像\footnote[3]{これを $f$ の逆写像とよぶ}であることを示せ.}{
\item $R^{-1} \subset B \times A$ より, 問\ref{sec:set}-\ref{pname:brelation} (3) において記号を置き換えればすぐに示される.
\footnote[4]{問\ref{sec:set}-\ref{pname:brelation}の他の3つも同様である.}
\item $\langle z, x\rangle \in (R \circ S)^{-1}$ とすると, $\langle x, z\rangle \in R \circ S$ より, $\langle x, y\rangle \in S \land \langle y, z\rangle \in R$ となる $y$ が存在する.
この $y$ に対して, $\langle z, y\rangle \in R^{-1}$ かつ $\langle y, x\rangle \in S^{-1}$ より, $\langle z, x\rangle \in S^{-1} \circ R^{-1}$ となる. よって $(R \circ S)^{-1} \subset S^{-1} \circ R^{-1}$ となる.
また, 同様の議論を逆にたどることにより, $S^{-1} \circ R^{-1} \subset (R \circ S)^{-1}$ となる. 以上より, $(R \circ S)^{-1} = S^{-1} \circ R^{-1}$ が成り立つ.
\item $x \in R^{-1}[Y_1] \cap R^{-1}[Y_2]$ とすると, $(\exists y\in Y_1,\ \langle y, x\rangle \in R^{-1})\land (\exists y^{'}\in Y_2,\ \langle y^{'}, x\rangle \in R^{-1})$ が成り立つ. $y, y^{'}$ に関して,
$\langle x, y\rangle \in R$ かつ $\langle x, y^{'}\rangle \in R$ が成り立つから, $R$ が写像であることより $y = y^{'}$ となる. よって $\exists y \in Y_1 \cap Y_2,\ \langle y, x\rangle \in R^{-1}$ が成り立つから, $x \in R^{-1}[Y_1 \cap Y_2]$ より, $R^{-1}[Y_1 \cap Y_2] \supset R^{-1}[Y_1] \cap R^{-1}[Y_2]$ となる.
\item $\text{dom} (f^{-1}) = B,\ \text{rng} (f^{-1}) = A$ であることと $f^{-1}$ が写像であること, および $f^{-1}$ が単射であることを示せばよい.
まず,
\begin{align*}
    \text{dom} (f^{-1}) &= \{ y\ |\ \exists x,\ \langle y, x\rangle \in f^{-1}\}\\
    &= \{y\ |\ \exists x,\ \langle x, y\rangle \in f \}\\
    &= \text{rng} (f)\\
    &= B\hspace{1zw}\text{($\because$\ $f$ が全射より)}
\end{align*}
より, $\text{dom} (f^{-1}) = B$ となり, 同様にして $\text{rng} (f^{-1}) = A$ が成り立つ.\\
次に, $\langle y, x\rangle \in f^{-1}$ かつ $\langle y, x^{'} \rangle \in f^{-1}$ とすると, $\langle x, y\rangle \in f$ かつ $\langle x^{'}, y\rangle \in f$ であり,  $f$ が写像であることから $y = f(x) = f(x^{'})$ となる.
ここで $f$ は単射であることから, $x = x^{'}$ となるから $f^{-1}$ は写像となる.\\
最後に $f^{-1}$ が写像であることから, $\langle y, f^{-1}(y) \rangle \in f^{-1}$ かつ $\langle y^{'}, f^{-1}(y^{'}) \rangle \in f^{-1}$ より, $\langle f^{-1}(y), y\rangle \in f$ かつ $\langle f^{-1}(y^{'}), y^{'} \rangle \in f$ となる.
ここで, $f^{-1}(y) = f^{-1}(y^{'})$ と仮定すると, $f$ が写像であることから, $y = y^{'}$ となるから $f^{-1}$ は単射となる.\\
以上より, $f^{-1}$ は $B$ から $A$ への全単射となる.
} 
\end{nmprob}



\begin{nmprob}
合成写像と二項関係.
\pbenum[fog]{
\item 写像 $f: A\to B,\ g: B\to C$ の合成関係 $g\circ f$ は $A$ から $C$ への写像であることを示せ.
\item 写像 $f: A\to B$ と $g: B\to C$ がそれぞれ全単射のとき, $g\circ f$ は $A$ から $C$ への全単射となることを示せ.
\item $X$ から $X$ への写像で, 任意の $x \in X$ に対して, その像が $x$ となるものを恒等写像といい $I_X$ で表す.
$I_X$ は全単射であることを示せ.
}{
\item
\begin{enumerate}
\item $\text{dom}(g\circ f) = A$ を示す.\\
$x \in \text{dom}(g\circ f)$ とすると, $\langle x, z \rangle \in g\circ f$ となる $z$ が存在するから, さらに
$\langle x, y\rangle \in f$ となる $y$ が存在する. よって, $x \in \text{dom}(f) = A$ となるから, $\text{dom}(g\circ f) \subset A$ となる.\\
また, $x \in A$ とすると, $A = \text{dom}(f)$ より, $\langle x, y\rangle \in f$ となる $y$ が存在するから, 同様に $\langle y, z\rangle \in g$ となる $z$ が存在する.
これより, $\langle x, z\rangle \in g\circ f$ より, $x \in \text{dom}(g\circ f)$ より, $A \subset \text{dom}(g\circ f)$ となる.\\
以上より, $A = \text{dom}(g\circ f)$ となる.
\item $\text{rng}(g\circ f) \subset C$ を示す.\\
$z \in \text{rng}(g\circ f)$ とすると, $\langle x, z\rangle \in g\circ f$ となる $x$ が存在するから, $\langle y, z\rangle \in g$ となる $y$ が存在する. よって $z \in \text{rng}(g) \subset C$ より, $z \in C$ となる.
よって, $\text{rng}(g\circ f) \subset C$ となる.
\item $\langle x, z\rangle \in g\circ f \land \langle x, z^{'}\rangle \in g\circ f \implies z = z^{'}$ を示す.\\
$\langle x, z\rangle \in g\circ f \land \langle x, z^{'}\rangle \in g\circ f$ とすると, $\langle x, y\rangle \in f \land \langle y, z\rangle \in g$ となる $y$ と
$\langle x, y^{'}\rangle \in f \land \langle y^{'}, z^{'}\rangle \in g$ となる $y^{'}$ が存在する. ここで $f$ が写像より, $y = y^{'}$ であり, $g$ も写像であるから $z = z^{'}$ となる.
\end{enumerate}
以上より, $g\circ f$ は $A$ から $C$ への写像である.\footnote[1]{合成関係が写像となっているときは, 合成写像とよぶ.}
\item
(1)より, $g\circ f$ は $A$ から $C$ への写像であるので, $g\circ f$ が全単射であることを示す.
\begin{enumerate}
\item 全射性を示す.\\
$z \in C$ とすると, $g$ が全射より, $\langle y, z\rangle \in g$ となる $y \in B$ が存在する. ここで, $f$ も全射より, この $y$ に対して, $\langle x, y\rangle \in f$ となる $x \in A$ が存在する.
これより, $\langle x, z\rangle \in g \circ f$ となる $x$ が存在することから $z \in \text{rng}(g \circ f)$ となり, $\text{rng}(g \circ f) = C$ となる. よって, $g\circ f$ は全射となる.
\item 単射性を示す.\\
$(g\circ f)(x) = (g\circ f)(x^{'})$ とすると, $g$ が単射より, $f(x) = f(x^{'})$ であり, $f$ も単射であることから $x = x^{'}$ となる. よって, $g\circ f$ は単射である.
\end{enumerate}
以上より, $g\circ f$ は全単射である.
\item
\begin{enumerate}
\item 全射性を示す.\\
$x \in X$ とすると, $I_X$ が写像より, $\langle x, y\rangle \in I_X$ となる $y$ がただ一つ存在し, 恒等写像の定義より $y = x$ である.
よって, $\forall x \in X,\ \langle x, x\rangle \in I_X$ となるから, $X \subset \text{rng}(I_X)$ より, $X = \text{rng}(I_X)$ で $I_X$ は全射である.
\item 単射性を示す.\\
任意の $x, x^{'} \in X$ に対して, $I_X(x) = I_X(x^{'})$ とすると, $I_X(x) = x,\ I_X(x^{'}) = x^{'}$ より, $x = x^{'}$ となるから, $I_X$ は単射である.
\end{enumerate}
以上より, $I_X$ は全単射である.
}
\end{nmprob}



\begin{nmprob}
逆写像と二項関係.
\pbenum[comp_inv]{
\item 写像 $f: A\to B$ と $g: B\to C$ の合成写像 $g\circ f: A\to C$ が全単射であるとき, $g$ は全射であり, $f$ は単射であることを示せ.
\item 写像 $f: A\to B$ が全単射のとき, $f^{-1} \circ f = I_A,\ f \circ f^{-1} = I_B$ を示せ.
\item 写像 $f: A\to B,\ g: B\to C,\ h: C\to D$ に対して $h\circ (g\circ f) = (h \circ g)\circ f$ を示せ.
\item 写像 $f: A\to B,\ g: B\to A$ に対して, $f\circ g = I_B,\ g\circ f= I_A$ であるとき, $f, g$ はそれぞれ全単射であり,
$g = f^{-1},\ f = g^{-1}$ となることを示せ.
}{
\item
\begin{enumerate}
\item $g$ が全射であることを示す.\\
$C \subset \text{rng}(g)$ を示せばよい. $z \in C$ とすると, $g \circ f$ が全単射であるから, $\langle x, z\rangle \in g \circ f$ となる $x \in A$ が存在する.
そして, この $x, z$ に対して, $\langle x, y\rangle \in f \land \langle y, z\rangle \in g$ となる $y$ が存在する( $f$ が写像より $y = f(x)$ となる)
よって $z \in C \implies \exists y, \langle y, z\rangle \in g$ より, $z \in \text{rng}(g)$ となり, $g$ は全射となる.
\item $f$ が単射であることを示す.\\
任意の $x, x^{'} \in A$ に対して, $f(x) = f(x^{'})$ と仮定すると, $g\circ f(x) = g\circ f(x^{'})$ となる. ここで, $g \circ f$ は単射であるから, $x = x^{'}$ となる. よって $f$ は単射となる.
\end{enumerate}
\item
まず, $f^{-1}\circ f\subset I_A$ を示す.\\
$\langle x, y\rangle \in f^{-1}\circ f$ とすると, 問\ref{sec:set}-\ref{pname:fog}\ (1) より, $f^{-1}\circ f$ は $A$ から $A$ への写像であるから, $x, y\in A (= \text{dom}(g\circ f))$ であり,
また, $\langle x, z\rangle \in f\land \langle z, y\rangle \in f^{-1}$ となる $z$ が存在し, $\langle z, x\rangle \in f^{-1} \land \langle z, y\rangle \in f^{-1}$ となる. ここで, 問\ref{sec:set}-\ref{pname:finv} (4)より,
$f^{-1}$ は $B$ から $A$ への写像であるから, $x = y$ となる. よって, $\langle x, y\rangle \in f^{-1}\circ f$ ならば $x = y$ より, $\langle x, y\rangle \in I_A$ となるので $f^{-1}\circ f \subset I_A$ となる.\\
次に, $I_A \subset f^{-1}\circ f$ を示す.\\
$\langle x, y\rangle \in I_A$ とすると, $x \in \text{dom}(I_A) = A$ より, $\langle x, f(x)\rangle \in f \land \langle f(x), x\rangle \in f^{-1}$ となるから $\langle x, x\rangle \in f^{-1}\circ f$ となる.
また, $I_A$ は恒等写像であるから $x = y$ となる. よって, $\langle x, y\rangle \in f^{-1}\circ f$ となるので $I_A \subset f^{-1}\circ f$ となる.\\
以上より, $f^{-1}\circ f = I_A$ となる. $f\circ f^{-1} = I_B$ も同様に示せる.
\item
$\langle x, w\rangle \in h\circ (g\circ f)$ とすると, $\langle x, z\rangle \in g \circ f,\ \langle z, w\rangle \in h$ となる $z$ が存在し,
$\langle x, y\rangle \in f,\ \langle y, z\rangle \in g$ となる $y$ が存在する. よって, $\langle y, w\rangle \in h\circ g$ となり,
$\langle x, w\rangle \in (h \circ g) \circ f$ となる. よって, $h \circ (g \circ f) \subset (h \circ g)\circ f$ となる.
同様に, $h \circ (g \circ f) \supset (h \circ g)\circ f$ となるので, $h \circ (g \circ f) = (h \circ g)\circ f$ となる.\footnote[1]{この結合法則は一般の二項関係においても成り立つ.}
\item まず, 問\ref{sec:set}-\ref{pname:fog}(3) より, $I_A, I_B$ は共に全単射である. よって (1) より $f,\ g$ は共に全単射である.\\
次に, (2)より, $f\circ g = I_B$ ならば $f^{-1}\circ (f\circ g) = f^{-1}\circ I_B$ から $g = f^{-1}$ となる. 同様に, $f = g^{-1}$ となる.
}
\end{nmprob}

\begin{nmprob}
順序関係と順序集合
\makeatletter\tagsleft@true\makeatother
\pbenumex[ordr]{
$R$ がある集合 $A$ 上の二項関係であるとき, $\langle x, y\rangle \in R$ であることを以後 $xRy$ と表す.
}{
\item $\leq$ が 集合 $A$ 上の二項関係であり, 任意の $x, y, z \in A$ に対して, 次の $(O1) \sim (O3)$ を満たすとき, $\leq$ は $A$ 上の順序(関係)であるといい,
集合 $A$ 上に一つの順序関係 $\leq$ が定められているとき, $\langle A, \leq \rangle$ を順序集合という.
\begin{align}
    x \leq x & & &(\text{反射律})\tag{O1}\\
    x \leq y \land y \leq x & \implies x = y & &(\text{反対称律})\tag{O2}\\
    x \leq y \land y \leq z & \implies x \leq z & &(\text{推移律})\tag{O3}
\end{align}
$x \leq y \land x \neq y$ が成り立つとき, またそのときに限り $x < y$ と二項関係 $<$ を定めれば, 次の (O4), (O5) が成り立つことを示せ.
\begin{align}
    x < y & \implies y \nless x \coloneqq \lnot (y < x) \tag{O4}\\
    x < y \land y < z & \implies x < z \tag{O5}
\end{align}
\item (1) とは逆に (O4), (O5) を満たす二項関係 $<$ が集合 $A$ 上に与えられたとき, $x < y \lor x = y$ が成り立つとき, またその時に限り $x \leq y$ と二項関係 $\leq$ を
定めれば $\leq$ は順序関係になることを示せ.
\item 順序集合 $\langle A, \leq \rangle$ において, 集合 $A$ の任意の二元 $x, y$ に関して $x \leq y$ または $y \leq x$ が成り立つとき, $\leq$ は全順序(関係)といい, $\langle A, \leq \rangle$ は全順序集合という.
$\langle A, \leq \rangle$ が全順序集合のとき, $x \nless y \iff y \leq x$ を示せ. 
}{
\item まず, $x < y$ が成り立つ下で $y < x$ が成り立つと仮定すると, $x \leq y \land y \leq x$ より (O2) から $x = y$ となるが, これは $x \neq y$ に矛盾する. よって (O4) が成り立つ.\\
次に, $x < y \land y < z$ が成り立つと仮定すると, $x \leq y \land y \leq z$ が成り立つから, (O3) より $x \leq z$ が成り立つ. ここで, $x = z$ が成り立つと仮定すると, $z < y \land y < z$ が成り立ち, $y = z$ となるが,
これは $y \neq z$ に矛盾する. よって $x \neq z$ となり, $x < z$ より (O5) が成り立つ.
\item
\begin{enumerate}
\item 反射律の成立\\
$x = x$ が常に成立ことから, $x < x \lor x = x$ も常に成立する. よって, 反射律が成立する.
\item 反対称律の成立\\
$x \leq y \land y \leq x$ が成り立つが $x \neq y$ だと仮定すると, $x < y \land y < x$ が成り立つ. これより, (O5) から $x < x$ となるが, (O4) より $(x < x) \land \lnot(x < x)$ が成り立ち, 矛盾式が導出される. よって $x = y$ より 反対称律が成立する.
\item 推移律の成立\\
$x \leq y \land y \leq z$ が成り立つと仮定すると, $(x < y \lor x = y) \land (y < z \lor y = z)$ が成り立つ. $x < y \land y < z$ の場合は (O5) より $x < z$ より $x \leq z$ となり,
$x < y \land y = z$ の場合は $x < z$ で $x \leq z$ となる. $x = y \land y < z$ の場合も $x < z$ より $x \leq z$ であり, $x = y \land y = z$ の場合は $x = z$ で $x \leq z$ となるから, 推移律が成り立つ.
\end{enumerate}
\item $x \nless y$ とすると, $x \nleq y \lor x = y$ となる. $x \nleq y$ の場合は $\leq$ が全順序より, $y \leq x$ であり, $x = y$ の場合も $y \leq x$ となるから $x \nless y \implies y \leq x$ となる.\\
$y \leq x$ として, $x < y$ が成り立つと仮定する. $x < y$ より, $x \neq y$ であるから $y \leq x$ より $y < x$ が成り立つ. よって $x < y \land y < x$ が成り立つが, これは(1)で述べた通り $x \neq y$ に矛盾する. よって $y \leq x \implies x \nless y$ となる.\\
以上より, $x \nless y \iff y \leq x$ が成り立つ.
}
\makeatletter\tagsleft@false\makeatother
\end{nmprob}



\begin{nmprob}
最大元, 極大元, 部分順序集合, 上界, 上限.
\pbenumex{
$\langle A, \leq \rangle$ を一つの与えられた順序集合とする.
$A$ の空でない部分集合 $M$ を考える. $a,b \in M$ に関して $a \leq b$ が成り立つとき, またその時に限り $a \leq_M b$ と二項関係 $\leq_M$ を定めると, これは明らかに $M$ 上の順序関係となる.
ここで $\langle M, \leq_M \rangle$ を $\langle A, \leq \rangle$ の部分順序集合といい, 誤解が生まれない場合には $\langle M, \leq \rangle$ で表す.

}{
\item $A$ に一つの元 $a$ があって, $\forall x \in A,\ x \leq a$ が成り立つとき, $a$ を $A$ の最大元といい $\max A$ と表す.
最大限が任意の順序集合の台集合\footnote[1]{順序集合 $\langle A, \leq \rangle$ における集合 $A$ をその台集合という. また, 順序関係を明示しなくても誤解が生まれない場合は $A$ のことを順序集合とよぶことがある.}
に存在するとは限らないが, 存在すれば一意的に定まることを示せ.
\item $A$ の元 $a$ に関して, $\lnot (\exists x \in A,\ a < x)$ が成り立つとき, $a$ を $A$ の極大元という.
極大元の存在やその一意性は一般には保証されないが, $\max A$ が存在するとき, $a = \max A$ となることを示せ. 
\item $A$ が全順序集合のとき, $a$ が $A$ の最大限であることと $a$ が $A$ の極大元であることは同値であることを示せ.
\item $A$ の元 $a$ 関して, $\forall x \in M,\ x \leq a$ が成り立つとき, $a$ を $M$ の $A$ における上界という.
$M$ の $A$ における上界全体の集合を $M^*$ とすると, $M^* \neq \emptyset$ のとき, $M$ は $A$ において上に有界という.
$M$ が $A$ において上に有界かつ $\min M^*$ が存在するとき, $\min M^*$ を $M$ の $A$ における上限といい $\sup M$ と表す.
$\sup M$ が存在すれば一意的であることを示せ.
\item $\sup M$ が存在するという下で, $\sup M \in M \iff (\max M (\in M) \text{が存在})$ を示せ.
}{
\item $\max A$ が存在するとし, $a = \max A,\ a^{'} = \max A$ かつ $a \neq a^{'}$ とすると,
$a, a^{'}$ 共に $A$ の最大限より $a \leq a^{'}$ かつ $a^{'} \leq a$ より $a = a^{'}$ となり矛盾する.
よって, $\max A$ が存在すれば, それは一意に定まる.
\item $a = \max A$ とし, $a$ が $A$ の極大元でないとする. $a$ は $A$ の極大元ではないから $a < x$ となる $x \in A$ が存在する.
これより $a \leq x$ かつ $a \neq x$ となるが, $a = \max A$ より $x \leq a$ でもあるから $a = x \land a \neq x$ となり矛盾する.
よって $a = \max A$ は $A$ の極大元となる.
\item
$A$ が全順序集合のとき, 
\begin{align*}
    \forall x \in A,\ x\leq a &\iff \forall x \in A,\ \lnot (a < x)&&\text{($\because$ 問\ref{sec:set}-\ref{pname:ordr}\ (3)より)}\\
    &\iff \lnot (\exists x \in A,\ a < x)
\end{align*}
より, $A$ が全順序集合のとき, $a$ が $A$ の最大限となることと, $A$ の極大元になることは同値である.
\item $\sup M$ が存在すれば $\min M^*$ が存在し, 最小元の一意性より $\sup M$ は一意的に定まる.
\item $a \in A$ に関して 
\begin{empheq}[left={a = \sup M \iff \empheqlbrace}]{alignat=2}
    & \forall x \in M,\ x \leq a && \quad (\because \text{$a$ は $M$ の上界}) \tag{a}\\
    & \forall x^{'} \in M^*,\ a \leq x^{'} && \quad (\because \text{$a$ は上界の最小値}) \tag{b}
\end{empheq}
となるから, $\sup M \in M$ とすると (a) より, $\max M = \sup M$ となる. また, $a = \max M$ とすると, 明らかに (a) が成り立ち, また, $a \in M$ より (b) も成り立つ.
よって, $a = \sup M$ となり, $\sup M = \max M$ より, $\sup M \in M$ となる.
}
\newpage
\hspace{-3zw}{\color{forestgreen}●●\ 最大元, 上限の補足\ ●●}

最大元と上限についてまとめると次の表のようになる(最小元, 下限も同様)
\begin{table}[hbtp]
    \caption{$\max M, \sup M$ の存在}
    \begin{center}
        \begin{tabular}{c|c|c}
            $\max M$ の存在 & $\sup M$ の存在 & 説明 \\ \hline \hline
            存在する & 存在する & $\sup M = \max M,\ \sup M \in M$\\ \hline
            存在する & 存在しない & ありえない.\\ \hline
            存在しない & 存在する & $\sup M \notin M$ となる.\\ \hline
            存在しない & 存在しない & 起こりうる.\\ \hline
        \end{tabular}
    \end{center}
\end{table}

\begin{itemize}
    \item $\max M$ が存在 $\implies$ $\sup M = \max M$
    \item $\max M$ が存在しない $\implies$ $\sup M$ が存在し $\sup M \notin M$ , または $\sup M$ は存在しない.
    \item $\sup M$ が存在し, $\sup M \in M$ $\implies$ $\max M = \sup M$
    \item $\sup M$ が存在し, $\sup M \notin M$ $\implies$ $\max M$ は存在しない.
    \item $\sup M$ が存在しない $\implies$ $\max M$ は存在しない.
\end{itemize}
\end{nmprob}
\setcounter{table}{0}



\begin{nmprob}
順序写像, 順序単射, 順序同型写像, 順序同型
\pbenum{
\item 2 つの順序集合 $\langle A, \leq \rangle$ および $\langle A^{'}, \leq^{'} \rangle$ を考える. $f: A \to A^{'}$ で $a, b \in A$ に対して $a \leq b \implies f(a) \leq^{'} f(b)$ となるとき,
$f$ は $\langle A, \leq \rangle$ から $\langle A^{'}, \leq^{'} \rangle$ への順序写像とよばれる.\footnote[1]{誤解がない生まれない場合には $A$ から $A^{'}$ への順序写像や単に順序写像という場合がある}
$f$ が順序写像で, さらに $a, b \in A$ に対して $f(a) \leq^{'} f(b) \implies a \leq b$ となるとき,
$f$ は単射となることを示せ.\footnote[2]{この $f$ を $\langle A, \leq \rangle$ から $\langle A^{'}, \leq^{'} \rangle$ への順序単射という. これも誤解の生まれない場合には $A$ から $A^{'}$ への順序単射や単に順序単射という場合がある.}
\item $f$ が $A$ から $A^{'}$ への順序単射かつ $f$ が全射のとき, $f$ は $\langle A, \leq \rangle$ から $\langle A^{'}, \leq^{'} \rangle$ への順序同型写像とよばれる.\footnote[3]{注1, 2と同様.}
$f$ が $A$ から $A^{'}$ への順序同型写像のとき, $f^{-1}$ は $A^{'}$ から $A$ への順序同型写像となることを示せ.
\item $\langle A, \leq \rangle$ から $\langle A^{'}, \leq^{'} \rangle$ への順序同型写像が存在するとき, 両者は順序同型であるといい, $\langle A, \leq \rangle \simeq \langle A^{'}, \leq^{'} \rangle$ と表す(または略して $A \simeq A^{'}$ と表す)
$\langle A, \leq \rangle \simeq \langle A, \leq \rangle$ となることを示せ.
\item $\langle A, \leq \rangle \simeq \langle A^{'}, \leq^{'} \rangle \implies \langle A^{'}, \leq^{'} \rangle \simeq \langle A, \leq \rangle$ となることを示せ.
\item $\langle A, \leq \rangle \simeq \langle A^{'}, \leq^{'} \rangle \land \langle A^{'}, \leq^{'} \rangle \simeq \langle A^{''}, \leq^{''} \rangle \implies \langle A, \leq \rangle \simeq \langle A^{''}, \leq^{''} \rangle$ となることを示せ.
}{
\item $f(a) = f(b)$ とすると, $f(a) \leq^{'} f(b)$ かつ $f(b) \leq^{'} f(a)$ となるから, $a \leq b$ かつ $b \leq a$ より $a = b$ となる.
よって, $f$ は単射となる.
\item $f: A \to A^{'}$ は全単射より, \ref{sec:set}-\ref{pname:finv}\ (4) より $f^{-1}$ は $A^{'}$ から $A$ への全単射写像である.
ここで, $a^{'}, b^{'} \in A^{'}$ とすると, $f$ が全射より, $\langle a, a^{'} \rangle \in f,\ \langle b, b^{'} \rangle \in f$ となる $a, b \in A$ が存在する. ここで, $f$ は写像より, $a^{'} = f(a),\ b^{'} = f(b)$ となる.
また, $\langle a^{'}, a\rangle \in f^{-1}$ で $a = f^{-1}(a^{'})$ となり, 同様に $b = f^{-1}(b^{'})$ となる.\\
よって, $a^{'} \leq^{'} b^{'}$ とすると, $f(a) \leq^{'} f(b)$ より, $f$ が順序同型写像であることとから $f^{-1}(a^{'}) \leq f^{-1}(b^{'})$ となり, $f^{-1}$ は順序写像となる.
また, さらに $f^{-1}(a^{'}) \leq f^{-1}(b^{'})$ とすると, $a \leq b$ より, $f$ の順序単射性より, $a^{'} \leq b^{'}$ となり, $f^{-1}$ は順序単射となる.
ここで, $f^{-1}$ は全(単)射であったから $f$ は順序同型写像となる.
\item $I_A$ が $\langle A, \leq \rangle$ から $\langle A, \leq \rangle$ への順序同型写像となる. よって, $A \simeq A^{'}$ となる.
\item $\langle A, \leq \rangle$ から $\langle A^{'}, \leq^{'} \rangle$ への順序同型写像の一つを $f$ とすると, (3) より, $f^{-1}$ は $\langle A^{'}, \leq^{'} \rangle$ から $\langle A, \leq \rangle$ への順序同型写像となる.
よって, $A \simeq A^{'} \implies A^{'} \simeq A$ となる.
\item $A$ から $A^{'}$ への順序同型写像の一つを $f$, $A^{'}$ から $A^{''}$ への順序同型写像の一つを $f^{'}$ とし, $f^{''} = f^{'} \circ f$ とする.
まず, 問\ref{sec:set}-\ref{pname:fog}\ (2) より $f^{''}$ は $A$ から $A^{''}$ への全単射である. 次に $a \leq b$ とすると, $f(a) \leq f(b)$ より $f^{'}(f(a)) \leq f^{'}(f(b))$, すなわち $f^{''}(a) \leq f^{''}(b)$ となり, $f^{''}$ は順序写像となる.
同様に, $f^{''}(a) \leq f^{''}(b)$ とすると, $a \leq b$ となり, $f^{''}$ は順序単射となるから, 全射であることとから $f^{''}$ は $A$ から $A^{''}$ への順序同型写像となる.
よって, $A \simeq A^{'} \land A^{'} \simeq A^{''} \implies A \simeq A^{''}$ となる.
}
\end{nmprob}



\begin{nmprob}
選択公理を理解する.
\pbenum{
\item $\Lambda$ から $A_\lambda$ への写像を 集合族$(A_\lambda)_{\lambda \in \Lambda}$という. ここで, $\Lambda$ から $A_\lambda$ への写像
$a$ のうち, $a_\lambda \in A_\lambda$ を満たすものの全体を集合族 $(A_\lambda)_{\lambda \in \Lambda}$ の直積といい, $\prod_{\lambda \in \Lambda}A_\lambda$ で
表す. 今 $\prod_{n \in \bm{N}} A_n$ を $(A_1, A_2, \cdots, A_n)$ と表すとき, $( A_1, A_2, \cdots, A_n)$ は $n$-対の性質を持つことを示せ(ほぼ明らか)
\item 選択公理(AC)
\begin{align*}
    \Lambda \neq \emptyset \land \forall \lambda \in \Lambda,\ A_\lambda \neq \emptyset \implies \prod_{\lambda \in \Lambda} A_\lambda \neq \emptyset
\end{align*}
から従属選択公理(DC)\\
\begin{align*}
    \begin{cases}
        A \neq \emptyset\\
        (\forall x \in A)[\exists y \in A,\ \langle x, y\rangle \in R]\hspace{2zw}(R \subset A \times A)
    \end{cases}
    \implies \exists f: \bm{N} \to A,\ \langle f(n), f(n+1)\rangle \in R    
\end{align*}
を示せ.
\item 従属選択公理(DC)から可算選択公理(CC)
\begin{align*}
    \forall n \in \bm{N},\ A_n \neq \emptyset \implies \prod_{n \in \bm{N}} A_n \neq \emptyset
\end{align*}
を示せ.
}{
\item $(A_1, A_2, \cdots, A_n)$ は 写像 $A$ によって各 $n \in \bm{N}$ を写した先 $A(1), A(2), \cdots, A(n)$ を表す.
ここで, 写像の相等条件を考えれば, $(A_1, A_2, \cdots, A_n) = (B_1, B_2, \cdots, B_n) \implies A_1 = B_1 \land A_2 = B_2 \land \cdots \land A_n = B_n$
となり $n$-対の性質を持つ.\footnote{これにより, $\prod_{n \in \bm{N}}A_n$ は直積集合 $A_1 \times A_2 \times \cdots \times A_n$ と同一視される.}
\item $R_x = \{y \in A\ |\ \langle x, y\rangle \in R \}$ とすれば, DCの仮定より $R_x \neq \emptyset$ となるから, ACから $\prod_{x \in A}R_x \neq \emptyset$, 成り立つ.
すなわち, $\exists g: A \to A,\ (\forall x \in A)[\langle x, g(x)\rangle \in R_x]$ が成り立つ. そこで, $A$ の任意の元を $x_0$ とし, $x_n = g(x_{n-1})$ と
帰納的に $(x_n)_{n \in \bm{N}}$ を作れば, $\forall n \in \bm{N},\ \langle x_n, x_{n+1} \rangle \in R$ となる. したがって, $f(n) = x_n$ となるように $f$ を
定めれば, $\langle f(n), f(n+1) \rangle \in R$ となる.
\item $P = \biggl\{\ p\ \bigm|\ (\exists\ n \in \bm{N})\ \left[p : \{0, 1, \cdots, n\} \to \bigcup_{i = 0}^n A_n\ \land \ (\forall i \in \{0, 1, \cdots, n\})\ [p(i) \in A_i]\right]\ \biggr\}$
を考える. $p$ は写像であるが, 写像は二項関係の特別な場合(すなわち $p \subset \{0, 1, \cdots, n\} \times \bigcup_{i = 0}^n A_n$ )であることに注意して
$R = \left\{\ \langle p, q\rangle \in P \times P\ \middle|\ p \subsetneq q\ \right\}$ と二項関係 $R$ を定めると, $pRq$ ならば $q$ は写像として $p$ の真の拡大となる.
今, $\forall n \in \bm{N},\ A_n \neq \emptyset$ より, $\forall p \in P,\ \exists q \in P,\ \langle p, q \rangle \in R$ が成り立つ($A_{n + 1}$ から元を 1つ選んで拡大すればよい)
よって, DCより $f : \bm{N} \to P$ で $\forall n \in \bm{N},\ \langle f(n), f(n+1) \rangle \in R$ となるものが存在する. ここで $f(n) \subset \bm{N} \times P$ に
であることに注意して, $a = \bigcup_{n \in \bm{N}}f(n)$ とすれば, 任意の $n$ に対して, $\text{dom}(f(n)) \subsetneq \text{dom}(f(n+1))$ より
$\forall n \in \bm{N},\ n \in \text{dom}(f(n))$ となるから $\bm{N} \subset \bigcup_{n \in \bm{N}}\text{dom}(f(n))$ となる. また, $\forall n \in \bm{N},\ \text{dom}(f(n)) \subset \bm{N}$
より, $\bigcup_{n \in \bm{N}}\text{dom}(f(n)) \subset \bm{N}$ となるから, $\bigcup_{n \in \bm{N}}\text{dom}(f(n)) = \bm{N}$ となる. ここで
$\text{dom}(a) = \bigcup_{n \in \bm{N}}\text{dom}(f(n))$ より, $\text{dom}(a) = \bm{N}$ となる.
さらに, 集合 $P$ の定義より $\forall n \in \bm{N},\ a(n) \in A_n$ となる. 以上より $a \in \prod_{n \in \bm{N}}A_n$ となり, $\prod_{n \in \bm{N}}A_n \neq \emptyset$ となる. 
}
\end{nmprob}



\begin{nmprob}
選択公理の簡単な応用例.
\pbenumex[ac]{
    $f$ を $A$ から $B$ への写像とするとき, 次を示せ. ただし, $I_X$ は $X$ から $X$ への恒等写像を表すものとする.
}{
\item $f$ が全射 $\iff$ $f\circ g = I_B$ となるような写像 $g: B \to A$ が存在する
\item $f$ が単射 $\iff$ $h\circ f = I_A$ となるような写像 $h: A \to B$ が存在する
\item $A$ から $B$ への単射が存在する $\iff$ $B$ から $A$ への全射が存在する
}{
\item
\begin{enumerate}
    \item $\implies$\\
    $f$ は全射より, $\forall b \in B,\ f^{-1}\{b\} \neq \emptyset$ となる. よって, 選択公理より $g \in \prod_{b \in B}f^{-1}\{b\}$ となる $g : B \to A$ が存在する.
    この $g$ は 任意の $b \in B$ に対して, $g(b) \in f^{-1}\{b\}$ となるから, $\forall b \in B,\ f(g(b)) = b$ となるから $f \circ g = I_B$ を満たす.
    \item $\impliedby$\\
    $f \circ g = I_B$ となる $g : B \to A$ の存在を仮定すると, $\forall b \in B,\ f(g(b)) = b$ となる. $g(b) \in A$ より, $f$ は全射となる. 
\end{enumerate}
以上より, 「 $f$ が全射 $\iff$ $f\circ g = I_B$ となるような写像 $g: B \to A$ が存在する」が成り立つ.
\item
\begin{enumerate}
    \item $\implies$\\
    $f$ の終域を $B$ から $\text{rng}(f)$ へ縮小すると $f$ は全単射となる. このとき, 逆写像 $f^{-1} : \text{rng}(f) \to A$ が存在し, 今 $a \in A$ を適当にとり, $h : B \to A$ を
    \begin{align*}
        h(b) =
        \begin{cases}
            a & (y \in B - \text{rng}(f)\text{のとき})\\
            f^{-1}(b) & (y \in \text{rng}(f)\text{のとき})
        \end{cases}
    \end{align*}
    のように定めれば, $h \circ f = I_A$ を満たす.
    \item $\impliedby$\\
    $f(a) = f(a^{'}) \implies a = h(f(a)) = h(f(a^{'})) = a^{'}$ より $f$ は単射となる.
\end{enumerate}
以上より, 「 $f$ が単射 $\iff$ $h\circ f = I_A$ となるような写像 $h: A \to B$ が存在する 」が成り立つ.
\item
\begin{enumerate}
    \item $\implies$\\
    $A$ から $B$ への単射を $\phi$ とすれば(2)より, $\psi \circ \phi = I_A$ となる写像 $\psi : A \to B$ が存在する.
    この $\psi$ は(1)より全射である((1)において $B$ と $A$ を逆にみればよい)
    \item $\impliedby$\\
    $B$ から $A$ への全射を $\psi$ とすれば(1)より, $\psi \circ \phi = I_A$ となる写像 $\phi : A \to B$ が存在する
    ((1)において $B$ と $A$ を逆にみればよい)ここで, この $\phi$ は(2)より単射である.
\end{enumerate}
以上より, 「$A$ から $B$ への単射が存在する $\iff$ $B$ から $A$ への全射が存在する」が成り立つ.
}
\end{nmprob}



\begin{nmprob}
問題というよりかは確認.
\makeatletter\tagsleft@true\makeatother
\pbenum{
\item $R$ を集合 $A$ 上の同値関係とする. $A$ の元 $x$ に対して, 集合 $\{ y\in A\ |\ xRy \}$ を $x$ の $R$ による同値類といい, 以後 $[x]_R$ と表す.
また, 同値類全体の集合を $A$ の $R$ による商集合といい $A/R$ と表す. ここで, $A/R$ によって定まる集合族が $A$ の直和分割となることを示せ. ただし,
ある集合 $X$ の部分集合族 $(X_i)_{i\in I}$ が次の三条件を満たすとき, $(X_i)_{i\in I}$ は $X$ の直和分割と呼ばれる. 
\begin{align}
    &\forall i\in I,\ X_i \neq \emptyset\\
    &\bigcup_{i\in I} X_i = X\\
    &i, j \in I \land i \neq j \implies X_i \cap X_j = \emptyset
\end{align}
\setcounter{equation}{0}
\item 空でない集合 $A$ のある集合族 $(C_i)_{i\in I}$ が $A$ の直和分割であるとする. 任意の $A$ の元 $x$ に対し, $x \in C_i$ となる $i \in I$ がただ一つ存在することを示せ.
\item (2)において, 関係 $R$ を $xRy \iff \exists i \in I,\ x \in C_i \land y \in C_i$ により定めると, $R$ は $A$ 上の同値関係となることを示せ.
}{
\item $A/R$ によって定まる族を $(X)_{X\in A/R}$ で表すことにすると
\begin{enumerate}
    \item 任意の $X$ を一つとり, その代表元\footnote[1]{$X \in A/R$ に対する $X$ の元}を $x$ とすると, $x \in X$ より $X \neq \emptyset$ となる
    ($A/R \coloneqq \{ [x]_R\ |\ x \in A\}$ としているので, 代表元の存在が保証される)
    \item 任意の $x \in A$ に対し, $x \in [x]_R \in A/R$ より, $x \in X$ となる $X \in A/R$ が存在することから $\bigcup_{X\in A/R} X = A$ となる.
    \item 任意の $X, Y \in A/R$ に対して, $X \neq Y$ のとき, $X \cap Y \neq \emptyset$ と仮定する. $c \in X \cap Y$ とすれば
    $xRc$ かつ $yRc$ となる($x, y$ はそれぞれ $X, Y$ の代表元とする) ここで, $R$ は同値関係より $xRy$ となり $y \in [x]_R\ (= X)$ となる. 
    $y \in [x]_R$ であるとき, $[x]_R = [y]_R$ であるので, $X = Y$ となるが, これは $X \neq Y$ に矛盾. よって $X \cap Y = \emptyset$ となる.
\end{enumerate}
以上より, $A/R$ によって定まる集合族は $A$ の直和分割となる.
\item まず, 直和分割の条件(2) より, $\bigcup_{i \in I}C_i = A$ より, 任意の $A$ の元 $x$ に対して $x \in C_i$ となる $i \in I$ が存在する.\\
また, $x \in C_i$ かつ $x \in C_j$ となる $i, j\ (i \neq j)$ が存在すると仮定すると, $x \in C_i \cap C_j$ より, 直和分割の条件(3)に矛盾することから, 
$x \in C_i$ かつ $x \in C_j$ となる $i, j\ (i \neq j)$ は存在しない.\\
以上より, 任意の $A$ の元 $x$ に対し, $x \in C_i$ となる $i \in I$ がただ一つ存在する.
\item \
\begin{enumerate}
    \item (2) より 任意の $x \in A$ に対して, $x \in C_i$ となる $i$ が存在することから, $\forall x \in A,\ xRx$ で反射律が成立.
    \item 対称律は明らかに成立.
    \item $xRy \land yRz$ が成り立つとすると, $(\exists i \in I,\ x \in C_i \land y \in C_i) \land (\exists j \in I,\ y \in C_j \land z \in C_j)$ が成り立つ.
    ここで $i \neq j$ とすると, $y \in C_i \cap C_j$ より, $(C_i)_{i\in I}$ が直和分割であることに反するから, $i = j$ となる. これより, $\exists k \in I,\ x \in C_k \land z \in C_k$ が
    成り立つので $xRz$ で推移律が成立.
\end{enumerate}
以上より, $R$ は $A$ 上の同値関係となる.
}
\makeatletter\tagsleft@false\makeatother
\end{nmprob}



\begin{nmprob}
Bernstein の定理を理解
\pbenumex{
集合 $A$ から $B$ への全単射($B$ から $A$ への全単射でもある)が存在するとき $A$ と $B$ は対等であるといい, $A \sim B$ で表す.
このとき, 次の問に答えよ.
}{
\item $A$ から $B$ への単射が存在し, かつ $B$ から $A$ への単射が存在すれば $A \sim B$ となることを示せ.
\item $A$ から $B$ への全射が存在し, かつ $B$ から $A$ への全射が存在すれば $A \sim B$ となることを示せ.
\item $A \sim B^{'}$ となるような $B^{'} \subset B$ が存在し, かつ $B \sim A^{'}$ となるような $A^{'} \subset A$ が存在すれば $A \sim B$ となることを示せ. 
}{
\item $f$ を $A$ から $B$ への単射, $g$ を $B$ から $A$ への単射とする.\\
このとき, $B_0 = B - f[A]$ とし, $A_n = g[B_{n-1}], B_n = f[A_n]$ として,
$A$ の部分集合族 $(A_n)_{n \in \{1,2,3,\ldots \}}$, $B$ の部分集合族 $(B_n)_{n \in \{0,1,2,\ldots \}}$ を定める. また, $A - \bigcup_{n=1}^{\infty}A_n = A^*, B - \bigcup_{n=0}^{\infty}B_n = B^*$ とする.\\
ここで,
\begin{align*}
    f[A^*] &= f[A] - f\left[\bigcup_{n=1}^{\infty}A_n\right]\\
    &= (B - B_0) -f\left[\bigcup_{n=1}^{\infty}A_n\right]\hspace{1zw}\text{($\because$\ $f$ が単射より)}\\
    &= B - \left(B_0 \cup f\left[\bigcup_{n=1}^{\infty}A_n\right]\right)\\
    &= B - \left(B_0 \cup \bigcup_{n=1}^{\infty} f[A_n]\right)\\
    &= B - \left(B_0 \cup \bigcup_{n=1}^{\infty} B_n\right)\hspace{1zw}\text{($\because$\ $f[A_n] = B_n$ より)}\\
    &= B - \bigcup_{n=0}^{\infty}B_n\\
    &= B^*
\end{align*}
また,
\begin{align*}
    g\left[\bigcup_{n=0}^{\infty}B_n\right] &= \bigcup_{n=1}^{\infty}g[B_{n-1}]\\
    &= \bigcup_{n=1}^{\infty}A_n\hspace{1zw}\text{($\because$\ $g[B_{n-1}] = A_n$ より)}\\
\end{align*}
となる. よって $f$ の定義域を $A^*$, 終集合を $B^*$ に変えた写像と, $g$ の定義域を $\bigcup_{n=0}^{\infty}B_n$, 終集合を $\bigcup_{n=1}^{\infty}A_n$ に変えた写像は, それぞれ全単射となる.
そこで, $A$ から $B$ への写像 $F$ を $a \in A^*$ のとき $F(a) = f(a)$, $a \in \bigcup_{n=1}^{\infty}A_n$ のときは $F(a) = g^{-1}(a)$ とすれば $F$ は全単射となり, $A \sim B$ となる.
\item 問\ref{sec:set}-\ref{pname:ac} (3)よりすぐに示せる.
\item $A$ から $B^{'}$ への全単射の写像の終集合を $B$ に拡大すれば, その写像は単射となる. 同様に $B$ から $A^{'}$ への全単射も $B$ から $A$ への単射にすることが可能である.
よって(1) とから $A \sim B$ となる.
}
\newpage
\hspace{-3zw}{\color{forestgreen}●●\ Bernstein の定理の証明の補足\ ●●}

Bernstein の定理の証明は次の図をイメージすると良い.
\begin{figure}[htbp]
    \centering
    \begin{tikzpicture}
        \coordinate (O);
        \coordinate (A) at ($(O) + (3, 4)$);
        \coordinate (LA) at ($(O) + (1, 3)$);
        \draw[rounded corners] (O) rectangle($(O) + (6, 4)$);
        \path [pattern=north east lines, pattern color=magenta, opacity=.3] (O) rectangle ($(O) + (6, 4)$);
        \draw (A) node [above] {$A$};
        \draw ($(O) + (0, 2)$) -- ++(2, 0) -- ++(0, 2);
        \draw ($(O) + (2, 2)$) -- ++(2, 0) -- ++(0, 2);
        \draw ($(O) + (2, 2)$) -- ++(0, -2);
        \filldraw[fill=green!70!red, fill opacity=.5] ($(O) + (4, 2)$) -- ++(0, -2) [rounded corners] -- ++(2, 0) [sharp corners] -- ++(0, 2) -- ++(-2, 0) --cycle;
        \draw (LA) node[above] (A1) {$g[B_0] = A_1$};
        \draw ($(LA) + (2, 0)$) node[above] (A2) {$g[B_1] = A_2$};
        \draw ($(LA) + (4, 0)$) node[above] (A3) {$g[B_2] = A_3$};
        \draw ($(LA) + (0, -2)$) node[above] (A4) {$g[B_3] = A_4$};
        \draw ($(LA) + (2, -2)$) node[above] {$\ldots$};
        \draw ($(LA) + (4, -2)$) node[above] {$A^{*}$};

        \coordinate (O2) at ($(O) + (8, 0)$);
        \coordinate (B) at ($(O2) + (3, 4)$);
        \coordinate (LB) at ($(O2) + (1, 3)$);

        \draw[rounded corners] (O2) rectangle($(O2) + (6, 4)$);
        \path [pattern=north east lines, pattern color=magenta, opacity=.3] (O2) -- ++(0, 2) -- ++(2, 0) -- ++(0, 2) -- ++(4, 0) -- ++(0, -4) -- ++(-6, 0) --cycle;
        \draw (B) node [above] {$B$};
        \draw ($(O2) + (0, 2)$) -- ++(2, 0) -- ++(0, 2);
        \draw ($(O2) + (2, 2)$) -- ++(2, 0) -- ++(0, 2);
        \draw ($(O2) + (2, 2)$) -- ++(0, -2);
        \filldraw[fill=green!70!red, fill opacity=.5] ($(O2) + (4, 2)$) -- ++(0, -2) [rounded corners] -- ++(2, 0) [sharp corners] -- ++(0, 2) -- ++(-2, 0) --cycle;
        \node[above=-4pt of LB, align=left] (B0) {$B - f[A]$\\$ = B_0$};
        \draw ($(LB) + (2, 0)$) node[above] (B1) {$f[A_1] = B_1$};
        \draw ($(LB) + (4, 0)$) node[above] (B2) {$f[A_2] = B_2$};
        \draw ($(LB) + (0, -2)$) node[above] (B3) {$f[A_3] = B_3$};
        \draw ($(LB) + (2, -2)$) node[above] {$\ldots$};
        \draw ($(LB) + (4, -2)$) node[above] {$B^{*}$};

        \draw [bend right,distance=50,blue,->] (A1) to node [above] {$g^{-1}$} (B0);
        \draw [bend left,distance=50,blue,->] (A2) to node [above] {$g^{-1}$} (B1);
        \draw [bend right,distance=50,blue,->] (A3) to node [above] {$g^{-1}$} (B2);
        \draw [bend right,distance=70,blue,->] (A4) to node [below] {$g^{-1}$} (B3);
        \draw [bend right,distance=70,red,->] ($(LA) + (4, -2)$) to node [below] {$f$} ($(LB) + (4, -2)$);

    \end{tikzpicture}
    \caption{Bernstein の定理のイメージ}
\end{figure}

上図において $g^{-1}, f$ をまとめた写像が全単射だと証明している. なお, マゼンダ色の領域は $A \to f[A]$ の対応を表す. また, 図中における各集合の共通部分が無いのは, $f, g$ の単射性による.
\end{nmprob}
\setcounter{figure}{0}

 % 集合
\pgsc{代数学}{chocolate}{algebra}

\begin{nmprob}
ペアノシステムの数学的帰納法.
\pbenumex[peano]{
次の条件 (N1) から (N3) を満たす, 集合 $\bm{N}$, その一つの元 $0$, および写像 $\sigma :\bm{N} \to \bm{N}$ の対 $\langle \bm{N}, 0, \sigma \rangle$
をペアノシステムという.
\begin{description}\setlength{\leftskip}{16pt}
    \item[\rm (N1)] $\sigma :\bm{N} \to \bm{N}$ は単射である.
    \item[\rm (N2)] $0 \notin \sigma(\bm{N})$
    \item[\rm (N3)] $S \subset \bm{N}$ のとき, $S$ が次の二条件を満たせば $S = \bm{N}$ となる.
    \begin{align}
        0 \in S\\
        \sigma(S) \subset S
    \end{align}
    \setcounter{equation}{0}
\end{description}
特に, (N3) を数学的帰納法の公理とよぶ. また, $\sigma$ は後継者写像といい, $n$ に対して $\sigma(n)$ はその後継者とよばれる.
}{
\item 任意の $n \in \bm{N}$ に対して, $n \in S \implies \sigma(n) \in S$ が成り立つとすると, $\sigma(S) \subset S$ が成り立つことを示せ. ただし, $S \subset \bm{N}$ とする.
\item 任意の $n \in \bm{N}$ に対して, $\sigma(n) \neq n$ を示せ.
\item 任意の $n \in \bm{N}\backslash \{0\}$ に対して, $\exists m\in \bm{N},\ \sigma (m) = n$ を示せ. 
}{
\item $y \in \sigma (S)$ とすると, $\sigma (S)$ の定義から, $\langle x, y\rangle \in \sigma$ となる $x \in S$ が存在する.
今, $\forall n \in \bm{N},\ n \in S \implies \sigma (n) \in S$ が成り立つので, $\sigma (x) \in S$ となる. また, $\sigma$ は写像より $y = \sigma (x)$ となるから, $y \in S$ となる.
よって, $\sigma (S) \subset S$ となる.
\item $S = \{ n \in \bm{N}\ |\  \sigma (n) \neq n \}$ とする. まず, (N2) より $0 \in S$ となる.
次に, 任意の $n \in \bm{N}$ を一つとると, $n \in S$ ならば $\sigma (n) \neq n$ より, (N1) から $\sigma (\sigma (n)) \neq \sigma(n)$ となり, $\sigma (n) \in S$ となる.
よって, (1) より, $\sigma (S) \subset S$ となるから, (N3) より $S = \bm{N}$ となる. これは, $\forall n \in \bm{N},\ \sigma (n) \neq n$ を意味する.
\footnote[1]{$P(0)$が真, かつ$P(n)$が真ならば$P(\sigma (n))$ が真であれば $S = \{ n\in \bm{N}\ |\ P(n)\}$ とすることで N3 から $S = \bm{N}$ を示す方法はよく使われる.\vspace{30pt}}
\item $S = \{0\} \cup \sigma (\bm{N})$ とすると, 明らかに $0 \in S$ となる. また, 任意の $n \in \bm{N}$ に対して, $n \in S$ とすると, $\sigma (n) \in \sigma (\bm{N}) \subset S$ となる.
よって, (N3) より, $S = \bm{N}$ となる. ゆえに $\bm{N}$ の $0$ 以外の元は $\sigma (\bm{N})$ に含まれる.
以上より, 任意の $n \in \bm{N}\backslash \{0\}$ に対して, $\exists m\in \bm{N},\ \sigma (m) = n$ となる. ちなみに, (N2) とから, $\bm{N}\backslash \{0\} = \sigma (\bm{N})$ が成り立つ(外延性から示せる)
} 
\end{nmprob}



\begin{nmprob}
ペアノシステムの一意性.
\renewcommand{\labelenumii}{(\Roman{enumii})}
\pbenum{
\item $X$ を一つの集合とし, $X$ の一つの元 $x_0$ と写像 $\phi :X \to X$ とが与えられたとする.
このとき, 次の二条件を満たすような写像 $f :\bm{N} \to X$ がただ一つ存在することを示せ.

\renewcommand{\theequation}{\roman{equation}}
\begin{align}
& f(0) = x_0\\
& \forall n \in \bm{N},\ f(\sigma (n)) = \phi (f(n))
\end{align}
\setcounter{equation}{0}
\item $\langle \bm{N}, 0, \sigma \rangle$ と $\langle \bm{N^{'}}, 0^{'}, \sigma^{'} \rangle$ が共にペアノシステムであるとき,
$\bm{N}$ から $\bm{N^{'}}$ への全単射 $f$ で, $f(0) = 0^{'}$ かつ, 任意の $n \in \bm{N}$ に対して $f(\sigma (n)) = \sigma^{'}(f(n))$ となるもの
が一意的に存在することを示せ.
}{
\item
\begin{enumerate}
\item
まず, $f$ の一意性から示す.\\
相違な写像 $f: \bm{N} \to X,\ f^{'}: \bm{N} \to X$ が共に条件 (\rnum{1}), (\rnum{2}) を満たすとする.
ここで, $S = \{ n \in \bm{N}\ |\ f(n) = f^{'}(n)\}$ とすると, (\rnum{1}) より $0 \in S$ である.
また, 任意の $n \in \bm{N}$ に対して, $n \in S$ とすると, (\rnum{2}) より $f(\sigma (n)) = \phi (f(n)) = \phi (f^{'}(n)) = f^{'}(\sigma (n))$ となるから, $\sigma (n) \in S$ となるから
問\ref{sec:algebra}-\ref{pname:peano}\ (1) より, $S \subset \sigma (S)$ となる. よって, 数学的帰納法の公理より, $S = \bm{N}$ となる. よって, $f = f^{'}$ となり, 矛盾. よって, $f$ が存在するとすれば一意である.
\item
次に, $f$ の存在を示す.\\
まず, 次の二条件を満たす $\bm{N} \times X$ の部分集合 $R$ を考える.
\begin{align}
    \langle & 0, x_0\rangle \in R\\
    \forall & (n\in \bm{N})[\langle n, x\rangle \in R \implies \langle \sigma (n), \phi (x) \rangle \in R]
\end{align}
\setcounter{equation}{0}
条件(1), (2) を満たす集合全体を $\mathcal{F}$ とすると, $\bm{N} \times X \in \mathcal{F}$ より $\mathcal{F} \neq \emptyset$ である(条件 (2) は $\sigma (n) \in \bm{N},\ \phi (x) \in X$ より満たされる)
よって, $\bigcap_{R \in \mathcal{F}} R$ を考えることができる.\\
ここで, 
\begin{align*}
f = \bigcap_{R \in \mathcal{F}} R    
\end{align*}
としたとき, この $f$ が $\bm{N}$ から $X$ への写像であり, 二条件 (\rnum{1}), (\rnum{2}) を満たすことを示す.\\
まず, $f$ が $\bm{N}$ から $X$ への写像であることを示すために, $T = \{ n \in \bm{N}\ |\ \langle n, x\rangle \in f \text{なる} x \text{が唯一つ存在する}\}$ を考える.
\begin{enumerate}
\item $0 \in T$ を示す.\\
$f$ についてすぐにわかる通り, $f \in \mathcal{F}$ であるから, $0 \in T$ となる. 
\item $\forall n \in \bm{N},\ n \in T \implies \sigma (n) \in T$ を示す.\\
$n \in T$ とすると, $\langle \sigma (n), x_n \rangle \in f$ となる $x_n \in X$ がただ一つ存在する. まず, $f \in \mathcal{F}$ より, 条件 (2) から $\langle \sigma (n), \phi (x_n) \rangle \in f$ より,
$\exists x \in X,\ \langle n, x\rangle \in f$ となる. よって, この $x$ の一意性(つまり $x = \phi (x_n)$ のみである)を示せば, $\sigma (n) \in T$ となる.
一意性を得るために $\langle \sigma (n), y \rangle \in f$ となる $y \in X\backslash \{\phi (x_n) \}$ が存在すると仮定する.
ここで, $g = f\backslash \{ \langle \sigma (n), y \rangle\}$ として, $g \in \mathcal{F}$ を示す.
\begin{itemize}
\item 条件 (1)\\
ペアノシステムの公理 N2 より, $0 \notin \bm{N}$ であるから, $\sigma (n) \neq 0$ より, $\langle 0, x_0 \rangle \in g$ である.
\item 条件 (2)\\
$\langle n_g, x_g\rangle \in g$ を任意にとったとき, $\langle n_g, x_g\rangle \in f$ であり, $f \in \mathcal{F}$ とから, $f$ が条件 (2) を満たすことから, $\langle \sigma (n_g), \phi (x_g)\rangle \in f$ となる.
$n_g \neq n$ の場合はペアノシステムの公理 N1 より, $\langle \sigma (n_g), x_g\rangle \neq \langle \sigma (n), y\rangle$ である(第一成分が異なることより)から, $\langle \sigma (n_g), x_g\rangle \in g$ となる.
また, $n_g = n$ の場合は, 今 $n \in T$ で $\langle n, x_n \rangle \in f$ となる $x_n \in X$ はただ一つ存在することから, $x_g = x_n$ でなくてはならない. よって, $\langle \sigma (n_g), \phi (x_g) = \phi (x_n) \rangle \in f$
であり, $y \neq \phi (x_n)$ より, $\langle \sigma (n_g), x_g\rangle \neq \langle \sigma (n), y\rangle$ となるから $\langle \sigma (n_g), x_g\rangle \in g$ となる.
以上より, $\langle n_g, x_g\rangle \in g \implies \langle \sigma (n_g), x_g\rangle \in g$ となる.
\end{itemize}
以上より, $g \in \mathcal{F}$ である. さて, 今 $g \subsetneq f$ である. ここで, $f = \bigcap_{R \in \mathcal{F}} R$ より, $\forall R \in \mathcal{F},\ f \subset R$ より,
$g \in \mathcal{F}$ とから $g \subset f$ となるが, これは $g \subsetneq f$ に矛盾する. よって, $\langle \sigma (n), y \rangle \in f$ となる $y \in X\backslash \{\phi (x_n) \}$ は存在しない,
すなわち $\langle \sigma (n), x_n \rangle \in f$ となる $x_n \in X$ がただ一つ存在する. 以上より, $\sigma (n) \in T$ となり, $\forall n \in \bm{N},\ n \in T \implies \sigma (n) \in T$ が成り立つ.
\end{enumerate}
上の (a), (b) より, 問\ref{sec:algebra}-\ref{pname:peano}\ (1)より $T = \bm{N}$ となる. よって, $f$ は $\bm{N}$ から $X$ への写像となる. また, $0 \in T$ を示す際に述べた通り, $f \in \mathcal{F}$ より, $f$ は二条件 (1), (2) を満たす.
これより, すぐにわかる通り $f$ が満たすべき二条件 (\rnum{1}), (\rnum{2}) も満たされる. 以上より, $f$ の存在性が示された.
\end{enumerate}
\item
(1) において, $X = \bm{N}^{'},\ \phi = \sigma,\ x_0 = 0^{'}$ とすればよい.
}

\hspace{-3zw}{\color{orangered}●●\ 前問題の補足\ ●●}

前問題における各写像の様子を次図に示す.
\begin{figure}[htbp]
    \centering
    \begin{tikzpicture}
        \coordinate (O);       
        \node (A0) at ($(O) + (0, 1)$) {$0$};
        \node (B0) at ($(O) + (0, -1)$) {$x_0$};
        \node (A1) at ($(A0) + (2, 0)$) {$\sigma (0)$};
        \node (B1) at ($(B0) + (2, 0)$) {$\phi (x_0)$};
        \node (A2) at ($(A1) + (2, 0)$) {$\ldots$};
        \node (B2) at ($(B1) + (2, 0)$) {$\ldots$};
        \node (A3) at ($(A2) + (2, 0)$) {$n$};
        \node (B3) at ($(B2) + (2, 0)$) {$f(n)$};
        \node (A4) at ($(A3) + (2, 0)$) {$\sigma (n)$};
        \node (B4) at ($(B3) + (2, 0)$) {$\phi (f(n))$};

        \draw [|->] (A0) to node [above] {$\sigma$} (A1);
        \draw [|->] (A1) to node [above] {$\sigma$} (A2);
        \draw [|->] (A2) to node [above] {$\sigma$} (A3);
        \draw [|->] (A3) to node [above] {$\sigma$} (A4);
        \draw [|->] (B0) to node [below] {$\phi$} (B1);
        \draw [|->] (B1) to node [below] {$\phi$} (B2);
        \draw [|->] (B2) to node [below] {$\phi$} (B3);
        \draw [|->] (B3) to node [below] {$\phi$} (B4);
        \draw [bend right,distance=10,->] (A0) to node [left] {$f$} (B0);
        \draw [bend right,distance=10,->] (A1) to node [left] {$f$} (B1);
        \draw [bend right,distance=10,->] (A3) to node [left] {$f$} (B3);
        \draw [bend right,distance=10,->] (A4) to node [left] {$f$} (B4);
    \end{tikzpicture}
    \caption{前問題のイメージ}
\end{figure}

また, 全単射の存在だけでなく一意性も示したのは以下のような要素の順番の入れ替えを防ぐためである.
\begin{figure}[htbp]
    \begin{minipage}[b]{0.45\linewidth}
    \centering
    \begin{tikzpicture}
        \coordinate (O);
        \coordinate (A1) at ($(O) + (-1, 1)$);
        \coordinate (A2) at ($(O) + (-1, 0)$);
        \coordinate (A3) at ($(O) + (-1, -1)$);
        \coordinate (B1) at ($(O) + (1, 1)$);
        \coordinate (B2) at ($(O) + (1, 0)$);
        \coordinate (B3) at ($(O) + (1, -1)$);
        \fill (A1) circle [radius=0.06];
        \fill (A2) circle [radius=0.06];
        \fill (A3) circle [radius=0.06];
        \fill (B1) circle [radius=0.06];
        \fill (B2) circle [radius=0.06];
        \fill (B3) circle [radius=0.06];
        \draw (A1) node [above] {$0$};
        \draw (A2) node [above] {$1$};
        \draw (A3) node [above] {$2$};
        \draw (B1) node [above] {$0^{'}$};
        \draw (B2) node [above] {$1^{'}$};
        \draw (B3) node [above] {$2^{'}$};
        \draw (A2) circle [x radius=0.8, y radius=1.6];
        \draw (B2) circle [x radius=0.8, y radius=1.6];
        \draw [->] (A1) to (B1);
        \draw [->] (A2) to (B2);
        \draw [->] (A3) to (B3);
    \end{tikzpicture}
    \caption{全単射の例 1}
    \end{minipage}
    \begin{minipage}[b]{0.45\linewidth}
    \centering
    \begin{tikzpicture}
        \coordinate (O);
        \coordinate (A1) at ($(O) + (-1, 1)$);
        \coordinate (A2) at ($(O) + (-1, 0)$);
        \coordinate (A3) at ($(O) + (-1, -1)$);
        \coordinate (B1) at ($(O) + (1, 1)$);
        \coordinate (B2) at ($(O) + (1, 0)$);
        \coordinate (B3) at ($(O) + (1, -1)$);
        \fill (A1) circle [radius=0.06];
        \fill (A2) circle [radius=0.06];
        \fill (A3) circle [radius=0.06];
        \fill (B1) circle [radius=0.06];
        \fill (B2) circle [radius=0.06];
        \fill (B3) circle [radius=0.06];
        \draw (A1) node [above] {$0$};
        \draw (A2) node [above] {$1$};
        \draw (A3) node [above] {$2$};
        \draw (B1) node [above] {$0^{'}$};
        \draw (B2) node [above] {$1^{'}$};
        \draw (B3) node [above] {$2^{'}$};
        \draw (A2) circle [x radius=0.8, y radius=1.6];
        \draw (B2) circle [x radius=0.8, y radius=1.6];
        \draw [->] (A1) to (B2);
        \draw [->] (A2) to (B3);
        \draw [->] (A3) to (B1);
    \end{tikzpicture}
    \caption{全単射の例 2}
    \end{minipage}
\end{figure}
\setcounter{figure}{0}
\end{nmprob}



\begin{nmprob}
整列性から一般的な数学的帰納法.
\pbenumex{
    以降の問題では, 正の整数全体の集合を $\mathbb{Z}^+$ で表す. このとき, 次の問に答えよ.
}{
\item 次の条件を満たす $\mathbb{Z}^+$ の部分集合 $S$ を考える.
\begin{align}
    &1 \in S\\
    &\forall n \in \mathbb{Z}^+ [n \in S \implies n + 1 \in s]
\end{align}
整列性(任意の空でない自然数の集合は最小限を持つ)を認めた上で, $S = \mathbb{Z}^+$ を満たすことを示せ.
\item $\mathbb{Z}^+$ の元の各々に対し, 命題 $P(n)$ が与えられたとし, それについて次の二つのことが示されたとする.
\setcounter{equation}{0}
\begin{align}
    &P(1) \text{は真}\\
    &\forall n \in \mathbb{Z}^+ [P(n) \text{が真} \implies P(n+1) \text{も真}]
\end{align}
\setcounter{equation}{0}
このとき, 全ての $n \in \mathbb{Z}^+$ に対して, $P(n)$ が真となることを示せ.
}{
\item $\mathbb{Z}^+ - S = S^{'}$ とし, $S^{'} = \emptyset$ となることを背理法により示す.\\
$S^{'} \neq \emptyset$ と仮定すると, 整列性より $n_0 = \min S^{'}$ となる $n_0 \in \mathbb{Z}^+$ が存在する.
今, $n_0 > 1$ より, $n_0 - 1 \geq 1$ となり $n_0 - 1 \in \mathbb{Z}^+$ である.
ここで $n_0 - 1 < n_0 = \min \mathbb{Z}^+$ より $n_0 - 1 \notin S^{'}$ , すなわち $n_0 - 1 \in S$ となる.\\
これより, $S$ が満たす条件(2)より, $n_0 \in S$ となり, $n_0 \in S^{'}$ に矛盾.\\
以上より, $S^{'} = \emptyset$ であり, $S = \mathbb{Z}^+$ が成り立つ.
\item $S =  \{\ n \in \mathbb{Z}^+\ |\ P(n) \text{が真} \}$ と集合 $S$ を定義すると, (1)より $S = \mathbb{Z}^+$ が成り立つ.
よって, 全ての $n \in \mathbb{Z}^+$ に対して, $P(n)$ が真となる
}
\end{nmprob}

 % 代数学
\pgsc{位相}{mediumvioletred}{topology}

\begin{nmprob}
まずは数学の道具(概念)に慣れるための基礎問題.\\
\pbenum{
\item $n$次元実空間における任意の2点 $\bm{x}, \bm{y} \in \mathbb{R}^n$に対し, シュワルツの不等式
\begin{equation*}
    |(\bm{x}\ |\ \bm{y})| \leq \|\bm{x}\|\cdot \|\bm{y}\|
\end{equation*}
を示せ.
\item $n$次元実空間$\mathbb{R}^n$における2点$\bm{x} = (x_1, x_2, \cdots, x_n)$と$\bm{y} = (y_1, y_2, \cdots, y_n)$の距離$d(\bm{x}, \bm{y})$を次のように定義する.
\begin{equation*}
    d(\bm{x}, \bm{y}) = \sqrt{\sum_{i=1}^{n}(x_i - y_i)^2}
\end{equation*}
このとき, 三角不等式 $d(\bm{x}, \bm{z}) \leq d(\bm{x}, \bm{y}) + d(\bm{y}, \bm{z})$ を示せ.
}{
\item いくつかやり方があるが, ここでは天下りだが代数的に済む解法を示す.
\begin{enumerate}
    \item $\bm{y} = \bm{0}$ のとき\\
    \ \\
    両辺 0 で成立.
    \item $\bm{y} \neq \bm{0}$ のとき\\
    \ \\
    任意の実数 $a, b$ に対して, 
    \begin{eqnarray*}
        0 &\leq& \|a\bm{x}+b\bm{y}\|^2\\
        &=& \|a\|^2\|\bm{x}\|^2 + 2ab\cdot (\bm{x}\ |\ \bm{y}) + \|b\|^2\|\bm{y}\|^2
    \end{eqnarray*}
    ここで, $a = \|\bm{y}\|^2,\ b = -(\bm{x}\ |\ \bm{y})$ とすると,
    \begin{eqnarray*}
        0 &\leq& \|\bm{y}\|^4\|\bm{x}\|^2-2\|\bm{y}\|^2|(\bm{x}\ |\ \bm{y})|^2 + |\bm{y}\|^2|(\bm{x}\ |\ \bm{y})|^2\\
        &=& \|\bm{y}\|^2(\|\bm{x}\|^2\|\bm{y}\|^2 - |(\bm{x}\ |\ \bm{y})|^2)
    \end{eqnarray*}
    今 $\|\bm{y}\|^2 \neq 0$ より, 両辺 $\|\bm{y}\|^2$ で割り, 平方根をとれば $|(\bm{x}\ |\ \bm{y})| \leq \|\bm{x}\|\cdot \|\bm{y}\|$ が成り立つ.
\end{enumerate}
以上より, シュワルツの不等式が示された.
\item まず, 通常の三角不等式 $\|\bm{x} + \bm{y}\| \leq \|\bm{x}\| + \|\bm{y}\|$ を示す.
これは(1)のシュワルツの不等式を利用することで次のように示される.
\begin{eqnarray*}
    \|\bm{x} + \bm{y}\|^2 &=& \|\bm{x}\|^2 + 2(\bm{x}\ |\ \bm{y}) + \|\bm{y}\|^2\\
    &\leq& \|\bm{x}\|^2 + 2\|\bm{x}\||\bm{y}\| + |\bm{y}\|^2\\
    &=& (\|\bm{x}\| + \|\bm{y}\|)^2
\end{eqnarray*}
この三角不等式より, $\|\bm{x} - \bm{y} + \bm{y} - \bm{z}\| \leq \|\bm{x} -\bm{y}\| + \|\bm{y} - \bm{z}\|$ が成り立ち, $d(\bm{x}, \bm{y}) = \|\bm{x} - \bm{y}\|$ より, 
$d(\bm{x}, \bm{z}) \leq d(\bm{x}, \bm{y}) + d(\bm{y}, \bm{z})$ が成り立つ.
}
\newpage
前問について補足する.

まず, シュワルツの不等式を示す際に用いた証明は天下りすぎる. そこで, 二次関数と判別式を用いた証明が良く本で紹介されている.
これは, $\|\bm{x}\|^2t^2 - 2(\bm{x}\ |\ \bm{y})t + \|\bm{y}\|^2 = \| t\bm{x} - \bm{y}\|^2 \geq 0$ の判別式が $D \leq 0$ であることから証明する.
この手法からは, 等号成立条件が $\bm{x} = t\bm{y}$ となる実数 $t$ が存在することとすぐにわかる. 一方, 先の天下りな証明からは等号成立条件はわかりにくい.
先の証明からは, 等号が成立することから $\| \|\bm{y}\|^2\|\bm{x}\| - |(\bm{x}\ |\ \bm{y})| \|\bm{y}\|^2 \|^2 = 0$ より $t = (\bm{x}\ |\ \bm{y})/(\bm{y}\ |\ \bm{y})$ と具体的な $t$ を示すことと, 
逆に$\bm{x} = t\bm{y}$ を代入することから, 等号が成立することを示す. 

また, 三角不等式の方では等号成立条件はシュワルツの不等式の等号成立条件と $(\bm{x}\ |\ \bm{y}) = |(\bm{x}\ |\ \bm{y})|$ をまとめた, 
「$(\bm{x}\ |\ \bm{y}) \geq 0$ かつ $\bm{x} = t\bm{y}$ となる実数 $t$ が存在する」こととなる.
\end{nmprob}



\begin{nmprob}
\pbenum[nhk]{
\item 二つの集合$A, B$に対して, 次が成り立つことを示せ.
\begin{equation*}
    A \not\subset B \iff A \cap B^c \neq \emptyset
\end{equation*}
ただし, この問題以降 $A$ が $B$ の部分集合であることを $A \subset B$ と示すこととする.
\item $\langle X, d\rangle$ を距離空間とし, $A$ を $X$ の部分集合とする. このとき, $A$ の内部 $A^\circ$, 閉包 $\overline{A}$, 境界 $\partial A$ を次のように定義する. ただし, $x$ の近傍の全体を $\bm{V}(x)$ とする
(すなわち, $\bm{V}(x) \coloneqq \{ V \subset X\ |\ \exists \epsilon > 0, B(x\ ;\epsilon) \subset V\}$) また, $B(x\ ;\epsilon)$ は半径$\epsilon$ の開球($\epsilon$ -近傍)である.
\begin{eqnarray*}
    A^\circ &\coloneqq& \{ x \in X\ |\ \exists \epsilon > 0, B(x\ ;\epsilon) \subset A\}\\
    \overline{A} &\coloneqq& \{ x \in X\ |\ \forall V \in \bm{V}(x), V \cap A \neq \emptyset\}\\
    \partial A &\coloneqq& \{ x \in X\ |\ \forall V \in \bm{V}(x), V \cap A \neq \emptyset \land V - A \neq \emptyset \}
\end{eqnarray*} 
このとき, $(A^\circ)^c = \overline{A^c}$ となることを示せ.
}{
\item 以下の同値変形により示される.
\begin{eqnarray*}
    A \not\subset B &\iff& \exists x,\ \lnot (x \in A \implies x \in B )\\
    &\iff& \exists x,\ \lnot (\lnot(x \in A) \lor x \in B )\\
    &\iff& \exists x,\ x \in A \land x \in B^c\\
    &\iff& A \cap B^c \neq \emptyset
\end{eqnarray*}
\item まず, 以下の同値変形
\begin{eqnarray*}
    x \in (A^\circ)^c &\iff& x \notin A^\circ\\
    &\iff& \lnot(\exists \epsilon > 0)[B(x\ ;\epsilon) \subset A]\\
    &\iff& (\forall \epsilon > 0)[B(x\ ;\epsilon) \not\subset A]\\
    &\iff& (\forall \epsilon > 0)[B(x\ ;\epsilon) \cap A^c \neq \emptyset]\hspace{1zw}\text{($\because$ (1)より)}
\end{eqnarray*}
より $x \in (A^\circ)^c \iff (\forall \epsilon > 0)[B(x\ ;\epsilon) \cap A^c \neq \emptyset]$ が成り立つ.\\
これより, $(\forall \epsilon > 0)[B(x\ ;\epsilon) \cap A^c \neq \emptyset] \iff x \in \overline{A^c}$ が成り立つことを示せばよい.
\begin{enumerate}
    \item $\implies$\\
    近傍の定義より, 任意の $V \in \bm{V}(x)$に対し, $B(x\ ;\epsilon_0) \subset V$ となる $\epsilon_0 > 0$ が存在するが, 今, この $\epsilon_0$ に対して $B(x\ ;\epsilon_0) \cap A^c \neq \emptyset$
    となるので, $V \cap A^c \neq \emptyset$ となる. これより, $x \in \overline{A^c}$ となる.
    \item $\impliedby$\\ 
    任意の $\epsilon > 0$ に対して, $B(x\ ;\epsilon) \subset B(x\ ;\epsilon)$ が成り立つので, 任意の $\epsilon > 0$ に対して, $B(x\ ;\epsilon) \in \bm{V}(x)$ となる.
    よって, 閉包の定義より, $x \in \overline{A^c}$ となるとき, $(\forall \epsilon > 0)[B(x\ ;\epsilon) \cap A^c \neq \emptyset]$ が成り立つ.
\end{enumerate}
以上より, $(A^\circ)^c = \overline{A^c}$ が成り立つ.
}
\newpage
\hspace{-3zw}{\color{purple}●●\ 内部, 閉包, 境界の定義における補足\ ●●}

まず, なぜ内部の定義における $\epsilon$ の量化子がなぜ全称ではなく存在なのかについて. これは以下の図1をイメージするといい.
$\epsilon$ が大きい $\bm{x}$ の開球は $A$ に含まれなくなる.
\begin{figure}[htbp]
    \centering
    \begin{tikzpicture}
        \coordinate (A);
        \coordinate (C) at ($(A) + (-45:1.2)$);
        \coordinate (e1) at ($(C) + (-30:0.5)$);
        \coordinate (e2) at ($(C) + (120:1)$);
    
        \draw (A) node {A} circle(2);
        \fill (C) node [left] {$\bm{x}$} circle(0.05);
        \draw[dotted] (C) circle(0.5);
        \draw[dotted] (C) circle(1);
        \draw[dashed] (C) -- (e1);
        \draw[dashed] (C) -- (e2);
        \draw [bend right,distance=16] (C) to node [fill=white,inner sep=0.1pt,circle] {$\epsilon_1$} (e1);
        \draw [bend right,distance=14] (C) to node [fill=white,inner sep=0.1pt,circle] {$\epsilon_2$} (e2);
    \end{tikzpicture}
    \caption{内部のイメージ}
\end{figure}

次に, なぜ閉包の定義があのような定義になっているかについて. これは以下の図2をイメージするといい.
$\bm{x}$ のような境界上の点もきちんと閉包に含まれている.
\begin{figure}[htbp]
\centering
\begin{tikzpicture}
    \coordinate (A);
    \draw (A) node {A} circle(2);

    \coordinate (C) at ($(A) + (-45:2)$);
    \coordinate (e1) at ($(C) + (-30:0.5)$);
    \coordinate (e2) at ($(C) + (120:0.8)$);

    \fill (C) node [left] {$\bm{x}$} circle(0.05);
    \draw[dotted] (C) circle(0.5);
    \draw[dotted] (C) circle(0.8);
    \draw[dashed] (C) -- (e1);
    \draw[dashed] (C) -- (e2);
    \draw [bend right,distance=16] (C) to node [fill=white,inner sep=0.1pt,circle] {$\epsilon_1$} (e1);
    \draw [bend right,distance=14] (C) to node [fill=white,inner sep=0.1pt,circle] {$\epsilon_2$} (e2);

    \coordinate (D) at ($(A) + (135:1)$);
    \fill (D) node [left] {$\bm{y}$} circle(0.05);
    \draw[dotted] (D) circle(0.5);
    \draw[dotted] (D) circle(1.2);
    \coordinate (e3) at ($(D) + (-30:0.5)$);
    \coordinate (e4) at ($(D) + (120:1.2)$);
    \draw[dashed] (D) -- (e3);
    \draw[dashed] (D) -- (e4);
    \draw [bend left,distance=20] (D) to node [fill=white,inner sep=0.1pt,circle] {${\epsilon_1}^{'}$} (e3);
    \draw [bend right,distance=20] (D) to node [fill=white,inner sep=0.1pt,circle] {${\epsilon_2}^{'}$} (e4);
\end{tikzpicture}
\caption{閉包と境界のイメージ}
\end{figure}

最後に, 境界の定義だが, これは上図2の $\bm{y}$ ように閉包から境界以外の点を除いている.
\end{nmprob}
\setcounter{figure}{0}


\begin{nmprob}
\pbenumex{
$\langle X, d\rangle$ を距離空間とし, $A$ を $X$ の部分集合とする. このとき, 次の問に答えよ. ただし, 内部, 境界, 閉包の定義は前問と同じものとする.
}{
\item $A^\circ \subset A \subset \overline{A}$ を示せ.
\item $\partial A = \overline{A} - A^\circ$ を示せ.
\item $A$ の外部 $A^e$ を $A^e \coloneqq (A^c)^\circ$ のように定義する. このとき, $A^\circ \cup \partial A \cup A^e = X$ であることを示せ.
\item $A^\circ, \partial A, A^e$ がそれぞれ互いに素であることを示せ.
}{
\item まず, $x \in A^\circ$ のとき, $\exists \epsilon > 0, B(x\ ;\epsilon) \subset A$ であり, $x \in B(x\ ;\epsilon)$ とから $A^\circ \subset A$ となる.\\
次に, $x \in A$ のとき, 任意の $x$ の近傍 $V$ に関して, 近傍の定義より $x \in V$ となる. よって $V \cap A \neq \emptyset$ より, $A \subset \overline{A}$ となる.\\
以上より, $A^\circ \subset A \subset \overline{A}$ が成り立つ.
\item まず, $\partial A \subset \overline{A} - A^\circ$ を示す.\\
境界と閉包の定義より $\partial A \subset \overline{A}$ は明らか. よって $x \in \partial A \implies x \notin A^\circ$ を示せばよい.\\
さて, 任意の $\epsilon > 0$ に対して $B(x\ ;\epsilon) \in \bm{V}(x)$ である(問\ref{sec:topology}-\ref{pname:nhk}\ (2)参照)から, $x \in \partial A$ のとき, 境界の定義より $B(x\ ;\epsilon) - A \neq \emptyset$ が成り立つ.
これより, $B(x\ ;\epsilon) \not\subset A$ となる. 今, 任意の $\epsilon > 0$ に対して, $B(x\ ;\epsilon) \not\subset A$ が示され, これは $x \notin A^\circ$ であることに他ならない.\\
よって, $\partial A \subset \overline{A} - A^\circ$ が成り立つ.\\
次に, $\overline{A} - A^\circ \subset \partial A$ を示す.\\
これは, 先の議論を逆にたどることにより示せる(問\ref{sec:topology}-\ref{pname:nhk}\ (2)を参照するとよい)\\
以上より, $\partial A = \overline{A} - A^\circ$ が成り立つ.
\item まず, $\overline{A} = A^\circ \cup \partial A$ となることを示す.\\
これは次のように示される.
\begin{align*}
    A^\circ \cup \partial A &= A^\circ \cup (\overline{A} - A^\circ)&&\text{($\because$ (2)より)}\\
                            &= \overline{A}&&\text{($\because$ (1)より$A^\circ \subset \overline{A}$)} 
\end{align*}
次に, 問\ref{sec:topology}-\ref{pname:nhk}\ (2)より, $(A^\circ)^c = \overline{A^c}$ が成り立つから, $(A^e)^c = ((A^c)^\circ)^c = \overline{(A^c)^c} = \overline{A}$ が成り立つ. これより, $A^e \cup \overline{A} = X$ より, 
$\overline{A} = A^\circ \cup \partial A$ とから, $A^\circ \cup \partial A \cup A^e = X$ が成り立つ.
\item
まず, $A^\circ \cup \partial A$ だが, これは(2)より明らかに $A^\circ \cup \partial A = \emptyset$ が成り立つ.\\
次に $A^\circ \cap A^e = \emptyset$ を背理法により示す.\\
$A^\circ \cap A^e$ の元の存在を仮定すると, (1)より $x \in A^\circ \cap A^e \implies x \in \overline{A} \cap A^e$ が成り立つが, (3)より $(A^e)^c = \overline{A}$ であるから, $\overline{A} \cap A^e = \emptyset$
で矛盾. よって, $A^\circ \cap A^e = \emptyset$ が成り立つ.\\
最後に $\partial A \cap A^e = \emptyset$ を背理法により示す.\\
$\partial A \cap A^e$ の元の存在を仮定すると, (2)より $x \in \partial A \cap A^e \implies x \in \overline{A} \cap A^e$ が成り立ち, 先と同様にして矛盾. よって, $\partial A \cap A^e = \emptyset$ が成り立つ.\\
以上より, $A^\circ, \partial A, A^e$ はそれぞれ互いに素である.
}
\end{nmprob}
 % 位相
\pgsc{解析学}{midnightblue}{analysis}

\begin{nmprob}
制作中.
\pbenumex{
    制作中
}{
\item 未定
}{
\item 未定
}
\end{nmprob}
 % 解析学
\pgsc{問題置き場}{black}{pstock}

\begin{nmprob}
編集の都合上, 後から追加する予定の問題. 基本的に応用的な問題.
\pbenumex{
    以降の問題では, 正の整数全体の集合を $\mathbb{Z}^+$ で表す. このとき, 次の問に答えよ.
}{
\item 次の条件を満たす $\mathbb{Z}^+$ の部分集合 $S$ を考える.
\begin{align}
    &1 \in S\\
    &\forall n \in \mathbb{Z}^+ [n \in S \implies n + 1 \in s]
\end{align}
整列性(任意の空でない自然数の集合は最小限を持つ)を認めた上で, $S = \mathbb{Z}^+$ を満たすことを示せ.
\item $\mathbb{Z}^+$ の元の各々に対し, 命題 $P(n)$ が与えられたとし, それについて次の二つのことが示されたとする.
\setcounter{equation}{0}
\begin{align}
    &P(1) \text{は真}\\
    &\forall n \in \mathbb{Z}^+ [P(n) \text{が真} \implies P(n+1) \text{も真}]
\end{align}
このとき, 全ての $n \in \mathbb{Z}^+$ に対して, $P(n)$ が真となることを示せ.
}{
\item $\mathbb{Z}^+ - S = S^{'}$ とし, $S^{'} = \emptyset$ となることを背理法により示す.\\
$S^{'} \neq \emptyset$ と仮定すると, 整列性より $n_0 = \min S^{'}$ となる $n_0 \in \mathbb{Z}^+$ が存在する.
今, $n_0 > 1$ より, $n_0 - 1 \geq 1$ となり $n_0 - 1 \in \mathbb{Z}^+$ である.
ここで $n_0 - 1 < n_0 = \min \mathbb{Z}^+$ より $n_0 - 1 \notin S^{'}$ , すなわち $n_0 - 1 \in S$ となる.\\
これより, $S$ が満たす条件(2)より, $n_0 \in S$ となり, $n_0 \in S^{'}$ に矛盾.\\
以上より, $S^{'} = \emptyset$ であり, $S = \mathbb{Z}^+$ が成り立つ.
\item $S =  \{\ n \in \mathbb{Z}^+\ |\ P(n) \text{が真} \}$ と集合 $S$ を定義すると, (1)より $S = \mathbb{Z}^+$ が成り立つ.
よって, 全ての $n \in \mathbb{Z}^+$ に対して, $P(n)$ が真となる\footnote[1]{tmp}
}
\end{nmprob}
 % 問題ストック

\end{document}

