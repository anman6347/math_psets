\documentclass[dvipdfmx,a4j]{jarticle}

\usepackage{fancyhdr}
\usepackage{geometry}
\usepackage{tikz}

\geometry{top=120pt,bottom=60pt,left=70pt,right=70pt,headheight=20pt,headsep=4pt}

\usepackage{xcolor}
\definecolor{darkolivegreen}{RGB}{85,107,47}
\definecolor{chocolate}{RGB}{210,105,30}
\definecolor{midnightblue}{RGB}{25,25,112}
\definecolor{maroon}{RGB}{128,0,0}

\usepackage{tcolorbox}
\tcbuselibrary{raster,skins,breakable}

\usepackage{amsmath}
\usepackage{amssymb}
\usepackage{amsfonts}
\usepackage{txfonts}
\usepackage{bm}

\pagestyle{fancy}

\newcounter{secn}   % 章のカウンタ
\setcounter{secn}{-1}
\newcounter{pn}     % 章ごとのページカウンタ

\newcommand{\cnttwo}[1]{
    \ifnum \value{#1}<10
        0\arabic{#1}
    \else
        \arabic{#1}
    \fi
}

% 通常ページの設定
\fancypagestyle{pg}{
    \newgeometry{top=100pt,bottom=60pt,left=70pt,right=70pt,headheight=20pt,headsep=4pt}
    \lhead{
        \begin{tikzpicture}[remember picture,overlay,shift={(current page.north)}]
            \fill [black] (-9.65, -1.9) node[right] {{\Large \textbf{数学問題集}}};
            \fill [black] (9.15, -1.9) node[left] {{\Large \textbf{プリント}}{\large \cnttwo{secn}}{\normalsize -\cnttwo{pn}}};
            \fill [black] (-9.5,-2.8) rectangle (9, -2.2);
            \fill [white] (-9.2,-2.5) node[right] {{\normalsize \textbf{制作者 : }}{\large \textbf{Anman}}};
        \end{tikzpicture}
    }
    \lfoot{
        \begin{tikzpicture}[remember picture,overlay,shift={(current page.south)}]
            \fill [black] (-9.5, 1) rectangle (9, 1.6);
            \fill [white] (0, 1.3) node {\thepage};
        \end{tikzpicture}
    }
    \cfoot{}
    \renewcommand{\headrulewidth}{0pt}
}

% 通常ページのデザインをデフォルトにする
\pagestyle{pg}


% 章立てページのデザインを現在のページに反映
\newcommand{\pgs}[1]{
    \setcounter{pn}{0}
    \addtocounter{secn}{1}

    \fancypagestyle{pgs}{
        \newgeometry{top=130pt,bottom=60pt,left=70pt,right=70pt,headheight=20pt,headsep=4pt}
        \lhead{
            \begin{tikzpicture}[remember picture,overlay,shift={(current page.north)}]
                \fill [black] (-9.65, -1.9) node[right] {{\Large \textbf{数学問題集}}};
                \fill [black] (9.15, -1.9) node[left] {{\Large \textbf{プリント}}{\large \cnttwo{secn}}{\normalsize -\cnttwo{pn}}};
                \fill [black] (-9.5,-2.8) rectangle (9, -2.2);
                \fill [white] (-9.2,-2.5) node[right] {{\normalsize \textbf{制作者 : }}{\large \textbf{Anman}}};
                \fill [black] (0, -3.2) node {{\LARGE \textbf{#1}}};
                \fill [black] (-9.5,-4) rectangle (9, -3.5);
                \fill [white] (-9.2,-3.8) node[right] {{\small \textbf{更新日 : 4/6/2022 version 0.1}}};
            \end{tikzpicture}
        }
        \lfoot{
            \begin{tikzpicture}[remember picture,overlay,shift={(current page.south)}]
                \fill [black] (-9.5, 1) rectangle (9, 1.6);
                \fill [white] (0, 1.3) node {\thepage};
            \end{tikzpicture}
        }
        \cfoot{}
        \renewcommand{\headrulewidth}{0pt}
    }
    \thispagestyle{pgs}
}
% \pgs{タイトル名前}


% 章立てページのデザインを現在のページに反映(色も設定する)
% 以降の通常ページの色も変更
\newcommand{\pgsc}[2]{
    \setcounter{pn}{-1} % たぶん \pagestyleと\thispagestyle で ページが増えた扱いになるため?
    \addtocounter{secn}{1}

    \fancypagestyle{pgs}{
        \newgeometry{top=130pt,bottom=60pt,left=70pt,right=70pt,headheight=20pt,headsep=4pt}
        \lhead{
            \begin{tikzpicture}[remember picture,overlay,shift={(current page.north)}]
                \fill [#2] (-9.65, -1.9) node[right] {{\Large \textbf{数学問題集}}};
                \fill [#2] (9.15, -1.9) node[left] {{\Large \textbf{プリント}}{\large \cnttwo{secn}}{\normalsize -\cnttwo{pn}}};
                \fill [#2] (-9.5,-2.8) rectangle (9, -2.2);
                \fill [white] (-9.2,-2.5) node[right] {{\normalsize \textbf{制作者 : }}{\large \textbf{Anman}}};
                \fill [#2] (0, -3.2) node {{\LARGE \textbf{#1}}};
                \fill [#2] (-9.5,-4) rectangle (9, -3.5);
                \fill [white] (-9.2,-3.8) node[right] {{\small \textbf{更新日 : 4/6/2022 version 0.1}}};
            \end{tikzpicture}
        }
        \lfoot{
            \begin{tikzpicture}[remember picture,overlay,shift={(current page.south)}]
                \fill [#2] (-9.5, 1) rectangle (9, 1.6);
                \fill [white] (0, 1.3) node {\thepage};
            \end{tikzpicture}
        }
        \cfoot{}
        \renewcommand{\headrulewidth}{0pt}
    }


    \fancypagestyle{pg}{
        \newgeometry{top=100pt,bottom=60pt,left=70pt,right=70pt,headheight=20pt,headsep=4pt}
        \lhead{
            \begin{tikzpicture}[remember picture,overlay,shift={(current page.north)}]
                \fill [#2] (-9.65, -1.9) node[right] {{\Large \textbf{数学問題集}}};
                \fill [#2] (9.15, -1.9) node[left] {{\Large \textbf{プリント}}{\large \cnttwo{secn}}{\normalsize -\cnttwo{pn}}};
                \fill [#2] (-9.5,-2.8) rectangle (9, -2.2);
                \fill [white] (-9.2,-2.5) node[right] {{\normalsize \textbf{制作者 : }}{\large \textbf{Anman}}};
            \end{tikzpicture}
        }
        \lfoot{
            \begin{tikzpicture}[remember picture,overlay,shift={(current page.south)}]
                \fill [#2] (-9.5, 1) rectangle (9, 1.6);
                \fill [white] (0, 1.3) node {\thepage};
            \end{tikzpicture}
        }
        \cfoot{}
        \renewcommand{\headrulewidth}{0pt}
    }
    \pagestyle{pg}
    \thispagestyle{pgs}
}

% カウンタ設定
\let\np\newpage
\renewcommand{\newpage}{
    \np
    \addtocounter{pn}{1}
}

% 注釈
\newcommand{\annot}{
    \hrule width 6cm
    \vspace{1mm}
}
% text...\\
% \annot
% annotation...

% 問題枠のテンプレ
\newcounter{numpb} %%問題番号
\setcounter{numpb}{0}
\newtcolorbox{pb}[1][]{enhanced,boxrule=0.5mm,
    top=6pt,left=6pt,right=4pt,bottom=6pt,arc=0mm,
    frame style={left color=green!60!black, right color=blue!50!white},
    colback=white,
    boxrule=1pt,
    %underlay={
    %\node[black] at ([xshift=20pt,yshift=-13pt]interior.north west) {\stepcounter{numpb}\textbf{問\arabic{secn}--\arabic{numpb}.}};
    %},
    segmentation code={
        \draw[dashed] (segmentation.west)--(segmentation.east);
        %\node[red!70!white] at ([xshift=29pt,yshift=-7pt]segmentation.south west) {\textbf{解.}};
    },
    before upper={\setlength{\parindent}{1zw}},
    before lower={\setlength{\parindent}{1zw}},
}
\renewcommand{\labelenumi}{(\arabic{enumi})}
\renewcommand{\labelenumii}{(\roman{enumii})}
% 使い方
%\begin{pb}
%    \noindent
%    \stepcounter{numpb}\textbf{問\arabic{secn}--\arabic{numpb}.} 次の問に答えなさい.
%    \begin{enumerate}
%        \setlength{\parskip}{0cm} % 段落間
%        \setlength{\itemsep}{0cm} % 項目間
%        \setlength{\leftskip}{-8pt} % 左からの頭の位置
%        \item ara
%        \item aaa
%    \end{enumerate}
%    \tcblower
%    \noindent
%    \textbf{解.}\\
%    bbb
%\end{pb}


% 問題テンプレ箇条書きver
\newcommand{\pbenum}[2]{
    \vspace{2mm}
    \hrule
    \vspace{2mm}
    \noindent
    \stepcounter{numpb}\textbf{問\arabic{secn}--\arabic{numpb}:} 次の問に答えよ.
        \begin{enumerate}
            \setlength{\leftskip}{-8pt} % 左からの頭の位置
            #1
        \end{enumerate}
    \hrule
    \vspace{2mm}
    {\color{red}\textbf{解:}}
    \begin{enumerate}
        \setlength{\leftskip}{-8pt} % 左からの頭の位置
        #2
    \end{enumerate}
}

% 問題テンプレ箇条書きver 最初のメッセージも書き換える.
\newcommand{\pbenumex}[3]{
    \vspace{2mm}
    \hrule
    \vspace{2mm}
    \noindent
    \stepcounter{numpb}\textbf{問\arabic{secn}--\arabic{numpb}:} #1
        \begin{enumerate}
            \setlength{\leftskip}{-8pt} % 左からの頭の位置
            #2
        \end{enumerate}
    \hrule
    \vspace{2mm}
    {\color{red}\textbf{解:}}
    \begin{enumerate}
        \setlength{\leftskip}{-8pt} % 左からの頭の位置
        #3
    \end{enumerate}
}


% 本文
\begin{document}
\pgs{自習用数学問題集}% 章立てページ
この資料は, 自習用の数学問題集です. 証明問題に比重を置いています. 易しくはありません.\\
問題の順番に規則性はほとんどなく, 分野はごっちゃ, 難易度順に並んでいるわけでもありません. ただし, 後の問題で使うような知識が前半に来ないようには作成しているつもりです.

数学記号や用語の定義は, 教科書ごとに著しく差があれば問題文中で定義しますが, 多くの教科書で使われているものは一々定義することはありません.

また, この問題集の解答が正しい保証はありません. 参考にする場合は気を付けてください.
\newpage
\pgsc{集合と位相}{darkolivegreen}
まずは数学の道具(概念)に慣れるための基礎問題.\\
\pbenum{
\item $n$次元実空間における任意の2点 $\bm{x}, \bm{y} \in \mathbb{R}^n$に対し, シュワルツの不等式
\begin{equation*}
    |(\bm{x}\ |\ \bm{y})| \leq \|\bm{x}\|\cdot \|\bm{y}\|
\end{equation*}
を示せ.
\item $n$次元実空間$\mathbb{R}^n$における2点$\bm{x} = (x_1, x_2, \cdots, x_n)$と$\bm{y} = (y_1, y_2, \cdots, y_n)$の距離$d(\bm{x}, \bm{y})$を次のように定義する.
\begin{equation*}
    d(\bm{x}, \bm{y}) = \sqrt{\sum_{i=1}^{n}(x_i - y_i)^2}
\end{equation*}
このとき, 三角不等式 $d(\bm{x}, \bm{z}) \leq d(\bm{x}, \bm{y}) + d(\bm{y}, \bm{z})$ を示せ.
}{
\item いくつかやり方があるが, ここでは天下りだが代数的に済む解法を示す.
\begin{enumerate}
    \item $\bm{y} = \bm{0}$ のとき\\
    \ \\
    両辺 0 で成立.
    \item $\bm{y} \neq \bm{0}$ のとき\\
    \ \\
    任意の実数 $a, b$ に対して, 
    \begin{eqnarray*}
        0 &\leq& \|a\bm{x}+b\bm{y}\|^2\\
        &=& \|a\|^2\|\bm{x}\|^2 + 2ab\cdot (\bm{x}\ |\ \bm{y}) + \|b\|^2\|\bm{y}\|^2
    \end{eqnarray*}
    ここで, $a = \|\bm{y}\|^2,\ b = -(\bm{x}\ |\ \bm{y})$ とすると,
    \begin{eqnarray*}
        0 &\leq& \|\bm{y}\|^4\|\bm{x}\|^2-2\|\bm{y}\|^2|(\bm{x}\ |\ \bm{y})|^2 + |\bm{y}\|^2|(\bm{x}\ |\ \bm{y})|^2\\
        &=& \|\bm{y}\|^2(\|\bm{x}\|^2\|\bm{y}\|^2 - |(\bm{x}\ |\ \bm{y})|^2)
    \end{eqnarray*}
    今 $\|\bm{y}\|^2 \neq 0$ より, 両辺 $\|\bm{y}\|^2$ で割り, 平方根をとれば $|(\bm{x}\ |\ \bm{y})| \leq \|\bm{x}\|\cdot \|\bm{y}\|$ が成り立つ.
\end{enumerate}
以上より, シュワルツの不等式が示された.
\item まず, 通常の三角不等式 $\|\bm{x} + \bm{y}\| \leq \|\bm{x}\| + \|\bm{y}\|$ を示す.
これは(1)のシュワルツの不等式を利用することで次のように示される.
\begin{eqnarray*}
    \|\bm{x} + \bm{y}\|^2 &=& |\bm{x}\|^2 + 2(\bm{x}\ |\ \bm{y}) + |\bm{y}\|^2\\
    &\leq& \|\bm{x}\|^2 + 2\|\bm{x}\||\bm{y}\| + |\bm{y}\|^2\\
    &=& (\|\bm{x}\| + \|\bm{y}\|)^2
\end{eqnarray*}
この三角不等式より, $\|\bm{x} - \bm{y} + \bm{y} - \bm{z}\| \leq \|\bm{x} -\bm{y}\| + \|\bm{y} - \bm{z}\|$ が成り立ち, $d(\bm{x}, \bm{y}) = \|\bm{x} - \bm{y}\|$ より, 
$d(\bm{x}, \bm{z}) \leq d(\bm{x}, \bm{y}) + d(\bm{y}, \bm{z})$ が成り立つ.
}
\newpage
前問について補足する.

まず, シュワルツの不等式を示す際に用いた証明は天下りすぎる. そこで, 二次関数と判別式を用いた証明が良く本で紹介されている.
これは, $\|\bm{x}\|^2t^2 - 2(\bm{x}\ |\ \bm{y})t + \|\bm{y}\|^2 = \| t\bm{x} - \bm{y}\|^2 \geq 0$ の判別式が $D \leq 0$ であることから証明する.
この手法からは, 等号成立条件が $\bm{x} = t\bm{y}$ となる実数 $t$ が存在することとすぐにわかる. 一方, 先の天下りな証明からは等号成立条件はわかりにくい.
先の証明からは, 等号が成立することから $\| \|\bm{y}\|^2\|\bm{x}\| - |(\bm{x}\ |\ \bm{y})| \|\bm{y}\|^2 \|^2 = 0$ より $t = (\bm{x}\ |\ \bm{y})/(\bm{y}\ |\ \bm{y})$ と具体的な $t$ を示すことと, 
逆に$\bm{x} = t\bm{y}$ を代入することから, 等号が成立することを示す. 

また, 三角不等式の方では等号成立条件はシュワルツの不等式の等号成立条件と $(\bm{x}\ |\ \bm{y}) = |(\bm{x}\ |\ \bm{y})|$ をまとめた, 
「$(\bm{x}\ |\ \bm{y}) \geq 0$ かつ $\bm{x} = t\bm{y}$ となる実数 $t$ が存在する」こととなる.


\newpage
\pbenum{
\item 二つの集合$A, B$に対して, 次が成り立つことを示せ.
\begin{equation*}
    A \not\subset B \iff A \cap B^c \neq \emptyset
\end{equation*}
ただし, この問題以降 $A$ が $B$ の部分集合であることを $A \subset B$ と示すこととする.
\item $\langle X, d\rangle$ を距離空間とし, $A$ を $X$ の部分集合とする. このとき, $A$ の内部 $A^\circ$, 閉包 $\overline{A}$, 境界 $\partial A$ を次のように定義する. ただし, $x$ の近傍の全体を $\bm{V}(x)$ とする
(すなわち, $\bm{V}(x) \coloneqq \{ V \subset X\ |\ \exists \epsilon > 0, B(x\ ;\epsilon) \subset V\}$) また, $B(x\ ;\epsilon)$ は半径$\epsilon$ の開球($\epsilon$ -近傍)である.
\begin{eqnarray*}
    A^\circ &\coloneqq& \{ x \in X\ |\ \exists \epsilon > 0, B(x\ ;\epsilon) \subset A\}\\
    \overline{A} &\coloneqq& \{ x \in X\ |\ \forall V \in \bm{V}(x), V \cap A \neq \emptyset\}\\
    \partial A &\coloneqq& \{ x \in X\ |\ \forall V \in \bm{V}(x), V \cap A \neq \emptyset \land V - A \neq \emptyset \}
\end{eqnarray*} 
このとき, $(A^\circ)^c = \overline{A^c}$ となることを示せ.
}{
\item 以下の同値変形により示される.
\begin{eqnarray*}
    A \not\subset B &\iff& \exists x,\ \lnot (x \in A \implies x \in B )\\
    &\iff& \exists x,\ \lnot (\lnot(x \in A) \lor x \in B )\\
    &\iff& \exists x,\ x \in A \land x \in B^c\\
    &\iff& A \cap B^c \neq \emptyset
\end{eqnarray*}
\item まず, 以下の同値変形
\begin{eqnarray*}
    x \in (A^\circ)^c &\iff& x \notin A^\circ\\
    &\iff& \lnot(\exists \epsilon > 0)[B(x\ ;\epsilon) \subset A]\\
    &\iff& (\forall \epsilon > 0)[B(x\ ;\epsilon) \not\subset A]\\
    &\iff& (\forall \epsilon > 0)[B(x\ ;\epsilon) \cap A^c \neq \emptyset]\hspace{1zw}\text{($\because$ (1)より)}
\end{eqnarray*}
より $x \in (A^\circ)^c \iff (\forall \epsilon > 0)[B(x\ ;\epsilon) \cap A^c \neq \emptyset]$ が成り立つ.\\
これより, $(\forall \epsilon > 0)[B(x\ ;\epsilon) \cap A^c \neq \emptyset] \iff x \in \overline{A^c}$ が成り立つことを示せばよい.
\begin{enumerate}
    \item $\implies$\\
    近傍の定義より, 任意の $V \in \bm{V}(x)$に対し, $B(x\ ;\epsilon_0) \subset V$ となる $\epsilon_0 > 0$ が存在するが, 今, この $\epsilon_0$ に対して $B(x\ ;\epsilon_0) \cap A^c \neq \emptyset$
    となるので, $V \cap A^c \neq \emptyset$ となる. これより, $x \in \overline{A^c}$ となる.
    \item $\impliedby$\\ 
    任意の $\epsilon > 0$ に対して, $B(x\ ;\epsilon) \subset B(x\ ;\epsilon)$ が成り立つので, 任意の $\epsilon > 0$ に対して, $B(x\ ;\epsilon) \in \bm{V}(x)$ となる.
    よって, 閉包の定義より, $x \in \overline{A^c}$ となるとき, $(\forall \epsilon > 0)[B(x\ ;\epsilon) \cap A^c \neq \emptyset]$ が成り立つ.
\end{enumerate}
以上より, $(A^\circ)^c = \overline{A^c}$ が成り立つ.
}

\newpage
\pbenumex{
$\langle X, d\rangle$ を距離空間とし, $A$ を $X$ の部分集合とする. このとき, 次の問に答えよ. ただし, 内部, 境界, 閉包の定義は前問と同じものとする.
}{
\item $A^\circ \subset A \subset \overline{A}$ を示せ.
\item $\partial A = \overline{A} - A^\circ$ を示せ.
\item $A$ の外部 $A^e$ を $A^e \coloneqq (A^c)^\circ$ のように定義する. このとき, $A^\circ \cup \partial A \cup A^e = X$ であることを示せ.
\item $A^\circ, \partial A, A^e$ がそれぞれ互いに素であることを示せ.
}{
\item まず, $x \in A^\circ$ のとき, $\exists \epsilon > 0, B(x\ ;\epsilon) \subset A$ であり, $x \in B(x\ ;\epsilon)$ とから $A^\circ \subset A$ となる.\\
次に, $x \in A$ のとき, 任意の $x$ の近傍 $V$ に関して, 近傍の定義より $x \in V$ となる. よって $V \cap A \neq \emptyset$ より, $A \subset \overline{A}$ となる.\\
以上より, $A^\circ \subset A \subset \overline{A}$ が成り立つ.
\item まず, $\partial A \subset \overline{A} - A^\circ$ を示す.\\
境界と閉包の定義より $\partial A \subset \overline{A}$ は明らか. よって $x \in \partial A \implies x \notin A^\circ$ を示せばよい.\\
さて, 任意の $\epsilon > 0$ に対して $B(x\ ;\epsilon) \in \bm{V}(x)$ である(問1-2\ (2)参照)から, $x \in \partial A$ のとき, 境界の定義より $B(x\ ;\epsilon) - A \neq \emptyset$ が成り立つ.
これより, $B(x\ ;\epsilon) \not\subset A$ となる. 今, 任意の $\epsilon > 0$ に対して, $B(x\ ;\epsilon) \not\subset A$ が示され, これは $x \notin A^\circ$ であることに他ならない.\\
よって, $\partial A \subset \overline{A} - A^\circ$ が成り立つ.\\
次に, $\overline{A} - A^\circ \subset \partial A$ を示す.\\
これは, 先の議論を逆にたどることにより示せる(問1-2\ (2)を参照するとよい)\\
以上より, $\partial A = \overline{A} - A^\circ$ が成り立つ.
\item まず, $\overline{A} = A^\circ \cup \partial A$ となることを示す.\\
これは次のように示される.
\begin{align*}
    A^\circ \cup \partial A &= A^\circ \cup (\overline{A} - A^\circ)&&\text{($\because$ (2)より)}\\
                            &= \overline{A}&&\text{($\because$ (1)より$A^\circ \subset \overline{A}$)} 
\end{align*}
次に, 問1-2\ (2)より, $(A^\circ)^c = \overline{A^c}$ が成り立つから, $(A^e)^c = ((A^c)^\circ)^c = \overline{(A^c)^c} = \overline{A}$ が成り立つ. これより, $A^e \cup \overline{A} = X$ より, 
$\overline{A} = A^\circ \cup \partial A$ とから, $A^\circ \cup \partial A \cup A^e = X$ が成り立つ.
\item
まず, $A^\circ \cup \partial A$ だが, これは(2)より明らかに $A^\circ \cup \partial A = \emptyset$ が成り立つ.\\
次に $A^\circ \cap A^e = \emptyset$ を背理法により示す.\\
$A^\circ \cap A^e$ の元の存在を仮定すると, (1)より $x \in A^\circ \cap A^e \implies x \in \overline{A} \cap A^e$ が成り立つが, (3)より $(A^e)^c = \overline{A}$ であるから, $\overline{A} \cap A^e = \emptyset$
で矛盾. よって, $A^\circ \cap A^e = \emptyset$ が成り立つ.\\
最後に $\partial A \cap A^e = \emptyset$ を背理法により示す.\\
$\partial A \cap A^e$ の元の存在を仮定すると, (2)より $x \in \partial A \cap A^e \implies x \in \overline{A} \cap A^e$ が成り立ち, 先と同様にして矛盾. よって, $\partial A \cap A^e = \emptyset$ が成り立つ.\\
以上より, $A^\circ, \partial A, A^e$ はそれぞれ互いに素である.
}
\end{document}

