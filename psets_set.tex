\pgsc{集合}{darkolivegreen}{set}

\begin{nmprob}
ZCの公理における空集合. 数理論理学との連係はまだ無理(勉強中)で簡単な略した自己流のものに留める.
\pbenum[emptyset]{
\item $\text{emp} (x) \iff \forall z,\ z\notin x$ と定義する. $\text{emp} (x) \land \text{emp} (y) \implies x = y$ を示せ.
\item $\exists y,\ \text{emp} (y)$ を示せ.
}{
\item 以下の証明図(省略済み)より, $\text{emp} (x) \land \text{emp} (y) \implies x = y$ が成り立つ.
ここでは, 外延性の公理「 $\forall x \forall y((\forall z,\ z \in x \leftrightarrow z \in y) \to x = y)$ 」を用いている.
\begin{prooftree}
    \AxiomC{$[\text{emp}(x) \land \text{emp}(y)]_1$}
    \UnaryInfC{$\forall z, z\notin x$}
    \UnaryInfC{$s \notin x$}
    \UnaryInfC{$\lnot (s \in x)$}
    \AxiomC{$[s \in x]_2$}
    \BinaryInfC{$\bot$}
    \UnaryInfC{$s \in y$}
    \RightLabel{{\scriptsize 2}}
    \UnaryInfC{$s \in x \to s \in y$}
    \AxiomC{$[\text{emp}(x) \land \text{emp}(y)]_1$}
    \UnaryInfC{$\forall z, z\notin y$}
    \UnaryInfC{$s \notin y$}
    \UnaryInfC{$\lnot (s \in y)$}
    \AxiomC{$[s \in y]_3$}
    \BinaryInfC{$\bot$}
    \UnaryInfC{$s \in x$}
    \RightLabel{{\scriptsize 3}}
    \UnaryInfC{$s \in y \to s \in x$}
    \BinaryInfC{$s \in x \leftrightarrow s \in y$}
    \UnaryInfC{$\forall z,\ z \in x \leftrightarrow z \in y$}
    \AxiomC{}
    \UnaryInfC{$\forall x \forall y((\forall z,\ z \in x \leftrightarrow z \in y) \to x = y)$}
    \UnaryInfC{$\forall y((\forall z,\ z \in x \leftrightarrow z \in y) \to x = y)$}
    \UnaryInfC{$(\forall z,\ z \in x \leftrightarrow z \in y) \to x = y$}
    \BinaryInfC{$x = y$}
    \RightLabel{{\scriptsize 1}}
    \UnaryInfC{$(\text{emp}(x) \land \text{emp}(y)) \to x = y$}
\end{prooftree}
\item 以下の証明図より $\exists y,\ \text{emp}(y)$ となる.
分出公理「 $\phi$ は $x, z$ 以外の自由変数があれば, $w_1, \ldots w_n$ であり, $y$ を自由変数として含まない論理式のとき, $\forall z \forall w_1, \ldots w_n \exists y \forall x (x \in y \leftrightarrow (x \in z \land \phi))$ 」
を用いている.
\begin{prooftree}
    \AxiomC{$[\forall x(x \in y_0 \leftrightarrow (x \in z \land x \neq x))]_1$}
    \UnaryInfC{$x \in y_0 \leftrightarrow (x \in z \land x \neq x)$}
    %\BinaryInfC{$\forall x(x \in y \leftrightarrow (x \in z \land x \neq x))$}
    \UnaryInfC{$x \in y_0 \to (x \in z \land x \neq x)$}
    \AxiomC{$[x \in y_0]_2$}
    \BinaryInfC{$x \in z \land x \neq x$}
    \UnaryInfC{$x \neq x$}
    \AxiomC{}
    \UnaryInfC{$x = x$}
    \BinaryInfC{$\bot$}
    \RightLabel{{\scriptsize 2}}
    \UnaryInfC{$x \notin y_0$}
    \UnaryInfC{$\forall x, x \notin y_0$}
    \UnaryInfC{$\exists y \forall x, x \notin y$}
    \AxiomC{}
    \UnaryInfC{$\forall z\exists y \forall x(x \in y \leftrightarrow (x \in z \land x \neq x))$}
    \UnaryInfC{$\exists y \forall x(x \in y \leftrightarrow (x \in z \land x \neq x))$}
    \RightLabel{{\scriptsize 1}}
    \BinaryInfC{$\exists y \forall x, x \notin y$}
    \UnaryInfC{$\vdots$}
    \UnaryInfC{$\exists y, \forall z, z \notin y$}
    \UnaryInfC{$\exists y,\ \text{emp}(y)$}
\end{prooftree}
}
\hspace{-3zw}{\color{forestgreen}●●\ まとめ\ ●●}\\
(1), (2) より $\text{emp}(x)$ となる $x$ がただ一つ存在する. この唯一の $x$ を $\emptyset$ で表し, 空集合とよぶ.
よって, $\forall z, z \notin \emptyset$ が成り立つ.
\end{nmprob}



\begin{nmprob}
分出公理とクラス. 数理論理学との連係がまだ無理なため, この問題からはより素朴的な議論に留める. 後に編集可能性あり.
\pbenum{
\item $\forall x,\ x\in A \leftrightarrow \phi(x)$ を満たす集合 $A$ が存在すれば, それは一意に定まることを示せ.
\item (1) における $A$ を以後, $\{ x\ |\ \phi (x)\}$ で表す\footnote[1]{(1) と同様に, $\forall x,\ x \in A \leftrightarrow x \in z\land \phi(x)$ の場合は $\{ x\ |\ x \in z \land \phi(x) \}$ の表記も使用する.
また, $\{ x \in z\ |\ \phi(x) \}$ は $\{ x\ |\ x \in z \land \phi(x) \}$ の略記とする. }
が, (1) において集合 $A$ が存在しない場合にもこの表記を許すこととし, クラスとよぶことにする.
$A$ が存在すれば, クラス $\{ x\ |\ \phi (x)\}$ は集合であり, $A$ が存在しなければ, $\{ x\ |\ \phi (x)\}$ は真クラスであるとよぶ. $\{ x\ |\ x \neq x\}$ が集合であることを示せ.
\item $\{ x\ |\ x = x\}$ は真クラスであることを示せ.
}{
\item $(\forall x,\ x\in A \leftrightarrow \phi(x)) \land (\forall x,\ x\in B \leftrightarrow \phi(x)) \to A = B$ を示す.
\begin{prooftree}
    \AxiomC{$[\forall x,\ x\in A \leftrightarrow \phi(x)) \land (\forall x,\ x\in B \leftrightarrow \phi(x))]_1$}
    \UnaryInfC{$\forall x,\ x\in A \leftrightarrow \phi(x)$}
    \UnaryInfC{$s\in A \leftrightarrow \phi(s)$}
    \UnaryInfC{$\vdots$}
    \AxiomC{$[\forall x,\ x\in A \leftrightarrow \phi(x)) \land (\forall x,\ x\in B \leftrightarrow \phi(x))]_1$}
    \UnaryInfC{$\forall x,\ x\in B \leftrightarrow \phi(x)$}
    \UnaryInfC{$s\in B \leftrightarrow \phi(s)$}
    \UnaryInfC{$\vdots$}
    \BinaryInfC{$s \in A \leftrightarrow s \in B$}
    \UnaryInfC{$\forall z,\ z \in A \leftrightarrow z \in B$}
    \RightLabel{{\scriptsize 外延性の公理を用いる}}
    \UnaryInfC{$A = B$}
    \RightLabel{{\scriptsize 1}}
    \UnaryInfC{$(\forall x,\ x\in A \leftrightarrow \phi(x)) \land (\forall x,\ x\in B \leftrightarrow \phi(x)) \to A = B$}
\end{prooftree}
\item この問題以降, 素朴的な議論で示すこととする.\\
問題\ref{sec:set}-\ref{pname:emptyset}\ より, $\forall x,\ x \notin \emptyset$ より, $\forall x,\ x \in \emptyset \leftrightarrow x \neq x$ が成り立つから,
$\{ x\ |\ x \neq x\}$ は集合である. さらに外延性の公理による空集合の一意性より, $\{ x\ |\ x \neq x\} = \emptyset$ となる.
\item 
まず, クラス $\{x \in z\ |\ x \notin x\}$ を考える. 分出公理より, このクラスは集合である(また, 外延性の公理より一意でもある) $R = \{x \in z\ |\ x \notin x\}$ とすると,
$R \in R \leftrightarrow R \in z \land R \notin R$ \footnote[2]{これより, ラッセルの逆理を回避が可能となる. $R \in z$ を仮定しない限り, $R \notin R$ から $R \in R$ が言えないからである.}
となる. よって, $R \in z$ と仮定すると, $R \in R \land R \notin R$ が成り立ち矛盾するので $R \notin z$ となる.
これより, 分出公理の $z$ の任意性とから, $\forall z \exists R,\ R \notin z$ すなわち $\lnot (\exists z \forall R,\ R \in z)$ \footnote[3]{素朴的には, すべての集合を含む集合は存在しないと主張している.}が成り立つ.\\
さて, $\{ x\ |\ x = x\}$ が集合と仮定すると, $\forall x,\ x \in A \leftrightarrow x = x$ となる集合 $A$ が存在する.
よって, $x = x \to x \in A$ において, $x = x$ は常に成り立つこととから, $x \in A$ となるから, $\exists A\forall x,\ x \in A$ が成り立つが,
これは $\lnot (\exists z \forall R,\ R \in z)$ に矛盾する. よって, $\{ x\ |\ x = x\}$ は真クラスである.
}
\end{nmprob}



\begin{nmprob}
Kuratowski の順序対を理解しよう.
\pbenum{
\item $\langle a, b\rangle$ を $\langle a, b\rangle \coloneqq \{ \{a\}, \{a, b\} \}$ と定めるとき,
\begin{align*}
    \langle a, b\rangle = \langle c, d\rangle \iff (a = c) \land (b = d)
\end{align*}
となることを示せ. (補足: 外延的記法において, 同一の元を重複して書くことは禁じられていない. 同じものをいくつ書いてもその効果はただ1つだけ書いたのと同じものとしている)
\item 順序対を拡張して $\bm{n}$\textbf{-対}を次のように定義する.
\begin{align*}
    \langle a_1 \rangle \coloneqq a_1,\hspace{3zw} \langle a_1, \cdots, a_{n-1}, a_n \rangle \coloneqq \langle \langle a_1, \cdots, a_{n-1} \rangle, a_n \rangle 
\end{align*}
このとき,
\begin{align*}
    \langle a_1, \cdots, a_n \rangle = \langle b_1, \cdots, b_n \rangle \iff a_1 = b_1 \land \cdots a_n = b_n
\end{align*}
が成り立つことを示せ.
}{
\item $\impliedby$ は明らか. $\implies$ は背理法で示す.\\
まず, $a \neq c$ と仮定すると, $\{a\} \neq \{c\}$ より, $\{a\} = \{c, d\}$ とならなくてはならない. このとき, $a = c = d$ となり, $a \neq c$ に矛盾. よって $a = c$ でなくてはならない.\\
次に $b \neq d$ を仮定する. $a = c$ より $\langle a, b\rangle = \langle c, d\rangle \implies \{ \{a\}, \{a, b\} \} = \{ \{a\}, \{a, d\} \}$ が成り立つ. ここで,
\begin{enumerate}
    \item $a = b$ のとき\\
    $\{ \{a\}, \{a, b\} \} = \{\{a\}\}$ より, $\{a, d\} = \{a\}$ となるから, $d = a = b$ で $b \neq d$ に矛盾. 
    \item $a \neq b$ のとき\\
    $\{a, d\} = \{a\}$ または $\{a, d\} = \{a, b\}$ でなくてはならない. ここで $\{a, d\} = \{a, b\}$ とすると, $b = d$ で矛盾. また, $\{a, d\} = \{a\}$ とすると, $d = a$ となり, 今度は
    $\{ \{a\}, \{a, d\} \} = \{\{a\}\}$ となり, $\{a, b\} = \{a\}$ より, $b = a = d$ より矛盾.
\end{enumerate}
以上より, $b = d$ より, $a = c \land b = d$ が成り立つ.
\item
\begin{enumerate}
    \item $\implies$\\
    帰納法により示す.
    $n = 1$ のときは明らか.\\
    次に $\langle a_1, \cdots, a_k \rangle = \langle b_1, \cdots, b_k \rangle \implies a_1 = b_1 \land \cdots a_k = b_k$ が成り立つと仮定すると,
    \begin{align*}
        \langle a_1, \cdots, a_{k+1} \rangle = \langle b_1, \cdots, b_{k+1} \rangle &\implies \langle \langle a_1, \cdots, a_k \rangle, a_{k+1} \rangle = \langle \langle b_1, \cdots, b_k \rangle, b_{k+1} \rangle \\
        &\implies \langle a_1, \cdots, a_k \rangle = \langle b_1, \cdots, b_k \rangle \land a_{k+1} = b_{k+1}\\
        &\implies a_1 = b_1 \land \cdots a_k = b_k \land a_{k+1} = b_{k+1}
    \end{align*}
    が成り立つことから, $\langle a_1, \cdots, a_n \rangle = \langle b_1, \cdots, b_n \rangle \implies a_1 = b_1 \land \cdots a_n = b_n$ が成り立つ.\\
    (集合の集合を認めれば, (1)において元が集合でも問題ない...はず)
    \item $\impliedby$\\
    上の議論を逆にたどればよい.
\end{enumerate}
以上より, $\langle a_1, \cdots, a_n \rangle = \langle b_1, \cdots, b_n \rangle \iff a_1 = b_1 \land \cdots a_n = b_n$ が成り立つ.
}
\end{nmprob}



\begin{nmprob}
二項関係に関する用語を理解する.
\pbenumex[brelation]{
その元が全て順序対であるような集合を二項関係という. 二項関係 $R$ の定義域 $\text{dom} (R)$ および値域 $\text{rng} (R)$ を次のように定義する. (このとき, 明らかに $R \subset \text{dom} (R) \times \text{rng} (R)$ である)
\begin{align*}
    \text{dom} (R) &\coloneqq \{ x\ |\ \exists y,\ \langle x, y\rangle \in R\}\\
    \text{rng} (R) &\coloneqq \{ y\ |\ \exists x,\ \langle x, y\rangle \in R\}
\end{align*}
また, $R$ による集合 $X$ の像 $R[X]$ を
\begin{align*}
    R[X] \coloneqq \{ y\ |\ \exists x\in X,\ \langle x, y\rangle \in R\}
\end{align*}
によって定義する(ここで, $X$ は $\text{dom} (R)$ の部分集合とは限らない)\\
このとき, 次を証明せよ. ただし, $\text{dom} (R) = A,\ \text{rng} (R) \subset B$ として, $X, X_1 ,X_2$ はそれぞれ $A$ の部分集合とする.
}{
\item $X_1 \subset X_2 \implies R[X_1] \subset R[X_2]$
\item $R[X_1 \cup X_2] = R[X_1] \cup R[X_2]$
\item $R[X_1 \cap X_2] \subset R[X_1] \cap R[X_2]$
\item $R[A - X] \supset R[A] - R[X]$
}{
\item $y \in R[X_1]$ とすると, $\langle x, y\rangle \in R$ となる $x \in X_1$ が存在する. 今, $X_1 \subset X_2$
より $x$ は $x \in X_2$ でもあるから, $y \in R[X_2]$ となる. よって, $R[X_1] \subset R[X_2]$ が成り立つ.
\item
\begin{enumerate}
\item $R[X_1 \cup X_2] \subset R[X_1] \cup R[X_2]$ を示す.\\
$y \in R[X_1 \cup X_2]$ とすると, $\langle x, y\rangle \in R$ となる $x \in X_1 \cup X_2$ が存在する.
この $x$ に関して\footnote[1]{$x$ は複数存在する可能性があるが 1 つについて示せば十分であることに注意. 以後の問題も同様.}, $x \in X_1$ のときは $y \in R[X_1]$, $x \in X_2$ のときは $y \in R[X_2]$ となるので, $y \in R[X_1] \cup R[X_2]$ より, $R[X_1 \cup X_2] \subset R[X_1] \cup R[X_2]$ が成り立つ.
\item $R[X_1 \cup X_2] \supset R[X_1] \cup R[X_2]$ を示す.\\
$y \in R[X_1] \cup R[X_2]$ とすると, $\langle x, y\rangle \in R$ となる $x \in X_1$ が存在する, または $\langle x, y\rangle \in R$ となる $x \in X_2$ が存在する. いずれの場合も $x \in X_1 \cup X_2$ となるので
$y \in R[X_1 \cup X_2]$ となるから, $R[X_1 \cup X_2] \supset R[X_1] \cup R[X_2]$
\end{enumerate}
以上より, $R[X_1 \cup X_2] = R[X_1] \cup R[X_2]$ が成り立つ.
\item $y \in R[X_1 \cap X_2]$ と仮定すると, $\langle x, y\rangle \in R$ となる $x \in X_1 \cap X_2$ が存在する. この $x$ に関して $x \in X_1$ かつ $x \in X_2$ であるから, $y \in R[X_1] \cap R[X_2]$ となる.
よって $R[X_1 \cap X_2] \subset R[X_1] \cap R[X_2]$ となる. \\
ちなみに, $X_1 \cap X_2 = \emptyset$ かつ $R[X_1] \cap R[X_2] \neq \emptyset$ となるように $X_1, X_2$ を上手く設定すると, $R[X_1 \cap X_2] \not\supset R[X_1] \cap R[X_2]$ となる例が見つかる.
\item $y \in R[A] - R[X]$ とすると, $y \in R[A]$ かつ $y \notin R[X]$ となる. $y \in R[A]$ より $\langle x, y\rangle \in R$ となる $x \in A$ が存在する. ここで $x \in X$ と仮定すると, $y \in R[X]$ となってしまうので, $x \notin X$ となる.
よって $x \in A - X$ より, $y \in R[A - X]$ となる. 以上より, $R[A - X] \supset R[A] - R[X]$ が成り立つ.\\
ちなみに $A - X \neq \emptyset$ かつ $R[A] - R[X] = \emptyset$ となるように $X$ を設定すると, $R[A - X] \not\subset R[A] - R[X]$ となる例が見つかる.
}
\end{nmprob}



\begin{nmprob}
逆関係, 合成関係, 写像に関する用語を理解する.
\pbenum[finv]{
\item 関係 $R \subset A \times B$ が与えられたとき, その逆関係 $R^{-1}$ を $R^{-1} \coloneqq \{ \langle y, x\rangle\ |\ \langle x, y\rangle \in R\}$ により定める.
$R^{-1}[Y_1 \cap Y_2] \subset R^{-1}[Y_1] \cap R^{-1}[Y_2]$ を示せ. ただし, $Y_1, Y_2 \subset B$ とする.
\item 2つの関係 $R, S$ が与えられたとき, その合成関係 $R \circ S$ を $R \circ S \coloneqq \{ \langle x, z\rangle \ |\ \exists y,\ \langle x, y\rangle \in S \land \langle y, z\rangle \in R\}$ により定める.
$(R \circ S)^{-1} = S^{-1} \circ R^{-1}$ を示せ.
\item 関係 $R$ が $\langle x, y\rangle \in R \land \langle x, z\rangle \in R \implies y = z$ をみたすとき, $R$ を写像とよぶ.
特に $\text{dom} (R) = A,\ \text{rng} (R) \subset B $ のときは, $R$ は $A$ から $B$ への写像\footnote[1]{$A$ を始集合, $B$ を終集合とよぶ.}と呼ぶ. $R$ が写像のとき, (1) において $R^{-1}[Y_1 \cap Y_2] \supset R^{-1}[Y_1] \cap R^{-1}[Y_2]$ を示せ(つまり(1)とから $R$ が写像であれば $R^{-1}[Y_1 \cap Y_2] = R^{-1}[Y_1] \cap R^{-1}[Y_2]$ が成り立つ)
\item $f$ が $A$ から $B$ への写像であるとき, $A (= \text{dom}(f))$ の元 $x$ に関して $\langle x, y\rangle \in f$ となる $y \in B$ がただ一つ存在する. これを $f$ による $x$ の像とよび, $f(x)$ と表す.
ここで, $(\forall x, x^{'} \in A)[x \neq x^{'} \implies f(x) \neq f(x^{'})]$ が成り立つとき $f$ を単射とよぶ.\footnote[2]{$f$ が単射であることを示す際に, 対偶である $(\forall x, x^{'} \in A)[f(x) = f(x^{'}) \implies x = x^{'}]$ を示す場合が非常に多い.\\ また, $f$ が写像であるもとで, $f$ が単射であることと逆関係 $f^{-1}$ が写像であることは本質的に同じ.}
また, $\text{rng} (f) = B$ のとき, $f$ を全射といい, 全射かつ単射のとき全単射という. $f$ が $A$ から $B$ への全単射のとき, 逆関係 $f^{-1}$ が $B$ から $A$ への全単射の写像\footnote[3]{これを $f$ の逆写像とよぶ}であることを示せ.}{
\item $R^{-1} \subset B \times A$ より, 問\ref{sec:set}-\ref{pname:brelation} (3) において記号を置き換えればすぐに示される.
\footnote[4]{問\ref{sec:set}-\ref{pname:brelation}の他の3つも同様である.}
\item $\langle z, x\rangle \in (R \circ S)^{-1}$ とすると, $\langle x, z\rangle \in R \circ S$ より, $\langle x, y\rangle \in S \land \langle y, z\rangle \in R$ となる $y$ が存在する.
この $y$ に対して, $\langle z, y\rangle \in R^{-1}$ かつ $\langle y, x\rangle \in S^{-1}$ より, $\langle z, x\rangle \in S^{-1} \circ R^{-1}$ となる. よって $(R \circ S)^{-1} \subset S^{-1} \circ R^{-1}$ となる.
また, 同様の議論を逆にたどることにより, $S^{-1} \circ R^{-1} \subset (R \circ S)^{-1}$ となる. 以上より, $(R \circ S)^{-1} = S^{-1} \circ R^{-1}$ が成り立つ.
\item $x \in R^{-1}[Y_1] \cap R^{-1}[Y_2]$ とすると, $(\exists y\in Y_1,\ \langle y, x\rangle \in R^{-1})\land (\exists y^{'}\in Y_2,\ \langle y^{'}, x\rangle \in R^{-1})$ が成り立つ. $y, y^{'}$ に関して,
$\langle x, y\rangle \in R$ かつ $\langle x, y^{'}\rangle \in R$ が成り立つから, $R$ が写像であることより $y = y^{'}$ となる. よって $\exists y \in Y_1 \cap Y_2,\ \langle y, x\rangle \in R^{-1}$ が成り立つから, $x \in R^{-1}[Y_1 \cap Y_2]$ より, $R^{-1}[Y_1 \cap Y_2] \supset R^{-1}[Y_1] \cap R^{-1}[Y_2]$ となる.
\item $\text{dom} (f^{-1}) = B,\ \text{rng} (f^{-1}) = A$ であることと $f^{-1}$ が写像であること, および $f^{-1}$ が単射であることを示せばよい.
まず,
\begin{align*}
    \text{dom} (f^{-1}) &= \{ y\ |\ \exists x,\ \langle y, x\rangle \in f^{-1}\}\\
    &= \{y\ |\ \exists x,\ \langle x, y\rangle \in f \}\\
    &= \text{rng} (f)\\
    &= B\hspace{1zw}\text{($\because$\ $f$ が全射より)}
\end{align*}
より, $\text{dom} (f^{-1}) = B$ となり, 同様にして $\text{rng} (f^{-1}) = A$ が成り立つ.\\
次に, $\langle y, x\rangle \in f^{-1}$ かつ $\langle y, x^{'} \rangle \in f^{-1}$ とすると, $\langle x, y\rangle \in f$ かつ $\langle x^{'}, y\rangle \in f$ であり,  $f$ が写像であることから $y = f(x) = f(x^{'})$ となる.
ここで $f$ は単射であることから, $x = x^{'}$ となるから $f^{-1}$ は写像となる.\\
最後に $f^{-1}$ が写像であることから, $\langle y, f^{-1}(y) \rangle \in f^{-1}$ かつ $\langle y^{'}, f^{-1}(y^{'}) \rangle \in f^{-1}$ より, $\langle f^{-1}(y), y\rangle \in f$ かつ $\langle f^{-1}(y^{'}), y^{'} \rangle \in f$ となる.
ここで, $f^{-1}(y) = f^{-1}(y^{'})$ と仮定すると, $f$ が写像であることから, $y = y^{'}$ となるから $f^{-1}$ は単射となる.\\
以上より, $f^{-1}$ は $B$ から $A$ への全単射となる.
} 
\end{nmprob}



\begin{nmprob}
合成写像と二項関係.
\pbenum[fog]{
\item 写像 $f: A\to B,\ g: B\to C$ の合成関係 $g\circ f$ は $A$ から $C$ への写像であることを示せ.
\item 写像 $f: A\to B$ と $g: B\to C$ がそれぞれ全単射のとき, $g\circ f$ は $A$ から $C$ への全単射となることを示せ.
\item $X$ から $X$ への写像で, 任意の $x \in X$ に対して, その像が $x$ となるものを恒等写像といい $I_X$ で表す.
$I_X$ は全単射であることを示せ.
}{
\item
\begin{enumerate}
\item $\text{dom}(g\circ f) = A$ を示す.\\
$x \in \text{dom}(g\circ f)$ とすると, $\langle x, z \rangle \in g\circ f$ となる $z$ が存在するから, さらに
$\langle x, y\rangle \in f$ となる $y$ が存在する. よって, $x \in \text{dom}(f) = A$ となるから, $\text{dom}(g\circ f) \subset A$ となる.\\
また, $x \in A$ とすると, $A = \text{dom}(f)$ より, $\langle x, y\rangle \in f$ となる $y$ が存在するから, 同様に $\langle y, z\rangle \in g$ となる $z$ が存在する.
これより, $\langle x, z\rangle \in g\circ f$ より, $x \in \text{dom}(g\circ f)$ より, $A \subset \text{dom}(g\circ f)$ となる.\\
以上より, $A = \text{dom}(g\circ f)$ となる.
\item $\text{rng}(g\circ f) \subset C$ を示す.\\
$z \in \text{rng}(g\circ f)$ とすると, $\langle x, z\rangle \in g\circ f$ となる $x$ が存在するから, $\langle y, z\rangle \in g$ となる $y$ が存在する. よって $z \in \text{rng}(g) \subset C$ より, $z \in C$ となる.
よって, $\text{rng}(g\circ f) \subset C$ となる.
\item $\langle x, z\rangle \in g\circ f \land \langle x, z^{'}\rangle \in g\circ f \implies z = z^{'}$ を示す.\\
$\langle x, z\rangle \in g\circ f \land \langle x, z^{'}\rangle \in g\circ f$ とすると, $\langle x, y\rangle \in f \land \langle y, z\rangle \in g$ となる $y$ と
$\langle x, y^{'}\rangle \in f \land \langle y^{'}, z^{'}\rangle \in g$ となる $y^{'}$ が存在する. ここで $f$ が写像より, $y = y^{'}$ であり, $g$ も写像であるから $z = z^{'}$ となる.
\end{enumerate}
以上より, $g\circ f$ は $A$ から $C$ への写像である.\footnote[1]{合成関係が写像となっているときは, 合成写像とよぶ.}
\item
(1)より, $g\circ f$ は $A$ から $C$ への写像であるので, $g\circ f$ が全単射であることを示す.
\begin{enumerate}
\item 全射性を示す.\\
$z \in C$ とすると, $g$ が全射より, $\langle y, z\rangle \in g$ となる $y \in B$ が存在する. ここで, $f$ も全射より, この $y$ に対して, $\langle x, y\rangle \in f$ となる $x \in A$ が存在する.
これより, $\langle x, z\rangle \in g \circ f$ となる $x$ が存在することから $z \in \text{rng}(g \circ f)$ となり, $\text{rng}(g \circ f) = C$ となる. よって, $g\circ f$ は全射となる.
\item 単射性を示す.\\
$(g\circ f)(x) = (g\circ f)(x^{'})$ とすると, $g$ が単射より, $f(x) = f(x^{'})$ であり, $f$ も単射であることから $x = x^{'}$ となる. よって, $g\circ f$ は単射である.
\end{enumerate}
以上より, $g\circ f$ は全単射である.
\item
\begin{enumerate}
\item 全射性を示す.\\
$x \in X$ とすると, $I_X$ が写像より, $\langle x, y\rangle \in I_X$ となる $y$ がただ一つ存在し, 恒等写像の定義より $y = x$ である.
よって, $\forall x \in X,\ \langle x, x\rangle \in I_X$ となるから, $X \subset \text{rng}(I_X)$ より, $X = \text{rng}(I_X)$ で $I_X$ は全射である.
\item 単射性を示す.\\
任意の $x, x^{'} \in X$ に対して, $I_X(x) = I_X(x^{'})$ とすると, $I_X(x) = x,\ I_X(x^{'}) = x^{'}$ より, $x = x^{'}$ となるから, $I_X$ は単射である.
\end{enumerate}
以上より, $I_X$ は全単射である.
}
\end{nmprob}



\begin{nmprob}
逆写像と二項関係.
\pbenum[comp_inv]{
\item 写像 $f: A\to B$ と $g: B\to C$ の合成写像 $g\circ f: A\to C$ が全単射であるとき, $g$ は全射であり, $f$ は単射であることを示せ.
\item 写像 $f: A\to B$ が全単射のとき, $f^{-1} \circ f = I_A,\ f \circ f^{-1} = I_B$ を示せ.
\item 写像 $f: A\to B,\ g: B\to C,\ h: C\to D$ に対して $h\circ (g\circ f) = (h \circ g)\circ f$ を示せ.
\item 写像 $f: A\to B,\ g: B\to A$ に対して, $f\circ g = I_B,\ g\circ f= I_A$ であるとき, $f, g$ はそれぞれ全単射であり,
$g = f^{-1},\ f = g^{-1}$ となることを示せ.
}{
\item
\begin{enumerate}
\item $g$ が全射であることを示す.\\
$C \subset \text{rng}(g)$ を示せばよい. $z \in C$ とすると, $g \circ f$ が全単射であるから, $\langle x, z\rangle \in g \circ f$ となる $x \in A$ が存在する.
そして, この $x, z$ に対して, $\langle x, y\rangle \in f \land \langle y, z\rangle \in g$ となる $y$ が存在する( $f$ が写像より $y = f(x)$ となる)
よって $z \in C \implies \exists y, \langle y, z\rangle \in g$ より, $z \in \text{rng}(g)$ となり, $g$ は全射となる.
\item $f$ が単射であることを示す.\\
任意の $x, x^{'} \in A$ に対して, $f(x) = f(x^{'})$ と仮定すると, $g\circ f(x) = g\circ f(x^{'})$ となる. ここで, $g \circ f$ は単射であるから, $x = x^{'}$ となる. よって $f$ は単射となる.
\end{enumerate}
\item
まず, $f^{-1}\circ f\subset I_A$ を示す.\\
$\langle x, y\rangle \in f^{-1}\circ f$ とすると, 問\ref{sec:set}-\ref{pname:fog}\ (1) より, $f^{-1}\circ f$ は $A$ から $A$ への写像であるから, $x, y\in A (= \text{dom}(g\circ f))$ であり,
また, $\langle x, z\rangle \in f\land \langle z, y\rangle \in f^{-1}$ となる $z$ が存在し, $\langle z, x\rangle \in f^{-1} \land \langle z, y\rangle \in f^{-1}$ となる. ここで, 問\ref{sec:set}-\ref{pname:finv} (4)より,
$f^{-1}$ は $B$ から $A$ への写像であるから, $x = y$ となる. よって, $\langle x, y\rangle \in f^{-1}\circ f$ ならば $x = y$ より, $\langle x, y\rangle \in I_A$ となるので $f^{-1}\circ f \subset I_A$ となる.\\
次に, $I_A \subset f^{-1}\circ f$ を示す.\\
$\langle x, y\rangle \in I_A$ とすると, $x \in \text{dom}(I_A) = A$ より, $\langle x, f(x)\rangle \in f \land \langle f(x), x\rangle \in f^{-1}$ となるから $\langle x, x\rangle \in f^{-1}\circ f$ となる.
また, $I_A$ は恒等写像であるから $x = y$ となる. よって, $\langle x, y\rangle \in f^{-1}\circ f$ となるので $I_A \subset f^{-1}\circ f$ となる.\\
以上より, $f^{-1}\circ f = I_A$ となる. $f\circ f^{-1} = I_B$ も同様に示せる.
\item
$\langle x, w\rangle \in h\circ (g\circ f)$ とすると, $\langle x, z\rangle \in g \circ f,\ \langle z, w\rangle \in h$ となる $z$ が存在し,
$\langle x, y\rangle \in f,\ \langle y, z\rangle \in g$ となる $y$ が存在する. よって, $\langle y, w\rangle \in h\circ g$ となり,
$\langle x, w\rangle \in (h \circ g) \circ f$ となる. よって, $h \circ (g \circ f) \subset (h \circ g)\circ f$ となる.
同様に, $h \circ (g \circ f) \supset (h \circ g)\circ f$ となるので, $h \circ (g \circ f) = (h \circ g)\circ f$ となる.\footnote[1]{この結合法則は一般の二項関係においても成り立つ.}
\item まず, 問\ref{sec:set}-\ref{pname:fog}(3) より, $I_A, I_B$ は共に全単射である. よって (1) より $f,\ g$ は共に全単射である.\\
次に, (2)より, $f\circ g = I_B$ ならば $f^{-1}\circ (f\circ g) = f^{-1}\circ I_B$ から $g = f^{-1}$ となる. 同様に, $f = g^{-1}$ となる.
}
\end{nmprob}



\begin{nmprob}
順序関係と順序集合
\makeatletter\tagsleft@true\makeatother
\pbenumex[ordr]{
$R$ がある集合 $A$ 上の二項関係であるとき, $\langle x, y\rangle \in R$ であることを以後 $xRy$ と表す.
}{
\item $\leq$ が 集合 $A$ 上の二項関係であり, 任意の $x, y, z \in A$ に対して, 次の $(O1) \sim (O3)$ を満たすとき, $\leq$ は $A$ 上の順序(関係)であるといい,
集合 $A$ 上に一つの順序関係 $\leq$ が定められているとき, $\langle A, \leq \rangle$ を順序集合という.
\begin{align}
    x \leq x & & &(\text{反射律})\tag{O1}\\
    x \leq y \land y \leq x & \implies x = y & &(\text{反対称律})\tag{O2}\\
    x \leq y \land y \leq z & \implies x \leq z & &(\text{推移律})\tag{O3}
\end{align}
$x \leq y \land x \neq y$ が成り立つとき, またそのときに限り $x < y$ と二項関係 $<$ を定めれば, 次の (O4), (O5) が成り立つことを示せ.
\begin{align}
    x < y & \implies y \nless x \coloneqq \lnot (y < x) \tag{O4}\\
    x < y \land y < z & \implies x < z \tag{O5}
\end{align}
\item (1) とは逆に (O4), (O5) を満たす二項関係 $<$ が集合 $A$ 上に与えられたとき, $x < y \lor x = y$ が成り立つとき, またその時に限り $x \leq y$ と二項関係 $\leq$ を
定めれば $\leq$ は順序関係になることを示せ.
\item 順序集合 $\langle A, \leq \rangle$ において, 集合 $A$ の任意の二元 $x, y$ に関して $x \leq y$ または $y \leq x$ が成り立つとき, $\leq$ は全順序(関係)といい, $\langle A, \leq \rangle$ は全順序集合という.
$\langle A, \leq \rangle$ が全順序集合のとき, $x \nless y \iff y \leq x$ を示せ. 
}{
\item まず, $x < y$ が成り立つ下で $y < x$ が成り立つと仮定すると, $x \leq y \land y \leq x$ より (O2) から $x = y$ となるが, これは $x \neq y$ に矛盾する. よって (O4) が成り立つ.\\
次に, $x < y \land y < z$ が成り立つと仮定すると, $x \leq y \land y \leq z$ が成り立つから, (O3) より $x \leq z$ が成り立つ. ここで, $x = z$ が成り立つと仮定すると, $z < y \land y < z$ が成り立ち, $y = z$ となるが,
これは $y \neq z$ に矛盾する. よって $x \neq z$ となり, $x < z$ より (O5) が成り立つ.
\item
\begin{enumerate}
\item 反射律の成立\\
$x = x$ が常に成立ことから, $x < x \lor x = x$ も常に成立する. よって, 反射律が成立する.
\item 反対称律の成立\\
$x \leq y \land y \leq x$ が成り立つが $x \neq y$ だと仮定すると, $x < y \land y < x$ が成り立つ. これより, (O5) から $x < x$ となるが, (O4) より $(x < x) \land \lnot(x < x)$ が成り立ち, 矛盾式が導出される. よって $x = y$ より 反対称律が成立する.
\item 推移律の成立\\
$x \leq y \land y \leq z$ が成り立つと仮定すると, $(x < y \lor x = y) \land (y < z \lor y = z)$ が成り立つ. $x < y \land y < z$ の場合は (O5) より $x < z$ より $x \leq z$ となり,
$x < y \land y = z$ の場合は $x < z$ で $x \leq z$ となる. $x = y \land y < z$ の場合も $x < z$ より $x \leq z$ であり, $x = y \land y = z$ の場合は $x = z$ で $x \leq z$ となるから, 推移律が成り立つ.
\end{enumerate}
\item $x \nless y$ とすると, $x \nleq y \lor x = y$ となる. $x \nleq y$ の場合は $\leq$ が全順序より, $y \leq x$ であり, $x = y$ の場合も $y \leq x$ となるから $x \nless y \implies y \leq x$ となる.\\
$y \leq x$ として, $x < y$ が成り立つと仮定する. $x < y$ より, $x \neq y$ であるから $y \leq x$ より $y < x$ が成り立つ. よって $x < y \land y < x$ が成り立つが, これは(1)で述べた通り $x \neq y$ に矛盾する. よって $y \leq x \implies x \nless y$ となる.\\
以上より, $x \nless y \iff y \leq x$ が成り立つ.
}
\makeatletter\tagsleft@false\makeatother
\end{nmprob}



\begin{nmprob}
最大元, 極大元, 部分順序集合, 上界, 上限.
\pbenumex[max_sup]{
$\langle A, \leq \rangle$ を一つの与えられた順序集合とする.
$A$ の空でない部分集合 $M$ を考える. $a,b \in M$ に関して $a \leq b$ が成り立つとき, またその時に限り $a \leq_M b$ と二項関係 $\leq_M$ を定めると, これは明らかに $M$ 上の順序関係となる.
ここで $\langle M, \leq_M \rangle$ を $\langle A, \leq \rangle$ の部分順序集合といい, 誤解が生まれない場合には $\langle M, \leq \rangle$ で表す.

}{
\item $A$ に一つの元 $a$ があって, $\forall x \in A,\ x \leq a$ が成り立つとき, $a$ を $A$ の最大元といい $\max A$ と表す.
最大元が任意の順序集合の台集合\footnote[1]{順序集合 $\langle A, \leq \rangle$ における集合 $A$ をその台集合という. また, 順序関係を明示しなくても誤解が生まれない場合は $A$ のことを順序集合とよぶことがある.}
に存在するとは限らないが, 存在すれば一意的に定まることを示せ.
\item $A$ の元 $a$ に関して, $\lnot (\exists x \in A,\ a < x)$ が成り立つとき, $a$ を $A$ の極大元という.
極大元の存在やその一意性は一般には保証されないが, $\max A$ が存在するとき, $a = \max A$ となることを示せ. 
\item $A$ が全順序集合のとき, $a$ が $A$ の最大限であることと $a$ が $A$ の極大元であることは同値であることを示せ.
\item $A$ の元 $a$ 関して, $\forall x \in M,\ x \leq a$ が成り立つとき, $a$ を $M$ の $A$ における上界という.
$M$ の $A$ における上界全体の集合を $M^*$ とすると, $M^* \neq \emptyset$ のとき, $M$ は $A$ において上に有界という.
$M$ が $A$ において上に有界かつ $\min M^*$ が存在するとき, $\min M^*$ を $M$ の $A$ における上限といい $\sup M$ と表す.
$\sup M$ が存在すれば一意的であることを示せ.
\item $\sup M$ が存在するという下で, $\sup M \in M \iff (\max M (\in M) \text{が存在})$ を示せ.
}{
\item $\max A$ が存在するとし, $a = \max A,\ a^{'} = \max A$ かつ $a \neq a^{'}$ とすると,
$a, a^{'}$ 共に $A$ の最大限より $a \leq a^{'}$ かつ $a^{'} \leq a$ より $a = a^{'}$ となり矛盾する.
よって, $\max A$ が存在すれば, それは一意に定まる.
\item $a = \max A$ とし, $a$ が $A$ の極大元でないとする. $a$ は $A$ の極大元ではないから $a < x$ となる $x \in A$ が存在する.
これより $a \leq x$ かつ $a \neq x$ となるが, $a = \max A$ より $x \leq a$ でもあるから $a = x \land a \neq x$ となり矛盾する.
よって $a = \max A$ は $A$ の極大元となる.
\item
$A$ が全順序集合のとき, 
\begin{align*}
    \forall x \in A,\ x\leq a &\iff \forall x \in A,\ \lnot (a < x)&&\text{($\because$ 問\ref{sec:set}-\ref{pname:ordr}\ (3)より)}\\
    &\iff \lnot (\exists x \in A,\ a < x)
\end{align*}
より, $A$ が全順序集合のとき, $a$ が $A$ の最大限となることと, $A$ の極大元になることは同値である.
\item $\sup M$ が存在すれば $\min M^*$ が存在し, 最小元の一意性より $\sup M$ は一意的に定まる.
\item $a \in A$ に関して 
\begin{empheq}[left={a = \sup M \iff \empheqlbrace}]{alignat=2}
    & \forall x \in M,\ x \leq a && \quad (\because \text{$a$ は $M$ の上界}) \tag{a}\\
    & \forall x^{'} \in M^*,\ a \leq x^{'} && \quad (\because \text{$a$ は上界の最小値}) \tag{b}
\end{empheq}
となるから, $\sup M \in M$ とすると (a) より, $\max M = \sup M$ となる. また, $a = \max M$ とすると, 明らかに (a) が成り立ち, また, $a \in M$ より (b) も成り立つ.
よって, $a = \sup M$ となり, $\sup M = \max M$ より, $\sup M \in M$ となる.
}
\newpage
\hspace{-3zw}{\color{forestgreen}●●\ 最大元, 上限の補足\ ●●}

最大元と上限についてまとめると次の表のようになる(最小元, 下限も同様)
\begin{table}[hbtp]
    \caption{$\max M, \sup M$ の存在}
    \begin{center}
        \begin{tabular}{c|c|c}
            $\max M$ の存在 & $\sup M$ の存在 & 説明 \\ \hline \hline
            存在する & 存在する & $\sup M = \max M,\ \sup M \in M$\\ \hline
            存在する & 存在しない & ありえない.\\ \hline
            存在しない & 存在する & $\sup M \notin M$ となる.\\ \hline
            存在しない & 存在しない & 起こりうる.\\ \hline
        \end{tabular}
    \end{center}
\end{table}

\begin{itemize}
    \item $\max M$ が存在 $\implies$ $\sup M = \max M$
    \item $\max M$ が存在しない $\implies$ $\sup M$ が存在し $\sup M \notin M$ , または $\sup M$ は存在しない.
    \item $\sup M$ が存在し, $\sup M \in M$ $\implies$ $\max M = \sup M$
    \item $\sup M$ が存在し, $\sup M \notin M$ $\implies$ $\max M$ は存在しない.
    \item $\sup M$ が存在しない $\implies$ $\max M$ は存在しない.
\end{itemize}
\end{nmprob}
\setcounter{table}{0}



\begin{nmprob}
順序写像, 順序単射, 順序同型写像, 順序同型
\pbenum{
\item 2 つの順序集合 $\langle A, \leq \rangle$ および $\langle A^{'}, \leq^{'} \rangle$ を考える. $f: A \to A^{'}$ で $a, b \in A$ に対して $a \leq b \implies f(a) \leq^{'} f(b)$ となるとき,
$f$ は $\langle A, \leq \rangle$ から $\langle A^{'}, \leq^{'} \rangle$ への順序写像とよばれる.\footnote[1]{誤解がない生まれない場合には $A$ から $A^{'}$ への順序写像や単に順序写像という場合がある}
$f$ が順序写像で, さらに $a, b \in A$ に対して $f(a) \leq^{'} f(b) \implies a \leq b$ となるとき,
$f$ は単射となることを示せ.\footnote[2]{この $f$ を $\langle A, \leq \rangle$ から $\langle A^{'}, \leq^{'} \rangle$ への順序単射という. これも誤解の生まれない場合には $A$ から $A^{'}$ への順序単射や単に順序単射という場合がある.}
\item $f$ が $A$ から $A^{'}$ への順序単射かつ $f$ が全射のとき, $f$ は $\langle A, \leq \rangle$ から $\langle A^{'}, \leq^{'} \rangle$ への順序同型写像とよばれる.\footnote[3]{注1, 2と同様.}
$f$ が $A$ から $A^{'}$ への順序同型写像のとき, $f^{-1}$ は $A^{'}$ から $A$ への順序同型写像となることを示せ.
\item $\langle A, \leq \rangle$ から $\langle A^{'}, \leq^{'} \rangle$ への順序同型写像が存在するとき, 両者は順序同型であるといい, $\langle A, \leq \rangle \simeq \langle A^{'}, \leq^{'} \rangle$ と表す(または略して $A \simeq A^{'}$ と表す)
$\langle A, \leq \rangle \simeq \langle A, \leq \rangle$ となることを示せ.
\item $\langle A, \leq \rangle \simeq \langle A^{'}, \leq^{'} \rangle \implies \langle A^{'}, \leq^{'} \rangle \simeq \langle A, \leq \rangle$ となることを示せ.
\item $\langle A, \leq \rangle \simeq \langle A^{'}, \leq^{'} \rangle \land \langle A^{'}, \leq^{'} \rangle \simeq \langle A^{''}, \leq^{''} \rangle \implies \langle A, \leq \rangle \simeq \langle A^{''}, \leq^{''} \rangle$ となることを示せ.
}{
\item $f(a) = f(b)$ とすると, $f(a) \leq^{'} f(b)$ かつ $f(b) \leq^{'} f(a)$ となるから, $a \leq b$ かつ $b \leq a$ より $a = b$ となる.
よって, $f$ は単射となる.
\item $f: A \to A^{'}$ は全単射より, \ref{sec:set}-\ref{pname:finv}\ (4) より $f^{-1}$ は $A^{'}$ から $A$ への全単射写像である.
ここで, $a^{'}, b^{'} \in A^{'}$ とすると, $f$ が全射より, $\langle a, a^{'} \rangle \in f,\ \langle b, b^{'} \rangle \in f$ となる $a, b \in A$ が存在する. ここで, $f$ は写像より, $a^{'} = f(a),\ b^{'} = f(b)$ となる.
また, $\langle a^{'}, a\rangle \in f^{-1}$ で $a = f^{-1}(a^{'})$ となり, 同様に $b = f^{-1}(b^{'})$ となる.\\
よって, $a^{'} \leq^{'} b^{'}$ とすると, $f(a) \leq^{'} f(b)$ より, $f$ が順序同型写像であることとから $f^{-1}(a^{'}) \leq f^{-1}(b^{'})$ となり, $f^{-1}$ は順序写像となる.
また, さらに $f^{-1}(a^{'}) \leq f^{-1}(b^{'})$ とすると, $a \leq b$ より, $f$ の順序単射性より, $a^{'} \leq b^{'}$ となり, $f^{-1}$ は順序単射となる.
ここで, $f^{-1}$ は全(単)射であったから $f$ は順序同型写像となる.
\item $I_A$ が $\langle A, \leq \rangle$ から $\langle A, \leq \rangle$ への順序同型写像となる. よって, $A \simeq A^{'}$ となる.
\item $\langle A, \leq \rangle$ から $\langle A^{'}, \leq^{'} \rangle$ への順序同型写像の一つを $f$ とすると, (3) より, $f^{-1}$ は $\langle A^{'}, \leq^{'} \rangle$ から $\langle A, \leq \rangle$ への順序同型写像となる.
よって, $A \simeq A^{'} \implies A^{'} \simeq A$ となる.
\item $A$ から $A^{'}$ への順序同型写像の一つを $f$, $A^{'}$ から $A^{''}$ への順序同型写像の一つを $f^{'}$ とし, $f^{''} = f^{'} \circ f$ とする.
まず, 問\ref{sec:set}-\ref{pname:fog}\ (2) より $f^{''}$ は $A$ から $A^{''}$ への全単射である. 次に $a \leq b$ とすると, $f(a) \leq f(b)$ より $f^{'}(f(a)) \leq f^{'}(f(b))$, すなわち $f^{''}(a) \leq f^{''}(b)$ となり, $f^{''}$ は順序写像となる.
同様に, $f^{''}(a) \leq f^{''}(b)$ とすると, $a \leq b$ となり, $f^{''}$ は順序単射となるから, 全射であることとから $f^{''}$ は $A$ から $A^{''}$ への順序同型写像となる.
よって, $A \simeq A^{'} \land A^{'} \simeq A^{''} \implies A \simeq A^{''}$ となる.
}
\end{nmprob}



\begin{nmprob}
整列集合, 直前直後の元, 切片
\pbenum[w_ordset]{
\item $W$ が順序集合で, その空でない任意の部分集合が最小元をもつとき, $W$ を整列集合という.
$W$ は全順序集合であることを示せ.
\item $A$ を任意の順序集合とし, $a, b \in A$ とする. $a < b \land \lnot (\exists x \in A,\ a < x < b)$ が成り立つとき,
$A$ の中で $b$ は $a$ の直後の元, $a$ は $b$ は直前の元であるという. また, $\{ x \in A\ |\ x < a \}$ を $A$ の $a$ による切片といい, $\text{seg}_A (a)$ で表す.
\textbf{全順序集合} $A$ の中で $b$ が $a$ の\textbf{直前}の元であることと $b = \max \text{seg}_A (a)$ であることが同値であることを示せ.
\footnote[1]{これより, $a$ の直前の元は $a$ に対して一意に定まる. 直後の元も同様.}
\item tmp
}{
\item $W$ の任意の異なる二元 $a, b$ に対して, $\{a , b\} \subset W$ より, $\{a, b\}$ に最小元が存在する.
最小元の一意性(問 \ref{sec:set}-\ref{pname:max_sup}参考)より, $a, b$ のどちらか一方が $\min \{a, b\}$ となる.
ここで, $a$ が最小元のとき, $a \leq b$ で比較可能, また, $b$ が最小元のときも $b \leq a$ となり比較可能である. よって, $W$ は全順序集合となる.
\item
\begin{enumerate}
\item $\implies$\\
$b$ が $a$ の直前の元とすると, $b \in A$ かつ $b < a$ より, $b \in \text{seg}_A (a)$ となる.
また, $x \not\leq b$ となる $x \in \text{seg}_A (a)$ が存在すると仮定すると, 問\ref{sec:set}-\ref{pname:ordr} (3) より, $x > b$ となる.
また, $x \in \text{seg}_A (a)$ より, $x < a$ より, $b < x < a$ となるが, これは $\lnot (\exists x \in A,\ b < x < a)$ に矛盾する.
よって, $\forall x \in \text{seg}_A (a),\ x \leq b$ となり, $b = \max \text{seg}_A (a)$ となる.
\item $\impliedby$\\
$b = \max \text{seg}_A (a)$ とすると, まず, $b \in \text{seg}_A(a)$ より, $b < a$ となる.
次に, $b < x_0 < a$ となる $x_0 \in \text{seg}_A(a)$ が存在すると仮定する. ここで, $\forall x \in \text{seg}_A (a),\ x \leq b$ より,
$b < x_0 \leq b$ となる. $b < x_0$ より, $b \neq x_0$ であるから, $b < x_0 < b$ となるが, これは矛盾. よって, $\lnot (\exists x \in A,\ a < x < b)$ が成り立つ.
よって, $b$ は $a$ の直前の元である.
\end{enumerate}
}
\end{nmprob}



\begin{nmprob}
選択公理を理解する.
\pbenum{
\item $\Lambda$ から $A_\lambda$ への写像を 集合族$(A_\lambda)_{\lambda \in \Lambda}$という. ここで, $\Lambda$ から $A_\lambda$ への写像
$a$ のうち, $a_\lambda \in A_\lambda$ を満たすものの全体を集合族 $(A_\lambda)_{\lambda \in \Lambda}$ の直積といい, $\prod_{\lambda \in \Lambda}A_\lambda$ で
表す. 今 $\prod_{n \in \bm{N}} A_n$ を $(A_1, A_2, \cdots, A_n)$ と表すとき, $( A_1, A_2, \cdots, A_n)$ は $n$-対の性質を持つことを示せ(ほぼ明らか)
\item 選択公理(AC)
\begin{align*}
    \Lambda \neq \emptyset \land \forall \lambda \in \Lambda,\ A_\lambda \neq \emptyset \implies \prod_{\lambda \in \Lambda} A_\lambda \neq \emptyset
\end{align*}
から従属選択公理(DC)\\
\begin{align*}
    \begin{cases}
        A \neq \emptyset\\
        (\forall x \in A)[\exists y \in A,\ \langle x, y\rangle \in R]\hspace{2zw}(R \subset A \times A)
    \end{cases}
    \implies \exists f: \bm{N} \to A,\ \langle f(n), f(n+1)\rangle \in R    
\end{align*}
を示せ.
\item 従属選択公理(DC)から可算選択公理(CC)
\begin{align*}
    \forall n \in \bm{N},\ A_n \neq \emptyset \implies \prod_{n \in \bm{N}} A_n \neq \emptyset
\end{align*}
を示せ.
}{
\item $(A_1, A_2, \cdots, A_n)$ は 写像 $A$ によって各 $n \in \bm{N}$ を写した先 $A(1), A(2), \cdots, A(n)$ を表す.
ここで, 写像の相等条件を考えれば, $(A_1, A_2, \cdots, A_n) = (B_1, B_2, \cdots, B_n) \implies A_1 = B_1 \land A_2 = B_2 \land \cdots \land A_n = B_n$
となり $n$-対の性質を持つ.\footnote{これにより, $\prod_{n \in \bm{N}}A_n$ は直積集合 $A_1 \times A_2 \times \cdots \times A_n$ と同一視される.}
\item $R_x = \{y \in A\ |\ \langle x, y\rangle \in R \}$ とすれば, DCの仮定より $R_x \neq \emptyset$ となるから, ACから $\prod_{x \in A}R_x \neq \emptyset$, 成り立つ.
すなわち, $\exists g: A \to A,\ (\forall x \in A)[\langle x, g(x)\rangle \in R_x]$ が成り立つ. そこで, $A$ の任意の元を $x_0$ とし, $x_n = g(x_{n-1})$ と
帰納的に $(x_n)_{n \in \bm{N}}$ を作れば, $\forall n \in \bm{N},\ \langle x_n, x_{n+1} \rangle \in R$ となる. したがって, $f(n) = x_n$ となるように $f$ を
定めれば, $\langle f(n), f(n+1) \rangle \in R$ となる.
\item $P = \biggl\{\ p\ \bigm|\ (\exists\ n \in \bm{N})\ \left[p : \{0, 1, \cdots, n\} \to \bigcup_{i = 0}^n A_n\ \land \ (\forall i \in \{0, 1, \cdots, n\})\ [p(i) \in A_i]\right]\ \biggr\}$
を考える. $p$ は写像であるが, 写像は二項関係の特別な場合(すなわち $p \subset \{0, 1, \cdots, n\} \times \bigcup_{i = 0}^n A_n$ )であることに注意して
$R = \left\{\ \langle p, q\rangle \in P \times P\ \middle|\ p \subsetneq q\ \right\}$ と二項関係 $R$ を定めると, $pRq$ ならば $q$ は写像として $p$ の真の拡大となる.
今, $\forall n \in \bm{N},\ A_n \neq \emptyset$ より, $\forall p \in P,\ \exists q \in P,\ \langle p, q \rangle \in R$ が成り立つ($A_{n + 1}$ から元を 1つ選んで拡大すればよい)
よって, DCより $f : \bm{N} \to P$ で $\forall n \in \bm{N},\ \langle f(n), f(n+1) \rangle \in R$ となるものが存在する. ここで $f(n) \subset \bm{N} \times P$ に
であることに注意して, $a = \bigcup_{n \in \bm{N}}f(n)$ とすれば, 任意の $n$ に対して, $\text{dom}(f(n)) \subsetneq \text{dom}(f(n+1))$ より
$\forall n \in \bm{N},\ n \in \text{dom}(f(n))$ となるから $\bm{N} \subset \bigcup_{n \in \bm{N}}\text{dom}(f(n))$ となる. また, $\forall n \in \bm{N},\ \text{dom}(f(n)) \subset \bm{N}$
より, $\bigcup_{n \in \bm{N}}\text{dom}(f(n)) \subset \bm{N}$ となるから, $\bigcup_{n \in \bm{N}}\text{dom}(f(n)) = \bm{N}$ となる. ここで
$\text{dom}(a) = \bigcup_{n \in \bm{N}}\text{dom}(f(n))$ より, $\text{dom}(a) = \bm{N}$ となる.
さらに, 集合 $P$ の定義より $\forall n \in \bm{N},\ a(n) \in A_n$ となる. 以上より $a \in \prod_{n \in \bm{N}}A_n$ となり, $\prod_{n \in \bm{N}}A_n \neq \emptyset$ となる. 
}
\end{nmprob}



\begin{nmprob}
選択公理の簡単な応用例.
\pbenumex[ac]{
    $f$ を $A$ から $B$ への写像とするとき, 次を示せ. ただし, $I_X$ は $X$ から $X$ への恒等写像を表すものとする.
}{
\item $f$ が全射 $\iff$ $f\circ g = I_B$ となるような写像 $g: B \to A$ が存在する
\item $f$ が単射 $\iff$ $h\circ f = I_A$ となるような写像 $h: A \to B$ が存在する
\item $A$ から $B$ への単射が存在する $\iff$ $B$ から $A$ への全射が存在する
}{
\item
\begin{enumerate}
    \item $\implies$\\
    $f$ は全射より, $\forall b \in B,\ f^{-1}\{b\} \neq \emptyset$ となる. よって, 選択公理より $g \in \prod_{b \in B}f^{-1}\{b\}$ となる $g : B \to A$ が存在する.
    この $g$ は 任意の $b \in B$ に対して, $g(b) \in f^{-1}\{b\}$ となるから, $\forall b \in B,\ f(g(b)) = b$ となるから $f \circ g = I_B$ を満たす.
    \item $\impliedby$\\
    $f \circ g = I_B$ となる $g : B \to A$ の存在を仮定すると, $\forall b \in B,\ f(g(b)) = b$ となる. $g(b) \in A$ より, $f$ は全射となる. 
\end{enumerate}
以上より, 「 $f$ が全射 $\iff$ $f\circ g = I_B$ となるような写像 $g: B \to A$ が存在する」が成り立つ.
\item
\begin{enumerate}
    \item $\implies$\\
    $f$ の終域を $B$ から $\text{rng}(f)$ へ縮小すると $f$ は全単射となる. このとき, 逆写像 $f^{-1} : \text{rng}(f) \to A$ が存在し, 今 $a \in A$ を適当にとり, $h : B \to A$ を
    \begin{align*}
        h(b) =
        \begin{cases}
            a & (y \in B - \text{rng}(f)\text{のとき})\\
            f^{-1}(b) & (y \in \text{rng}(f)\text{のとき})
        \end{cases}
    \end{align*}
    のように定めれば, $h \circ f = I_A$ を満たす.
    \item $\impliedby$\\
    $f(a) = f(a^{'}) \implies a = h(f(a)) = h(f(a^{'})) = a^{'}$ より $f$ は単射となる.
\end{enumerate}
以上より, 「 $f$ が単射 $\iff$ $h\circ f = I_A$ となるような写像 $h: A \to B$ が存在する 」が成り立つ.
\item
\begin{enumerate}
    \item $\implies$\\
    $A$ から $B$ への単射を $\phi$ とすれば(2)より, $\psi \circ \phi = I_A$ となる写像 $\psi : A \to B$ が存在する.
    この $\psi$ は(1)より全射である((1)において $B$ と $A$ を逆にみればよい)
    \item $\impliedby$\\
    $B$ から $A$ への全射を $\psi$ とすれば(1)より, $\psi \circ \phi = I_A$ となる写像 $\phi : A \to B$ が存在する
    ((1)において $B$ と $A$ を逆にみればよい)ここで, この $\phi$ は(2)より単射である.
\end{enumerate}
以上より, 「$A$ から $B$ への単射が存在する $\iff$ $B$ から $A$ への全射が存在する」が成り立つ.
}
\end{nmprob}



\begin{nmprob}
問題というよりかは確認.
\makeatletter\tagsleft@true\makeatother
\pbenum{
\item $R$ を集合 $A$ 上の同値関係とする. $A$ の元 $x$ に対して, 集合 $\{ y\in A\ |\ xRy \}$ を $x$ の $R$ による同値類といい, 以後 $[x]_R$ と表す.
また, 同値類全体の集合を $A$ の $R$ による商集合といい $A/R$ と表す. ここで, $A/R$ によって定まる集合族が $A$ の直和分割となることを示せ. ただし,
ある集合 $X$ の部分集合族 $(X_i)_{i\in I}$ が次の三条件を満たすとき, $(X_i)_{i\in I}$ は $X$ の直和分割と呼ばれる. 
\begin{align}
    &\forall i\in I,\ X_i \neq \emptyset\\
    &\bigcup_{i\in I} X_i = X\\
    &i, j \in I \land i \neq j \implies X_i \cap X_j = \emptyset
\end{align}
\setcounter{equation}{0}
\item 空でない集合 $A$ のある集合族 $(C_i)_{i\in I}$ が $A$ の直和分割であるとする. 任意の $A$ の元 $x$ に対し, $x \in C_i$ となる $i \in I$ がただ一つ存在することを示せ.
\item (2)において, 関係 $R$ を $xRy \iff \exists i \in I,\ x \in C_i \land y \in C_i$ により定めると, $R$ は $A$ 上の同値関係となることを示せ.
}{
\item $A/R$ によって定まる族を $(X)_{X\in A/R}$ で表すことにすると
\begin{enumerate}
    \item 任意の $X$ を一つとり, その代表元\footnote[1]{$X \in A/R$ に対する $X$ の元}を $x$ とすると, $x \in X$ より $X \neq \emptyset$ となる
    ($A/R \coloneqq \{ [x]_R\ |\ x \in A\}$ としているので, 代表元の存在が保証される)
    \item 任意の $x \in A$ に対し, $x \in [x]_R \in A/R$ より, $x \in X$ となる $X \in A/R$ が存在することから $\bigcup_{X\in A/R} X = A$ となる.
    \item 任意の $X, Y \in A/R$ に対して, $X \neq Y$ のとき, $X \cap Y \neq \emptyset$ と仮定する. $c \in X \cap Y$ とすれば
    $xRc$ かつ $yRc$ となる($x, y$ はそれぞれ $X, Y$ の代表元とする) ここで, $R$ は同値関係より $xRy$ となり $y \in [x]_R\ (= X)$ となる. 
    $y \in [x]_R$ であるとき, $[x]_R = [y]_R$ であるので, $X = Y$ となるが, これは $X \neq Y$ に矛盾. よって $X \cap Y = \emptyset$ となる.
\end{enumerate}
以上より, $A/R$ によって定まる集合族は $A$ の直和分割となる.
\item まず, 直和分割の条件(2) より, $\bigcup_{i \in I}C_i = A$ より, 任意の $A$ の元 $x$ に対して $x \in C_i$ となる $i \in I$ が存在する.\\
また, $x \in C_i$ かつ $x \in C_j$ となる $i, j\ (i \neq j)$ が存在すると仮定すると, $x \in C_i \cap C_j$ より, 直和分割の条件(3)に矛盾することから, 
$x \in C_i$ かつ $x \in C_j$ となる $i, j\ (i \neq j)$ は存在しない.\\
以上より, 任意の $A$ の元 $x$ に対し, $x \in C_i$ となる $i \in I$ がただ一つ存在する.
\item \
\begin{enumerate}
    \item (2) より 任意の $x \in A$ に対して, $x \in C_i$ となる $i$ が存在することから, $\forall x \in A,\ xRx$ で反射律が成立.
    \item 対称律は明らかに成立.
    \item $xRy \land yRz$ が成り立つとすると, $(\exists i \in I,\ x \in C_i \land y \in C_i) \land (\exists j \in I,\ y \in C_j \land z \in C_j)$ が成り立つ.
    ここで $i \neq j$ とすると, $y \in C_i \cap C_j$ より, $(C_i)_{i\in I}$ が直和分割であることに反するから, $i = j$ となる. これより, $\exists k \in I,\ x \in C_k \land z \in C_k$ が
    成り立つので $xRz$ で推移律が成立.
\end{enumerate}
以上より, $R$ は $A$ 上の同値関係となる.
}
\makeatletter\tagsleft@false\makeatother
\end{nmprob}



\begin{nmprob}
Bernstein の定理を理解
\pbenumex{
集合 $A$ から $B$ への全単射($B$ から $A$ への全単射でもある)が存在するとき $A$ と $B$ は対等であるといい, $A \sim B$ で表す.
このとき, 次の問に答えよ.
}{
\item $A$ から $B$ への単射が存在し, かつ $B$ から $A$ への単射が存在すれば $A \sim B$ となることを示せ.
\item $A$ から $B$ への全射が存在し, かつ $B$ から $A$ への全射が存在すれば $A \sim B$ となることを示せ.
\item $A \sim B^{'}$ となるような $B^{'} \subset B$ が存在し, かつ $B \sim A^{'}$ となるような $A^{'} \subset A$ が存在すれば $A \sim B$ となることを示せ. 
}{
\item $f$ を $A$ から $B$ への単射, $g$ を $B$ から $A$ への単射とする.\\
このとき, $B_0 = B - f[A]$ とし, $A_n = g[B_{n-1}], B_n = f[A_n]$ として,
$A$ の部分集合族 $(A_n)_{n \in \{1,2,3,\ldots \}}$, $B$ の部分集合族 $(B_n)_{n \in \{0,1,2,\ldots \}}$ を定める. また, $A - \bigcup_{n=1}^{\infty}A_n = A^*, B - \bigcup_{n=0}^{\infty}B_n = B^*$ とする.\\
ここで,
\begin{align*}
    f[A^*] &= f[A] - f\left[\bigcup_{n=1}^{\infty}A_n\right]\\
    &= (B - B_0) -f\left[\bigcup_{n=1}^{\infty}A_n\right]\hspace{1zw}\text{($\because$\ $f$ が単射より)}\\
    &= B - \left(B_0 \cup f\left[\bigcup_{n=1}^{\infty}A_n\right]\right)\\
    &= B - \left(B_0 \cup \bigcup_{n=1}^{\infty} f[A_n]\right)\\
    &= B - \left(B_0 \cup \bigcup_{n=1}^{\infty} B_n\right)\hspace{1zw}\text{($\because$\ $f[A_n] = B_n$ より)}\\
    &= B - \bigcup_{n=0}^{\infty}B_n\\
    &= B^*
\end{align*}
また,
\begin{align*}
    g\left[\bigcup_{n=0}^{\infty}B_n\right] &= \bigcup_{n=1}^{\infty}g[B_{n-1}]\\
    &= \bigcup_{n=1}^{\infty}A_n\hspace{1zw}\text{($\because$\ $g[B_{n-1}] = A_n$ より)}\\
\end{align*}
となる. よって $f$ の定義域を $A^*$, 終集合を $B^*$ に変えた写像と, $g$ の定義域を $\bigcup_{n=0}^{\infty}B_n$, 終集合を $\bigcup_{n=1}^{\infty}A_n$ に変えた写像は, それぞれ全単射となる.
そこで, $A$ から $B$ への写像 $F$ を $a \in A^*$ のとき $F(a) = f(a)$, $a \in \bigcup_{n=1}^{\infty}A_n$ のときは $F(a) = g^{-1}(a)$ とすれば $F$ は全単射となり, $A \sim B$ となる.
\item 問\ref{sec:set}-\ref{pname:ac} (3)よりすぐに示せる.
\item $A$ から $B^{'}$ への全単射の写像の終集合を $B$ に拡大すれば, その写像は単射となる. 同様に $B$ から $A^{'}$ への全単射も $B$ から $A$ への単射にすることが可能である.
よって(1) とから $A \sim B$ となる.
}
\newpage
\hspace{-3zw}{\color{forestgreen}●●\ Bernstein の定理の証明の補足\ ●●}

Bernstein の定理の証明は次の図をイメージすると良い.
\begin{figure}[htbp]
    \centering
    \begin{tikzpicture}
        \coordinate (O);
        \coordinate (A) at ($(O) + (3, 4)$);
        \coordinate (LA) at ($(O) + (1, 3)$);
        \draw[rounded corners] (O) rectangle($(O) + (6, 4)$);
        \path [pattern=north east lines, pattern color=magenta, opacity=.3] (O) rectangle ($(O) + (6, 4)$);
        \draw (A) node [above] {$A$};
        \draw ($(O) + (0, 2)$) -- ++(2, 0) -- ++(0, 2);
        \draw ($(O) + (2, 2)$) -- ++(2, 0) -- ++(0, 2);
        \draw ($(O) + (2, 2)$) -- ++(0, -2);
        \filldraw[fill=green!70!red, fill opacity=.5] ($(O) + (4, 2)$) -- ++(0, -2) [rounded corners] -- ++(2, 0) [sharp corners] -- ++(0, 2) -- ++(-2, 0) --cycle;
        \draw (LA) node[above] (A1) {$g[B_0] = A_1$};
        \draw ($(LA) + (2, 0)$) node[above] (A2) {$g[B_1] = A_2$};
        \draw ($(LA) + (4, 0)$) node[above] (A3) {$g[B_2] = A_3$};
        \draw ($(LA) + (0, -2)$) node[above] (A4) {$g[B_3] = A_4$};
        \draw ($(LA) + (2, -2)$) node[above] {$\ldots$};
        \draw ($(LA) + (4, -2)$) node[above] {$A^{*}$};

        \coordinate (O2) at ($(O) + (8, 0)$);
        \coordinate (B) at ($(O2) + (3, 4)$);
        \coordinate (LB) at ($(O2) + (1, 3)$);

        \draw[rounded corners] (O2) rectangle($(O2) + (6, 4)$);
        \path [pattern=north east lines, pattern color=magenta, opacity=.3] (O2) -- ++(0, 2) -- ++(2, 0) -- ++(0, 2) -- ++(4, 0) -- ++(0, -4) -- ++(-6, 0) --cycle;
        \draw (B) node [above] {$B$};
        \draw ($(O2) + (0, 2)$) -- ++(2, 0) -- ++(0, 2);
        \draw ($(O2) + (2, 2)$) -- ++(2, 0) -- ++(0, 2);
        \draw ($(O2) + (2, 2)$) -- ++(0, -2);
        \filldraw[fill=green!70!red, fill opacity=.5] ($(O2) + (4, 2)$) -- ++(0, -2) [rounded corners] -- ++(2, 0) [sharp corners] -- ++(0, 2) -- ++(-2, 0) --cycle;
        \node[above=-4pt of LB, align=left] (B0) {$B - f[A]$\\$ = B_0$};
        \draw ($(LB) + (2, 0)$) node[above] (B1) {$f[A_1] = B_1$};
        \draw ($(LB) + (4, 0)$) node[above] (B2) {$f[A_2] = B_2$};
        \draw ($(LB) + (0, -2)$) node[above] (B3) {$f[A_3] = B_3$};
        \draw ($(LB) + (2, -2)$) node[above] {$\ldots$};
        \draw ($(LB) + (4, -2)$) node[above] {$B^{*}$};

        \draw [bend right,distance=50,blue,->] (A1) to node [above] {$g^{-1}$} (B0);
        \draw [bend left,distance=50,blue,->] (A2) to node [above] {$g^{-1}$} (B1);
        \draw [bend right,distance=50,blue,->] (A3) to node [above] {$g^{-1}$} (B2);
        \draw [bend right,distance=70,blue,->] (A4) to node [below] {$g^{-1}$} (B3);
        \draw [bend right,distance=70,red,->] ($(LA) + (4, -2)$) to node [below] {$f$} ($(LB) + (4, -2)$);

    \end{tikzpicture}
    \caption{Bernstein の定理のイメージ}
\end{figure}

上図において $g^{-1}, f$ をまとめた写像が全単射だと証明している. なお, マゼンダ色の領域は $A \to f[A]$ の対応を表す. また, 図中における各集合の共通部分が無いのは, $f, g$ の単射性による.
\end{nmprob}
\setcounter{figure}{0}

